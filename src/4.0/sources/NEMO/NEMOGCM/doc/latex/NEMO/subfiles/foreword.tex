\documentclass[../main/NEMO_manual]{subfiles}

\begin{document}

% ================================================================
% Chapter Foreword
% ================================================================
\chapter*{Foreword}

% ================================================================
% Abstract
% ================================================================
\section*{Abstract}

The ocean engine of NEMO (Nucleus for European Modelling of the Ocean) is a primitive equation model adapted to
regional and global ocean circulation problems.
It is intended to be a flexible tool for studying the ocean and its interactions with the others components of
the earth climate system over a wide range of space and time scales.

Prognostic variables are the three-dimensional velocity field, a non-linear sea surface height,
the \textit{Conservative} Temperature and the \textit{Absolute} Salinity.
In the horizontal direction, the model uses a curvilinear orthogonal grid and in the vertical direction,
a full or partial step $z$-coordinate, or $s$-coordinate, or a mixture of the two.
The distribution of variables is a three-dimensional Arakawa C-type grid.
Various physical choices are available to describe ocean physics, including TKE, and GLS vertical physics.

Within NEMO, the ocean is interfaced with a sea-ice model (SI$^3$)
 %or \href{https://github.com/CICE-Consortium/CICE}{CICE}),
passive tracer and biogeochemical models (TOP-PISCES) and,
via the \href{https://portal.enes.org/oasis}{OASIS} coupler, with several atmospheric general circulation models.
It also support two-way grid embedding via the \href{http://agrif.imag.fr}{AGRIF} software.


% ================================================================
% Disclaimer
% ================================================================
\section*{Disclaimer}

Like all components of NEMO, the ocean component is developed under
the \href{http://www.cecill.info}{CECILL license}, which is a French adaptation of the GNU GPL
(General Public License).
Anyone may use it freely for research purposes, and is encouraged to communicate back to the NEMO team
its own developments and improvements.

The model and the present document have been made available as a service to the community.
We cannot certify that the code and its manual are free of errors.
Bugs are inevitable and some have undoubtedly survived the testing phase.
Users are encouraged to bring them to our attention.

The author assumes no responsibility for problems, errors, or incorrect usage of NEMO.

% ================================================================
% Citation
% ================================================================
\section*{Citation}

Reference for papers and other publications is as follows:
\vspace{0.5cm}

{\sffamily
NEMO ocean engine,
Madec Gurvan and NEMO System Team, NEMO Consortium,
Issue 27, Notes du Pôle de modélisation de l'Institut Pierre-Simon Laplace (IPSL), ISSN 1288-1619,
\href{http://doi.org/10.5281/zenodo.1464816}{doi:10.5281/zenodo.1464816}
}

% ================================================================
% External resources
% ================================================================
\section*{External resources}

Additional information can be found on the \href{http://www.nemo-ocean.eu}{website} of the project and
the \href{http://forge.ipsl.jussieu.fr/nemo}{forge platform} of the source code.
A \href{http://listes.ipsl.fr/sympa/info/nemo-newsletter}{newsletter list} is also open for subscription to
receive top-down communication from the consortium (announcements, job opportunities, ...).

\biblio

\pindex

\end{document}
