\documentclass[../main/NEMO_manual]{subfiles}

\begin{document}
% ================================================================
% Chapter I/O & Diagnostics
% ================================================================
\chapter{Output and Diagnostics (IOM, DIA, TRD, FLO)}
\label{chap:DIA}

\minitoc

\newpage

% ================================================================
%       Old Model Output 
% ================================================================
\section{Old model output (default)}
\label{sec:DIA_io_old}

The model outputs are of three types: the restart file, the output listing, and the diagnostic output file(s).
The restart file is used internally by the code when the user wants to start the model with
initial conditions defined by a previous simulation.
It contains all the information that is necessary in order for there to be no changes in the model results
(even at the computer precision) between a run performed with several restarts and
the same run performed in one step.
It should be noted that this requires that the restart file contains two consecutive time steps for
all the prognostic variables, and that it is saved in the same binary format as the one used by the computer that
is to read it (in particular, 32 bits binary IEEE format must not be used for this file).

The output listing and file(s) are predefined but should be checked and eventually adapted to the user's needs.
The output listing is stored in the $ocean.output$ file.
The information is printed from within the code on the logical unit $numout$.
To locate these prints, use the UNIX command "\textit{grep -i numout}" in the source code directory.

By default, diagnostic output files are written in NetCDF format.
Since version 3.2, when defining \key{iomput}, an I/O server has been added which
provides more flexibility in the choice of the fields to be written as well as how
the writing work is distributed over the processors in massively parallel computing.
A complete description of the use of this I/O server is presented in the next section.

By default, \key{iomput} is not defined,
NEMO produces NetCDF with the old IOIPSL library which has been kept for compatibility and its easy installation.
However, the IOIPSL library is quite inefficient on parallel machines and, since version 3.2,
many diagnostic options have been added presuming the use of \key{iomput}.
The usefulness of the default IOIPSL-based option is expected to reduce with each new release.
If \key{iomput} is not defined, output files and content are defined in the \mdl{diawri} module and
contain mean (or instantaneous if \key{diainstant} is defined) values over a regular period of
nn\_write time-steps (namelist parameter). 

%\gmcomment{                    % start of gmcomment

% ================================================================
% Diagnostics
% ================================================================
\section{Standard model output (IOM)}
\label{sec:DIA_iom}

Since version 3.2, iomput is the NEMO output interface of choice.
It has been designed to be simple to use, flexible and efficient.
The two main purposes of iomput are: 

\begin{enumerate}
\item
  The complete and flexible control of the output files through external XML files adapted by
  the user from standard templates.
\item
  To achieve high performance and scalable output through the optional distribution of
  all diagnostic output related tasks to dedicated processes.
\end{enumerate}

The first functionality allows the user to specify, without code changes or recompilation, 
aspects of the diagnostic output stream, such as:

\begin{itemize}
\item
  The choice of output frequencies that can be different for each file (including real months and years).
\item
  The choice of file contents; includes complete flexibility over which data are written in which files
  (the same data can be written in different files).
\item
  The possibility to split output files at a chosen frequency.
\item
  The possibility to extract a vertical or an horizontal subdomain.
\item
  The choice of the temporal operation to perform, \eg: average, accumulate, instantaneous, min, max and once.
\item
  Control over metadata via a large XML "database" of possible output fields.
\end{itemize}

In addition, iomput allows the user to add in the code the output of any new variable (scalar, 2D or 3D)
in a very easy way.
All details of iomput functionalities are listed in the following subsections.
Examples of the XML files that control the outputs can be found in: \path{NEMOGCM/CONFIG/ORCA2_LIM/EXP00/iodef.xml},
\path{NEMOGCM/CONFIG/SHARED/field_def.xml} and \path{NEMOGCM/CONFIG/SHARED/domain_def.xml}. \\

The second functionality targets output performance when running in parallel (\key{mpp\_mpi}).
Iomput provides the possibility to specify N dedicated I/O processes (in addition to the NEMO processes)
to collect and write the outputs.
With an appropriate choice of N by the user, the bottleneck associated with the writing of
the output files can be greatly reduced.

In version 3.6, the iom\_put interface depends on
an external code called \href{https://forge.ipsl.jussieu.fr/ioserver/browser/XIOS/branchs/xios-1.0}{XIOS-1.0} 
(use of revision 618 or higher is required).
This new IO server can take advantage of the parallel I/O functionality of NetCDF4 to
create a single output file and therefore to bypass the rebuilding phase.
Note that writing in parallel into the same NetCDF files requires that your NetCDF4 library is linked to
an HDF5 library that has been correctly compiled (\ie with the configure option $--$enable-parallel).
Note that the files created by iomput through XIOS are incompatible with NetCDF3.
All post-processsing and visualization tools must therefore be compatible with NetCDF4 and not only NetCDF3.

Even if not using the parallel I/O functionality of NetCDF4, using N dedicated I/O servers,
where N is typically much less than the number of NEMO processors, will reduce the number of output files created.
This can greatly reduce the post-processing burden usually associated with using large numbers of NEMO processors.
Note that for smaller configurations, the rebuilding phase can be avoided,
even without a parallel-enabled NetCDF4 library, simply by employing only one dedicated I/O server.

\subsection{XIOS: Reading and writing restart file}

XIOS may be used to read single file restart produced by NEMO. Currently only the variables written to 
file \forcode{numror} can be handled by XIOS. To activate restart reading using XIOS, set \np{ln\_xios\_read}\forcode{ = .true. }
in \textit{namelist\_cfg}. This setting will be ignored when multiple restart files are present, and default NEMO 
functionality will be used for reading. There is no need to change iodef.xml file to use XIOS to read 
restart, all definitions are done within the NEMO code. For high resolution configuration, however, 
there may be a need to add the following line in iodef.xml (xios context):

\begin{xmllines}
<variable id="recv_field_timeout"        type="double">1800</variable>
\end{xmllines}

This variable sets timeout for reading. 

If XIOS is to be used to read restart from file generated with an earlier NEMO version (3.6 for instance),
dimension \forcode{z} defined in restart file must be renamed to \forcode{nav_lev}.\\

XIOS can also be used to write NEMO restart. A namelist parameter \np{nn\_wxios} is used to determine the 
type of restart NEMO will write. If it is set to 0, default NEMO functionality will be used - each 
processor writes its own restart file; if it is set to 1 XIOS will write restart into a single file; 
for \np{nn\_wxios = 2} the restart will be written by XIOS into multiple files, one for each XIOS server. 
Note, however, that \textbf{NEMO will not read restart generated by XIOS when \np{nn\_wxios = 2}}. The restart will 
have to be rebuild before continuing the run. This option aims to reduce number of restart files generated by NEMO only, 
and may be useful when there is a need to change number of processors used to run simulation. 

If an additional variable must be written to a restart file, the following steps are needed:
\begin{description}
   \item[step 1:] add variable name to a list of restart variables (in subroutine \rou{iom\_set\_rst\_vars,} \mdl{iom}) and 
define correct grid for the variable (\forcode{grid_N_3D} - 3D variable, \forcode{grid_N} - 2D variable, \forcode{grid_vector} - 
1D variable, \forcode{grid_scalar} - scalar),
   \item[step 2:] add variable to the list of fields written by restart.  This can be done either in subroutine 
\rou{iom\_set\_rstw\_core} (\mdl{iom}) or by calling  \rou{iom\_set\_rstw\_active} (\mdl{iom}) with the name of a variable 
as an argument. This convention follows approach for writing restart using iom, where variables are 
written either by \rou{rst\_write} or by calling \rou{iom\_rstput} from individual routines.
\end{description}


An older versions of XIOS do not support reading functionality. It's recommended to use at least XIOS2@1451.


\subsection{XIOS: XML Inputs-Outputs Server}

\subsubsection{Attached or detached mode?}

Iomput is based on \href{http://forge.ipsl.jussieu.fr/ioserver/wiki}{XIOS},
the io\_server developed by Yann Meurdesoif from IPSL.
The behaviour of the I/O subsystem is controlled by settings in the external XML files listed above.
Key settings in the iodef.xml file are the tags associated with each defined file.

\xmlline|<variable id="using_server" type="bool"></variable>|

The {\tt using\_server} setting determines whether or not the server will be used in \textit{attached mode}
(as a library) [{\tt> false <}] or in \textit{detached mode}
(as an external executable on N additional, dedicated cpus) [{\tt > true <}].
The \textit{attached mode} is simpler to use but much less efficient for massively parallel applications.
The type of each file can be either ''multiple\_file'' or ''one\_file''.

In \textit{attached mode} and if the type of file is ''multiple\_file'',
then each NEMO process will also act as an IO server and produce its own set of output files.
Superficially, this emulates the standard behaviour in previous versions.
However, the subdomain written out by each process does not correspond to
the \forcode{jpi x jpj x jpk} domain actually computed by the process (although it may if \forcode{jpni=1}).
Instead each process will have collected and written out a number of complete longitudinal strips.
If the ''one\_file'' option is chosen then all processes will collect their longitudinal strips and
write (in parallel) to a single output file.

In \textit{detached mode} and if the type of file is ''multiple\_file'',
then each stand-alone XIOS process will collect data for a range of complete longitudinal strips and
write to its own set of output files.
If the ''one\_file'' option is chosen then all XIOS processes will collect their longitudinal strips and
write (in parallel) to a single output file. 
Note running in detached mode requires launching a Multiple Process Multiple Data (MPMD) parallel job.
The following subsection provides a typical example but the syntax will vary in different MPP environments.

\subsubsection{Number of cpu used by XIOS in detached mode}

The number of cores used by the XIOS is specified when launching the model.
The number of cores dedicated to XIOS should be from \texttildelow1/10 to \texttildelow1/50 of the number of 
cores dedicated to NEMO.
Some manufacturers suggest using O($\sqrt{N}$) dedicated IO processors for N processors but
this is a general recommendation and not specific to NEMO.
It is difficult to provide precise recommendations because the optimal choice will depend on
the particular hardware properties of the target system 
(parallel filesystem performance, available memory, memory bandwidth etc.)
and the volume and frequency of data to be created.
Here is an example of 2 cpus for the io\_server and 62 cpu for nemo using mpirun:
\cmd|mpirun -np 62 ./nemo.exe : -np 2 ./xios_server.exe|

\subsubsection{Control of XIOS: the context in iodef.xml}

As well as the {\tt using\_server} flag, other controls on the use of XIOS are set in the XIOS context in iodef.xml.
See the XML basics section below for more details on XML syntax and rules.

\begin{table}
  \scriptsize
  \begin{tabularx}{\textwidth}{|lXl|}
    \hline
    variable name                                                           &
    description                                                             &
    example  \\
    \hline
    \hline
    buffer\_size                                                            &
    buffer size used by XIOS to send data from NEMO to XIOS.
    Larger is more efficient.
    Note that needed/used buffer sizes are summarized at the end of the job &
    25000000 \\
    \hline
    buffer\_server\_factor\_size                                            &
    ratio between NEMO and XIOS buffer size.
    Should be 2.                                                            &
    2        \\
    \hline
    info\_level                                                             &
    verbosity level (0 to 100)                                              &
    0        \\
    \hline
    using\_server                                                           &
    activate attached(false) or detached(true) mode                         &
    true     \\
    \hline
    using\_oasis                                                            &
    XIOS is used with OASIS(true) or not (false)                            &
    false    \\
    \hline
    oasis\_codes\_id                                                        &
    when using oasis, define the identifier of NEMO in the namcouple.
    Note that the identifier of XIOS is xios.x                              &
    oceanx   \\
    \hline
  \end{tabularx}
\end{table}

\subsection{Practical issues}

\subsubsection{Installation}

As mentioned, XIOS is supported separately and must be downloaded and compiled before it can be used with NEMO.
See the installation guide on the \href{http://forge.ipsl.jussieu.fr/ioserver/wiki}{XIOS} wiki for help and guidance.
NEMO will need to link to the compiled XIOS library.
The \href{https://forge.ipsl.jussieu.fr/nemo/wiki/Users/ModelInterfacing/InputsOutputs#Inputs-OutputsusingXIOS}
{XIOS with NEMO} guide provides an example illustration of how this can be achieved.

\subsubsection{Add your own outputs}

It is very easy to add your own outputs with iomput.
Many standard fields and diagnostics are already prepared (\ie, steps 1 to 3 below have been done) and
simply need to be activated by including the required output in a file definition in iodef.xml (step 4).
To add new output variables, all 4 of the following steps must be taken.

\begin{enumerate}
\item[1.]
  in NEMO code, add a \forcode{CALL iom\_put( 'identifier', array )} where you want to output a 2D or 3D array.
\item[2.]
  If necessary, add \forcode{USE iom ! I/O manager library} to the list of used modules in
  the upper part of your module.
\item[3.]
  in the field\_def.xml file, add the definition of your variable using the same identifier you used in the f90 code
  (see subsequent sections for a details of the XML syntax and rules).
  For example:

\begin{xmllines}
<field_definition>
	<field_group id="grid_T" grid_ref="grid_T_3D"> <!-- T grid -->
	...
		<field id="identifier" long_name="blabla" ... />   
		...
</field_definition> 
\end{xmllines}

Note your definition must be added to the field\_group whose reference grid is consistent with the size of
the array passed to iomput.
The grid\_ref attribute refers to definitions set in iodef.xml which, in turn,
reference grids and axes either defined in the code
(iom\_set\_domain\_attr and iom\_set\_axis\_attr in \mdl{iom}) or defined in the domain\_def.xml file.
\eg:

\begin{xmllines}
<grid id="grid_T_3D" domain_ref="grid_T" axis_ref="deptht"/>
\end{xmllines}

Note, if your array is computed within the surface module each \np{nn\_fsbc} time\_step,
add the field definition within the field\_group defined with the id "SBC":
\xmlcode{<field_group id="SBC" ...>} which has been defined with the correct frequency of operations
(iom\_set\_field\_attr in \mdl{iom})
\item[4.]
  add your field in one of the output files defined in iodef.xml
  (again see subsequent sections for syntax and rules)

\begin{xmllines}
<file id="file1" .../>   
...
	<field field_ref="identifier" />   
	...
</file>   
\end{xmllines}

\end{enumerate}

\subsection{XML fundamentals}

\subsubsection{ XML basic rules}

XML tags begin with the less-than character ("$<$") and end with the greater-than character ("$>$").
You use tags to mark the start and end of elements, which are the logical units of information in an XML document.
In addition to marking the beginning of an element, XML start tags also provide a place to specify attributes.
An attribute specifies a single property for an element, using a name/value pair, for example:
\xmlcode{<a b="x" c="y" d="z"> ... </a>}.
See \href{http://www.xmlnews.org/docs/xml-basics.html}{here} for more details.

\subsubsection{Structure of the XML file used in NEMO}

The XML file used in XIOS is structured by 7 families of tags:
context, axis, domain, grid, field, file and variable.
Each tag family has hierarchy of three flavors (except for context):

\begin{table}
  \scriptsize
  \begin{tabular*}{\textwidth}{|p{0.15\textwidth}p{0.4\textwidth}p{0.35\textwidth}|}
    \hline
    flavor  & description                                                                 &
                                                                                            example                          \\
    \hline
    \hline
    root    & declaration of the root element that can contain element groups or elements &
                                                                                            \xmlcode{<file_definition ... >} \\
    \hline
    group   & declaration of a group element that can contain element groups or elements  &
                                                                                            \xmlcode{<file_group      ... >} \\
    \hline
    element & declaration of an element that can contain elements                         &
                                                                                            \xmlcode{<file            ... >} \\
    \hline
  \end{tabular*}
\end{table}

Each element may have several attributes.
Some attributes are mandatory, other are optional but have a default value and other are completely optional.
Id is a special attribute used to identify an element or a group of elements.
It must be unique for a kind of element.
It is optional, but no reference to the corresponding element can be done if it is not defined.

The XML file is split into context tags that are used to isolate IO definition from
different codes or different parts of a code.
No interference is possible between 2 different contexts.
Each context has its own calendar and an associated timestep.
In \NEMO, we used the following contexts (that can be defined in any order):

\begin{table}
  \scriptsize
  \begin{tabular}{|p{0.15\textwidth}p{0.4\textwidth}p{0.35\textwidth}|}
    \hline
    context         &	description                                                                &
                                                                                                     example                              \\
    \hline
    \hline
    context xios    &	context containing information for XIOS                                    &
                                                                                                     \xmlcode{<context id="xios" ... >}   \\
    \hline
    context nemo    &	context containing IO information for NEMO (mother grid when using AGRIF)  & 
                                                                                                     \xmlcode{<context id="nemo" ... >}   \\
    \hline
    context 1\_nemo &	context containing IO information for NEMO child grid 1 (when using AGRIF) & 
                                                                                                     \xmlcode{<context id="1_nemo" ... >} \\
    \hline
    context n\_nemo &	context containing IO information for NEMO child grid n (when using AGRIF) & 
                                                                                                     \xmlcode{<context id="n_nemo" ... >} \\
    \hline
  \end{tabular}
\end{table}

\noindent The xios context contains only 1 tag:

\begin{table}
  \scriptsize
  \begin{tabular}{|p{0.15\textwidth}p{0.4\textwidth}p{0.35\textwidth}|}
    \hline
    context tag                                      &
                                                       description                                      &
                                                                                                          example                              \\
    \hline
    \hline
    variable\_definition                             &
                                                       define variables needed by XIOS.
                                                       This can be seen as a kind of namelist for XIOS. &
                                                                                                          \xmlcode{<variable_definition ... >} \\
    \hline
	\end{tabular}
\end{table}

\noindent Each context tag related to NEMO (mother or child grids) is divided into 5 parts 
(that can be defined in any order):

\begin{table}
  \scriptsize
  \begin{tabular}{|p{0.15\textwidth}p{0.4\textwidth}p{0.35\textwidth}|}
    \hline
    context tag        &	description                                                               &
                                                                                                       example                            \\
    \hline
    \hline
    field\_definition  &	define all variables that can potentially be outputted                    &
                                                                                                       \xmlcode{<field_definition ... >}  \\
    \hline
    file\_definition   &	define the netcdf files to be created and the variables they will contain &
                                                                                                       \xmlcode{<file_definition ... >}   \\
    \hline
    axis\_definition   &	define vertical axis                                                      &
                                                                                                       \xmlcode{<axis_definition ... >}   \\
    \hline
    domain\_definition &	define the horizontal grids                                               &
                                                                                                       \xmlcode{<domain_definition ... >} \\
    \hline
    grid\_definition   &	define the 2D and 3D grids (association of an axis and a domain)          &
                                                                                                       \xmlcode{<grid_definition ... >}   \\
    \hline
  \end{tabular}
\end{table}

\subsubsection{Nesting XML files}

The XML file can be split in different parts to improve its readability and facilitate its use.
The inclusion of XML files into the main XML file can be done through the attribute src:
\xmlline|<context src="./nemo_def.xml" />|
 
\noindent In NEMO, by default, the field and domain definition is done in 2 separate files:
\path{NEMOGCM/CONFIG/SHARED/field_def.xml} and \path{NEMOGCM/CONFIG/SHARED/domain_def.xml} that
are included in the main iodef.xml file through the following commands:
\begin{xmllines}
<field_definition src="./field_def.xml" />
<domain_definition src="./domain_def.xml" />
\end{xmllines}

\subsubsection{Use of inheritance}

XML extensively uses the concept of inheritance.
XML has a tree based structure with a parent-child oriented relation: all children inherit attributes from parent,
but an attribute defined in a child replace the inherited attribute value.
Note that the special attribute ''id'' is never inherited.
\\
\\
example 1: Direct inheritance.

\begin{xmllines}
<field_definition operation="average" >
	<field id="sst"                    />   <!-- averaged      sst --> 
	<field id="sss" operation="instant"/>   <!-- instantaneous sss --> 
</field_definition> 
\end{xmllines}

The field ''sst'' which is part (or a child) of the field\_definition will inherit the value ''average'' of
the attribute ''operation'' from its parent.
Note that a child can overwrite the attribute definition inherited from its parents.
In the example above, the field ''sss'' will for example output instantaneous values instead of average values.
\\
\\
example 2: Inheritance by reference.

\begin{xmllines}
<field_definition>
	<field id="sst" long_name="sea surface temperature" />
	<field id="sss" long_name="sea surface salinity"    />
</field_definition>
<file_definition>
	<file id="myfile" output_freq="1d" />   
		<field field_ref="sst"                            />  <!-- default def -->
		<field field_ref="sss" long_name="my description" />  <!-- overwrite   -->
	</file>
</file_definition> 
\end{xmllines}

Inherit (and overwrite, if needed) the attributes of a tag you are refering to.

\subsubsection{Use of groups}

Groups can be used for 2 purposes.
Firstly, the group can be used to define common attributes to be shared by the elements of
the group through inheritance.
In the following example, we define a group of field that will share a common grid ''grid\_T\_2D''.
Note that for the field ''toce'', we overwrite the grid definition inherited from the group by ''grid\_T\_3D''.

\begin{xmllines}
<field_group id="grid_T" grid_ref="grid_T_2D">
	<field id="toce" long_name="temperature"             unit="degC" grid_ref="grid_T_3D"/>
	<field id="sst"  long_name="sea surface temperature" unit="degC"                     />
	<field id="sss"  long_name="sea surface salinity"    unit="psu"                      />
	<field id="ssh"  long_name="sea surface height"      unit="m"                        />
	...
\end{xmllines}

Secondly, the group can be used to replace a list of elements.
Several examples of groups of fields are proposed at the end of the file \path{CONFIG/SHARED/field_def.xml}.
For example, a short list of the usual variables related to the U grid:

\begin{xmllines}
<field_group id="groupU" >
	<field field_ref="uoce"  />
	<field field_ref="suoce" />
	<field field_ref="utau"  />
</field_group>
\end{xmllines}

that can be directly included in a file through the following syntax:

\begin{xmllines}
<file id="myfile_U" output_freq="1d" />
	<field_group group_ref="groupU" />
	<field field_ref="uocetr_eff"   />  <!-- add another field -->
</file>   
\end{xmllines}

\subsection{Detailed functionalities}

The file \path{NEMOGCM/CONFIG/ORCA2_LIM/iodef_demo.xml} provides several examples of the use of
the new functionalities offered by the XML interface of XIOS.

\subsubsection{Define horizontal subdomains}

Horizontal subdomains are defined through the attributs zoom\_ibegin, zoom\_jbegin, zoom\_ni, zoom\_nj of
the tag family domain.
It must therefore be done in the domain part of the XML file. 
For example, in \path{CONFIG/SHARED/domain_def.xml}, we provide the following example of a definition of 
a 5 by 5 box with the bottom left corner at point (10,10).

\begin{xmllines}
<domain_group id="grid_T">
	<domain id="myzoom" zoom_ibegin="10" zoom_jbegin="10" zoom_ni="5" zoom_nj="5" />
\end{xmllines}

The use of this subdomain is done through the redefinition of the attribute domain\_ref of the tag family field.
For example:

\begin{xmllines}
<file id="myfile_vzoom" output_freq="1d" >
	<field field_ref="toce" domain_ref="myzoom"/>
</file>
\end{xmllines}

Moorings are seen as an extrem case corresponding to a 1 by 1 subdomain. 
The Equatorial section, the TAO, RAMA and PIRATA moorings are already registered in the code and
can therefore be outputted without taking care of their (i,j) position in the grid.
These predefined domains can be activated by the use of specific domain\_ref:
''EqT'', ''EqU'' or ''EqW'' for the equatorial sections and
the mooring position for TAO, RAMA and PIRATA followed by ''T'' (for example: ''8s137eT'', ''1.5s80.5eT'' ...)

\begin{xmllines}
<file id="myfile_vzoom" output_freq="1d" >
	<field field_ref="toce" domain_ref="0n180wT"/>
</file>
\end{xmllines}

Note that if the domain decomposition used in XIOS cuts the subdomain in several parts and if 
you use the ''multiple\_file'' type for your output files, 
you will endup with several files you will need to rebuild using unprovided tools (like ncpdq and ncrcat, 
\href{http://nco.sourceforge.net/nco.html#Concatenation}{see nco manual}).
We are therefore advising to use the ''one\_file'' type in this case.

\subsubsection{Define vertical zooms}

Vertical zooms are defined through the attributs zoom\_begin and zoom\_end of the tag family axis.
It must therefore be done in the axis part of the XML file.
For example, in \path{NEMOGCM/CONFIG/ORCA2_LIM/iodef_demo.xml}, we provide the following example:

\begin{xmllines}
<axis_group id="deptht" long_name="Vertical T levels" unit="m" positive="down" >
	<axis id="deptht" />
	<axis id="deptht_myzoom" zoom_begin="1" zoom_end="10" />
\end{xmllines}

The use of this vertical zoom is done through the redefinition of the attribute axis\_ref of the tag family field.
For example:

\begin{xmllines}
<file id="myfile_hzoom" output_freq="1d" >
	<field field_ref="toce" axis_ref="deptht_myzoom"/>
</file>
\end{xmllines}

\subsubsection{Control of the output file names}

The output file names are defined by the attributs ''name'' and ''name\_suffix'' of the tag family file.
For example:

\begin{xmllines}
<file_group id="1d" output_freq="1d" name="myfile_1d" > 
	<file id="myfileA" name_suffix="_AAA" > <!-- will create file "myfile_1d_AAA"  -->
		...
	</file>
	<file id="myfileB" name_suffix="_BBB" > <!-- will create file "myfile_1d_BBB" -->
		...
	</file>
</file_group>
\end{xmllines}

However it is often very convienent to define the file name with the name of the experiment,
the output file frequency and the date of the beginning and the end of the simulation
(which are informations stored either in the namelist or in the XML file).
To do so, we added the following rule:
if the id of the tag file is ''fileN'' (where N = 1 to 999 on 1 to 3 digits) or
one of the predefined sections or moorings (see next subsection),
the following part of the name and the name\_suffix (that can be inherited) will be automatically replaced by:

\begin{table}
  \scriptsize
  \begin{tabularx}{\textwidth}{|lX|}
    \hline
    \centering placeholder string &
    automatically replaced by                          \\
    \hline
    \hline
    \centering @expname@          &
    the experiment name (from cn\_exp in the namelist) \\
    \hline
    \centering @freq@             &
    output frequency (from attribute output\_freq)     \\
    \hline
    \centering @startdate@        &
    starting date of the simulation (from nn\_date0 in the restart or the namelist).
    \newline
    \verb?yyyymmdd?          format                   \\
    \hline
    \centering @startdatefull@    &
    starting date of the simulation (from nn\_date0 in the restart or the namelist).
    \newline
    \verb?yyyymmdd_hh:mm:ss? format                    \\
    \hline
    \centering @enddate@          &
    ending date of the simulation   (from nn\_date0 and nn\_itend  in the namelist).
    \newline
    \verb?yyyymmdd?          format                    \\
    \hline
    \centering @enddatefull@      &
    ending date of the simulation   (from nn\_date0 and nn\_itend  in the namelist).
    \newline
    \verb?yyyymmdd_hh:mm:ss? format                    \\
    \hline
  \end{tabularx}
\end{table}

\noindent For example, 
\xmlline|<file id="myfile_hzoom" name="myfile_@expname@_@startdate@_freq@freq@" output_freq="1d" >|

\noindent with the namelist:
\begin{forlines}
cn_exp    = "ORCA2"
nn_date0  = 19891231
ln_rstart = .false.
\end{forlines}

\noindent will give the following file name radical: \ifile{myfile\_ORCA2\_19891231\_freq1d}

\subsubsection{Other controls of the XML attributes from NEMO}

The values of some attributes are defined by subroutine calls within NEMO 
(calls to iom\_set\_domain\_attr, iom\_set\_axis\_attr and iom\_set\_field\_attr in \mdl{iom}).
Any definition given in the XML file will be overwritten.
By convention, these attributes are defined to ''auto'' (for string) or ''0000'' (for integer) in the XML file
(but this is not necessary).
\\

Here is the list of these attributes:
\\

\begin{table}
  \scriptsize
  \begin{tabularx}{\textwidth}{|X|c|c|c|}
    \hline
    tag ids affected by automatic definition of some of their attributes &
    name attribute                                                       &
    attribute value                      \\
    \hline
    \hline
    field\_definition                                                    &
    freq\_op                                                             &
    \np{rn\_rdt}                         \\
    \hline
    SBC                                                                  &
    freq\_op                                                             &
    \np{rn\_rdt} $\times$ \np{nn\_fsbc}  \\
    \hline
    ptrc\_T                                                              &
    freq\_op                                                             &
    \np{rn\_rdt} $\times$ \np{nn\_dttrc} \\
    \hline
    diad\_T                                                              &
    freq\_op                                                             &
    \np{rn\_rdt} $\times$ \np{nn\_dttrc} \\
    \hline
    EqT, EqU, EqW                                                        &
    jbegin, ni,                                                          &
    according to the grid                \\
    &
    name\_suffix                                                         &
    \\
    \hline
    TAO, RAMA and PIRATA moorings                                        &
    zoom\_ibegin, zoom\_jbegin,                                          &
    according to the grid                \\
    &
    name\_suffix                                                         &
    \\
    \hline
  \end{tabularx}
\end{table}

\subsubsection{Advanced use of XIOS functionalities}

\subsection{XML reference tables}
\label{subsec:IOM_xmlref}

\begin{enumerate}
\item
  Simple computation: directly define the computation when refering to the variable in the file definition.

\begin{xmllines}
<field field_ref="sst"  name="tosK"  unit="degK" > sst + 273.15 </field>
<field field_ref="taum" name="taum2" unit="N2/m4" long_name="square of wind stress module" >	taum * taum </field>
<field field_ref="qt"   name="stupid_check" > qt - qsr - qns </field>
\end{xmllines}

\item
  Simple computation: define a new variable and use it in the file definition.

in field\_definition:

\begin{xmllines}
<field id="sst2" long_name="square of sea surface temperature" unit="degC2" >  sst * sst </field >
\end{xmllines}

in file\_definition:

\begin{xmllines}
<field field_ref="sst2" > sst2 </field>
\end{xmllines}

Note that in this case, the following syntaxe \xmlcode{<field field_ref="sst2" />} is not working as
sst2 won't be evaluated.

\item
  Change of variable precision:

\begin{xmllines}
<!-- force to keep real 8 -->
<field field_ref="sst" name="tos_r8" prec="8" />
<!-- integer 2  with add_offset and scale_factor attributes -->
<field field_ref="sss" name="sos_i2" prec="2" add_offset="20." scale_factor="1.e-3" />
\end{xmllines}

Note that, then the code is crashing, writting real4 variables forces a numerical convection from 
real8 to real4 which will create an internal error in NetCDF and will avoid the creation of the output files.
Forcing double precision outputs with prec="8" (for example in the field\_definition) will avoid this problem.

\item
  add user defined attributes:

\begin{xmllines}
<file_group id="1d" output_freq="1d" output_level="10" enabled=".true."> <!-- 1d files --> 
	<file id="file1" name_suffix="_grid_T" description="ocean T grid variables" >
		<field field_ref="sst" name="tos" >
			<variable id="my_attribute1" type="string"  > blabla </variable>
			<variable id="my_attribute2" type="integer" > 3      </variable>
			<variable id="my_attribute3" type="float"   > 5.0    </variable>
		</field>
		<variable id="my_global_attribute" type="string" > blabla_global </variable>
	</file>
</file_group> 
\end{xmllines}

\item
  use of the ``@'' function: example 1, weighted temporal average

 - define a new variable in field\_definition

\begin{xmllines}
<field id="toce_e3t" long_name="temperature * e3t" unit="degC*m" grid_ref="grid_T_3D" >toce * e3t</field>
\end{xmllines}

 - use it when defining your file.

\begin{xmllines}
<file_group id="5d" output_freq="5d"  output_level="10" enabled=".true." >  <!-- 5d files -->  
	<file id="file1" name_suffix="_grid_T" description="ocean T grid variables" >
		<field field_ref="toce" operation="instant" freq_op="5d" > @toce_e3t / @e3t </field>
	</file>
</file_group> 
\end{xmllines}

The freq\_op="5d" attribute is used to define the operation frequency of the ``@'' function: here 5 day.
The temporal operation done by the ``@'' is the one defined in the field definition:
here we use the default, average.
So, in the above case, @toce\_e3t will do the 5-day mean of toce*e3t.
Operation="instant" refers to the temporal operation to be performed on the field''@toce\_e3t / @e3t'':
here the temporal average is alreday done by the ``@'' function so we just use instant to do the ratio of
the 2 mean values.
field\_ref="toce" means that attributes not explicitely defined, are inherited from toce field.
Note that in this case, freq\_op must be equal to the file output\_freq.

\item
  use of the ``@'' function: example 2, monthly SSH standard deviation

 - define a new variable in field\_definition

\begin{xmllines}
<field id="ssh2" long_name="square of sea surface temperature" unit="degC2" > ssh * ssh </field >
\end{xmllines}

 - use it when defining your file.

\begin{xmllines}
<file_group id="1m" output_freq="1m"  output_level="10" enabled=".true." >  <!-- 1m files -->  
	<file id="file1" name_suffix="_grid_T" description="ocean T grid variables" >
		<field field_ref="ssh" name="sshstd" long_name="sea_surface_temperature_standard_deviation" 
		operation="instant" freq_op="1m" >
			sqrt( @ssh2 - @ssh * @ssh )
		</field>
	</file>
</file_group> 
\end{xmllines}

The freq\_op="1m" attribute is used to define the operation frequency of the ``@'' function: here 1 month.
The temporal operation done by the ``@'' is the one defined in the field definition:
here we use the default, average.
So, in the above case, @ssh2 will do the monthly mean of ssh*ssh.
Operation="instant" refers to the temporal operation to be performed on the field ''sqrt( @ssh2 - @ssh * @ssh )'': 
here the temporal average is alreday done by the ``@'' function so we just use instant.
field\_ref="ssh" means that attributes not explicitely defined, are inherited from ssh field.
Note that in this case, freq\_op must be equal to the file output\_freq.

\item
  use of the ``@'' function: example 3, monthly average of SST diurnal cycle

 - define 2 new variables in field\_definition

\begin{xmllines}
<field id="sstmax" field_ref="sst" long_name="max of sea surface temperature" operation="maximum" />
<field id="sstmin" field_ref="sst" long_name="min of sea surface temperature" operation="minimum" />
\end{xmllines}

 - use these 2 new variables when defining your file.

\begin{xmllines}
<file_group id="1m" output_freq="1m"  output_level="10" enabled=".true." >  <!-- 1m files -->  
	<file id="file1" name_suffix="_grid_T" description="ocean T grid variables" >
		<field field_ref="sst" name="sstdcy" long_name="amplitude of sst diurnal cycle" operation="average" freq_op="1d" >
			@sstmax - @sstmin
		</field>
	</file>
</file_group> 
\end{xmllines}

\end{enumerate}

The freq\_op="1d" attribute is used to define the operation frequency of the ``@'' function: here 1 day.
The temporal operation done by the ``@'' is the one defined in the field definition:
here maximum for sstmax and minimum for sstmin.
So, in the above case, @sstmax will do the daily max and @sstmin the daily min.
Operation="average" refers to the temporal operation to be performed on the field ``@sstmax - @sstmin'':
here monthly mean (of daily max - daily min of the sst).
field\_ref="sst" means that attributes not explicitely defined, are inherited from sst field.

\subsubsection{Tag list per family}

\begin{table}
  \scriptsize
  \begin{tabularx}{\textwidth}{|l|X|X|l|X|}
    \hline
    tag name                                                                                     &
    description                                                                                  &
    accepted attribute                                                                           &
    child of                                                                                     &
    parent of                       \\
    \hline
    \hline
    simulation                                                                                   &
    this tag is the root tag which encapsulates all the content of the XML file                  &
    none                                                                                         &
    none                                                                                         &
    context                         \\
    \hline
    context                                                                                      &
    encapsulates parts of the XML file dedicated to different codes or different parts of a code &
    id (''xios'', ''nemo'' or ''n\_nemo'' for the nth AGRIF zoom), src, time\_origin             &
    simulation                                                                                   &
    all root tags: ... \_definition \\
    \hline
  \end{tabularx}
  \caption{Context tags}
\end{table}

\begin{table}
  \scriptsize
  \begin{tabularx}{\textwidth}{|l|X|X|X|l|}
    \hline
    tag name                                                                                &
    description                                                                             &
    accepted attribute                                                                      &
    child of                                                                                &
    parent of             \\
    \hline
    \hline
    field\_definition                                                                       &
    encapsulates the definition of all the fields that can potentially be outputted         &
    axis\_ref, default\_value, domain\_ref, enabled, grid\_ref, level, operation, prec, src &
    context                                                                                 &
    field or field\_group \\
    \hline
    field\_group                                                                            &
    encapsulates a group of fields                                                          &
    axis\_ref, default\_value, domain\_ref, enabled, group\_ref, grid\_ref,
    id, level, operation, prec, src                                                         &
    field\_definition, field\_group, file                                                   &
    field or field\_group \\
    \hline
    field                                                                                   &
    define a specific field                                                                 &
    axis\_ref, default\_value, domain\_ref, enabled, field\_ref, grid\_ref,
    id, level, long\_name, name, operation, prec, standard\_name, unit                      &
    field\_definition, field\_group, file                                                   &
    none                  \\
    \hline
  \end{tabularx}
  \caption{Field tags ("\tt{field\_*}")}
\end{table}

\begin{table}
  \scriptsize
  \begin{tabularx}{\textwidth}{|l|X|X|X|l|}
    \hline
    tag name                                                            &
    description                                                         &
    accepted attribute                                                  &
    child of                                                            &
    parent of           \\
    \hline
    \hline
    file\_definition                                                    &
    encapsulates the definition of all the files that will be outputted &
    enabled, min\_digits, name, name\_suffix, output\_level,
    split\_freq\_format, split\_freq, sync\_freq, type, src             &
    context                                                             &
    file or file\_group \\
    \hline
    file\_group                                                         &
    encapsulates a group of files that will be outputted                &
    enabled, description, id, min\_digits, name, name\_suffix, output\_freq, output\_level,
    split\_freq\_format, split\_freq, sync\_freq, type, src             &
    file\_definition, file\_group                                       &
    file or file\_group \\
    \hline
    file                                                                &
    define the contents of a file to be outputted                       &
    enabled, description, id, min\_digits, name, name\_suffix, output\_freq, output\_level,
    split\_freq\_format, split\_freq, sync\_freq, type, src             &
    file\_definition, file\_group                                       &
    field               \\
    \hline
  \end{tabularx}
  \caption{File tags ("\tt{file\_*}")}
\end{table}

\begin{table}
  \scriptsize
  \begin{tabularx}{\textwidth}{|l|X|X|X|X|}
    \hline
    tag name                                                                               &
    description                                                                            &
    accepted attribute                                                                     &
    child of                                                                               &
    parent of         \\
    \hline
    \hline
    axis\_definition                                                                       &
    define all the vertical axis potentially used by the variables                         &
    src                                                                                    &
    context                                                                                &
    axis\_group, axis \\
    \hline
    axis\_group                                                                            &
    encapsulates a group of vertical axis                                                  &
    id, lon\_name, positive, src, standard\_name, unit, zoom\_begin, zoom\_end, zoom\_size &
    axis\_definition, axis\_group                                                          &
    axis\_group, axis \\
    \hline
    axis                                                                                   &
    define a vertical axis                                                                 &
    id, lon\_name, positive, src, standard\_name, unit, zoom\_begin, zoom\_end, zoom\_size &
    axis\_definition, axis\_group                                                          &
    none					\\
    \hline
  \end{tabularx}
  \caption{Axis tags ("\tt{axis\_*}")}
\end{table}

\begin{table}
  \scriptsize
  \begin{tabularx}{\textwidth}{|l|X|X|X|X|}
    \hline
    tag name                                                            &
    description                                                         &
    accepted attribute                                                  &
    child of                                                            &
    parent of               \\
    \hline
    \hline
    domain\_\-definition                                                &
    define all the horizontal domains potentially used by the variables &
    src                                                                 &
    context                                                             &
    domain\_\-group, domain \\
    \hline
    domain\_group                                                       &
    encapsulates a group of horizontal domains                          &
    id, lon\_name, src, zoom\_ibegin, zoom\_jbegin, zoom\_ni, zoom\_nj  &
    domain\_\-definition, domain\_group                                 &
    domain\_\-group, domain \\
    \hline
    domain                                                              &
    define an horizontal domain                                         &
    id, lon\_name, src, zoom\_ibegin, zoom\_jbegin, zoom\_ni, zoom\_nj  &
    domain\_\-definition, domain\_group                                 &
    none                    \\
    \hline
  \end{tabularx}
  \caption{Domain tags ("\tt{domain\_*)}"}
\end{table}

\begin{table}
  \scriptsize
  \begin{tabularx}{\textwidth}{|l|X|X|X|X|}
    \hline
    tag name                                                                                     &
    description                                                                                  &
    accepted attribute                                                                           &
    child of                                                                                     &
    parent of                       \\
    \hline
    \hline
    grid\_definition                                                                               &
    define all the grid (association of a domain and/or an axis) potentially used by the variables &
    src                                                                                            &
    context                                                                                        &
    grid\_group, grid	\\
    \hline
    grid\_group                                                                                    &
    encapsulates a group of grids                                                                  &
    id, domain\_ref,axis\_ref                                                                      &
    grid\_definition, grid\_group                                                                  &
    grid\_group, grid	\\
    \hline
    grid                                                                                           &
    define a grid                                                                                  &
    id, domain\_ref,axis\_ref                                                                      &
    grid\_definition, grid\_group                                                                  &
    none 					\\
    \hline
  \end{tabularx}
  \caption{Grid tags ("\tt{grid\_*}")}
\end{table}

\subsubsection{Attributes list per family}

\begin{table}
  \scriptsize
  \begin{tabularx}{\textwidth}{|l|X|l|l|}
    \hline
    attribute name                           &
    description                              &
    example                                  &
    accepted by            \\
    \hline
    \hline
    axis\_ref                                &
    refers to the id of a vertical axis      &
    axis\_ref="deptht"                       &
    field, grid families   \\
    \hline
    domain\_ref                              &
    refers to the id of a domain             &
    domain\_ref="grid\_T"                    &
    field or grid families \\
    \hline
    field\_ref                               &
    id of the field we want to add in a file &
    field\_ref="toce"                        &
    field                  \\
    \hline
    grid\_ref                                &
    refers to the id of a grid               &
    grid\_ref="grid\_T\_2D"                  &
    field family           \\
    \hline
    group\_ref                               &
    refer to a group of variables            &
    group\_ref="mooring"                     &
    field\_group           \\
    \hline
  \end{tabularx}
  \caption{Reference attributes ("\tt{*\_ref}")}
\end{table}

\begin{table}
  \scriptsize
  \begin{tabularx}{\textwidth}{|l|X|l|l|}
    \hline
    attribute name                                     &
    description                                        &
    example                                            &
    accepted by   \\
    \hline
    \hline
    zoom\_ibegin                                       &
    starting point along x direction of the zoom.
    Automatically defined for TAO/RAMA/PIRATA moorings &
    zoom\_ibegin="1"                                   &
    domain family \\
    \hline
    zoom\_jbegin                                       &
    starting point along y direction of the zoom.
    Automatically defined for TAO/RAMA/PIRATA moorings &
    zoom\_jbegin="1"                                   &
    domain family \\
    \hline
    zoom\_ni                                           &
    zoom extent along x direction                      &
    zoom\_ni="1"                                       &
    domain family \\
    \hline
    zoom\_nj                                           &
    zoom extent along y direction                      &
    zoom\_nj="1"                                       &
    domain family \\
    \hline
  \end{tabularx}
  \caption{Domain attributes ("\tt{zoom\_*}")}
\end{table}

\begin{table}
  \scriptsize
  \begin{tabularx}{\textwidth}{|l|X|l|l|}
    \hline
    attribute name                                                                                       &
    description                                                                                          &
    example                                                                                              &
    accepted by                            \\
    \hline
    \hline
    min\_digits                                                                                          &
    specify the minimum of digits used in the core number in the name of the NetCDF file                 &
    min\_digits="4"                                                                                      &
    file family                            \\
    \hline
    name\_suffix                                                                                         &
    suffix to be inserted after the name and before the cpu number and the ''.nc'' termination of a file &
    name\_suffix="\_myzoom"                                                                              &
    file family                            \\
    \hline
    output\_level                                                                                        &
    output priority of variables in a file: 0 (high) to 10 (low).
    All variables listed in the file with a level smaller or equal to output\_level will be output.
    Other variables won't be output even if they are listed in the file.                                 &
    output\_level="10"                                                                                   &
    file family                            \\
    \hline
    split\_freq                                                                                          &
    frequency at which to temporally split output files.
    Units can be ts (timestep), y, mo, d, h, mi, s.
    Useful for long runs to prevent over-sized output files.                                             &
    split\_freq="1mo"                                                                                    &
    file family                            \\
    \hline
    split\_freq\-\_format                                                                                &
    date format used in the name of temporally split output files.
    Can be specified using the following syntaxes: \%y, \%mo, \%d, \%h \%mi and \%s                      &
    split\_freq\_format= "\%y\%mo\%d"                                                                    &
    file family                            \\
    \hline
    sync\_freq                                                                                           &
    NetCDF file synchronization frequency (update of the time\_counter).
    Units can be ts (timestep), y, mo, d, h, mi, s.                                                      &
    sync\_freq="10d"                                                                                     &
    file family                            \\
    \hline
    type (1)                                                                                             &
    specify if the output files are to be split spatially (multiple\_file) or not (one\_file)            &
    type="multiple\_file"                                                                                &
    file familly                           \\
    \hline
  \end{tabularx}
  \caption{File attributes}
\end{table}

\begin{table}
  \scriptsize
  \begin{tabularx}{\textwidth}{|l|X|l|l|}
    \hline
    attribute name                                                                                       &
    description                                                                                          &
    example                                                                                              &
    accepted by                            \\
    \hline
    \hline
    default\_value                                                                                       &
    missing\_value definition                                                                            &
    default\_value="1.e20"                                                                               &
    field family                           \\
    \hline
    level                                                                                                &
    output priority of a field: 0 (high) to 10 (low)                                                     &
    level="1"                                                                                            &
    field family                           \\
    \hline
    operation                                                                                            &
    type of temporal operation: average, accumulate, instantaneous, min, max and once                    &
    operation="average"                                                                                  &
    field family                           \\
    \hline
    output\_freq                                                                                         &
    operation frequency. units can be ts (timestep), y, mo, d, h, mi, s.                                 &
    output\_freq="1d12h"                                                                                 &
    field family                           \\
    \hline
    prec                                                                                                 &
    output precision: real 4 or real 8                                                                   &
    prec="4"                                                                                             &
    field family                           \\
    \hline
    long\_name                                                                                           &
    define the long\_name attribute in the NetCDF file                                                   &
    long\_name="Vertical T levels"                                                                       &
    field                                  \\
    \hline
    standard\_name                                                                                       &
    define the standard\_name attribute in the NetCDF file                                               &
    standard\_name= "Eastward\_Sea\_Ice\_Transport"                                                      &
    field                                  \\
    \hline
  \end{tabularx}
  \caption{Field attributes}
\end{table}

\begin{table}
  \scriptsize
  \begin{tabularx}{\textwidth}{|l|X|X|X|}
    \hline
    attribute name                                                                                       &
    description                                                                                          &
    example                                                                                              &
    accepted by                            \\
    \hline
    \hline
    enabled                                                                                              &
    switch on/off the output of a field or a file                                                        &
    enabled=".true."                                                                                     &
    field, file families                   \\
    \hline
    description                                                                                          &
    just for information, not used                                                                       &
    description="ocean T grid variables"                                                                 &
    all tags                               \\
    \hline
    id                                                                                                   &
    allow to identify a tag                                                                              &
    id="nemo"                                                                                            &
    accepted by all tags except simulation \\
    \hline
    name                                                                                                 &
    name of a variable or a file. If the name of a file is undefined, its id is used as a name           &
    name="tos"                                                                                           &
    field or file families                 \\
    \hline
    positive                                                                                             &
    convention used for the orientation of vertival axis (positive downward in \NEMO).                   &
    positive="down"                                                                                      &
    axis family                            \\
    \hline
    src                                                                                                  &
    allow to include a file                                                                              &
    src="./field\_def.xml"                                                                               &
    accepted by all tags except simulation \\
    \hline
    time\_origin                                                                                         &
    specify the origin of the time counter                                                               &
    time\_origin="1900-01-01 00:00:00"                                                                   &
    context                                \\
    \hline
    type (2)                                                                                             &
    define the type of a variable tag                                                                    &
    type="boolean"                                                                                       &
    variable                               \\
    \hline
    unit                                                                                                 &
    unit of a variable or the vertical axis                                                              &
    unit="m"                                                                                             &
    field and axis families                \\
    \hline
  \end{tabularx}
  \caption{Miscellaneous attributes}
\end{table}

\subsection{CF metadata standard compliance}

Output from the XIOS-1.0 IO server is compliant with 
\href{http://cfconventions.org/Data/cf-conventions/cf-conventions-1.5/build/cf-conventions.html}{version 1.5} of
the CF metadata standard. 
Therefore while a user may wish to add their own metadata to the output files (as demonstrated in example 4 of
section \autoref{subsec:IOM_xmlref}) the metadata should, for the most part, comply with the CF-1.5 standard.

Some metadata that may significantly increase the file size (horizontal cell areas and vertices) are controlled by
the namelist parameter \np{ln\_cfmeta} in the \ngn{namrun} namelist.
This must be set to true if these metadata are to be included in the output files.


% ================================================================
%       NetCDF4 support
% ================================================================
\section{NetCDF4 support (\protect\key{netcdf4})}
\label{sec:DIA_nc4}

Since version 3.3, support for NetCDF4 chunking and (loss-less) compression has been included.
These options build on the standard NetCDF output and allow the user control over the size of the chunks via
namelist settings.
Chunking and compression can lead to significant reductions in file sizes for a small runtime overhead.
For a fuller discussion on chunking and other performance issues the reader is referred to
the NetCDF4 documentation found \href{http://www.unidata.ucar.edu/software/netcdf/docs/netcdf.html#Chunking}{here}.

The new features are only available when the code has been linked with a NetCDF4 library
(version 4.1 onwards, recommended) which has been built with HDF5 support (version 1.8.4 onwards, recommended).
Datasets created with chunking and compression are not backwards compatible with NetCDF3 "classic" format but
most analysis codes can be relinked simply with the new libraries and will then read both NetCDF3 and NetCDF4 files.
NEMO executables linked with NetCDF4 libraries can be made to produce NetCDF3 files by
setting the \np{ln\_nc4zip} logical to false in the \textit{namnc4} namelist:

%------------------------------------------namnc4----------------------------------------------------

\nlst{namnc4}
%-------------------------------------------------------------------------------------------------------------

If \key{netcdf4} has not been defined, these namelist parameters are not read.
In this case, \np{ln\_nc4zip} is set false and dummy routines for a few NetCDF4-specific functions are defined.
These functions will not be used but need to be included so that compilation is possible with NetCDF3 libraries.

When using NetCDF4 libraries, \key{netcdf4} should be defined even if the intention is to
create only NetCDF3-compatible files.
This is necessary to avoid duplication between the dummy routines and the actual routines present in the library.
Most compilers will fail at compile time when faced with such duplication.
Thus when linking with NetCDF4 libraries the user must define \key{netcdf4} and
control the type of NetCDF file produced via the namelist parameter.

Chunking and compression is applied only to 4D fields and
there is no advantage in chunking across more than one time dimension since
previously written chunks would have to be read back and decompressed before being added to.
Therefore, user control over chunk sizes is provided only for the three space dimensions.
The user sets an approximate number of chunks along each spatial axis.
The actual size of the chunks will depend on global domain size for mono-processors or, more likely,
the local processor domain size for distributed processing.
The derived values are subject to practical minimum values (to avoid wastefully small chunk sizes) and
cannot be greater than the domain size in any dimension.
The algorithm used is:

\begin{forlines}
ichunksz(1) = MIN(idomain_size, MAX((idomain_size-1) / nn_nchunks_i + 1 ,16 ))
ichunksz(2) = MIN(jdomain_size, MAX((jdomain_size-1) / nn_nchunks_j + 1 ,16 ))
ichunksz(3) = MIN(kdomain_size, MAX((kdomain_size-1) / nn_nchunks_k + 1 , 1 ))
ichunksz(4) = 1
\end{forlines}

\noindent As an example, setting:

\begin{forlines}
nn_nchunks_i=4, nn_nchunks_j=4 and nn_nchunks_k=31
\end{forlines}

\noindent for a standard ORCA2\_LIM configuration gives chunksizes of {\small\tt 46x38x1} respectively in
the mono-processor case (\ie global domain of {\small\tt 182x149x31}).
An illustration of the potential space savings that NetCDF4 chunking and compression provides is given in 
table \autoref{tab:NC4} which compares the results of two short runs of the ORCA2\_LIM reference configuration with
a 4x2 mpi partitioning.
Note the variation in the compression ratio achieved which reflects chiefly the dry to wet volume ratio of
each processing region.

%------------------------------------------TABLE----------------------------------------------------
\begin{table}
  \scriptsize
  \centering
  \begin{tabular}{lrrr}
    Filename                    & NetCDF3	& NetCDF4  & Reduction \\
                                & filesize	& filesize & \%        \\
                                & (KB)		& (KB)	  &           \\
    ORCA2\_restart\_0000.nc     & 16420 	& 8860 	  & 47\%      \\
    ORCA2\_restart\_0001.nc     & 16064 	& 11456    & 29\%      \\
    ORCA2\_restart\_0002.nc     & 16064		& 9744	  & 40\%      \\
    ORCA2\_restart\_0003.nc     & 16420		& 9404	  & 43\%      \\
    ORCA2\_restart\_0004.nc     & 16200 	& 5844	  & 64\%      \\
    ORCA2\_restart\_0005.nc     & 15848 	& 8172	  & 49\%      \\
    ORCA2\_restart\_0006.nc     & 15848 	& 8012 	  & 50\%      \\
    ORCA2\_restart\_0007.nc     & 16200 	& 5148 	  & 69\%      \\
    ORCA2\_2d\_grid\_T\_0000.nc & 2200 		& 1504	  & 32\%      \\
    ORCA2\_2d\_grid\_T\_0001.nc & 2200 		& 1748	  & 21\%      \\
    ORCA2\_2d\_grid\_T\_0002.nc & 2200 		& 1592	  & 28\%      \\
    ORCA2\_2d\_grid\_T\_0003.nc & 2200 		& 1540	  & 30\%      \\
    ORCA2\_2d\_grid\_T\_0004.nc & 2200 		& 1204	  & 46\%      \\
    ORCA2\_2d\_grid\_T\_0005.nc & 2200 		& 1444	  & 35\%      \\
    ORCA2\_2d\_grid\_T\_0006.nc & 2200 		& 1428	  & 36\%      \\
    ORCA2\_2d\_grid\_T\_0007.nc & 2200		& 1148	  & 48\%      \\
    ...                         & ...		& ...      & ...       \\
    ORCA2\_2d\_grid\_W\_0000.nc & 4416		& 2240	  & 50\%      \\
    ORCA2\_2d\_grid\_W\_0001.nc & 4416		& 2924	  & 34\%      \\
    ORCA2\_2d\_grid\_W\_0002.nc & 4416		& 2512	  & 44\%      \\
    ORCA2\_2d\_grid\_W\_0003.nc & 4416		& 2368	  & 47\%      \\
    ORCA2\_2d\_grid\_W\_0004.nc & 4416		& 1432	  & 68\%      \\
    ORCA2\_2d\_grid\_W\_0005.nc & 4416		& 1972	  & 56\%      \\
    ORCA2\_2d\_grid\_W\_0006.nc & 4416		& 2028	  & 55\%      \\
    ORCA2\_2d\_grid\_W\_0007.nc & 4416		& 1368	  & 70\%      \\
  \end{tabular}
  \caption{
    \protect\label{tab:NC4}
    Filesize comparison between NetCDF3 and NetCDF4 with chunking and compression
  }
\end{table}
%----------------------------------------------------------------------------------------------------

When \key{iomput} is activated with \key{netcdf4} chunking and compression parameters for fields produced via
\np{iom\_put} calls are set via an equivalent and identically named namelist to \textit{namnc4} in
\np{xmlio\_server.def}.
Typically this namelist serves the mean files whilst the \ngn{ namnc4} in the main namelist file continues to
serve the restart files.
This duplication is unfortunate but appropriate since, if using io\_servers, the domain sizes of
the individual files produced by the io\_server processes may be different to those produced by
the invidual processing regions and different chunking choices may be desired.
 
% -------------------------------------------------------------------------------------------------------------
%       Tracer/Dynamics Trends
% -------------------------------------------------------------------------------------------------------------
\section{Tracer/Dynamics trends  (\protect\ngn{namtrd})}
\label{sec:DIA_trd}

%------------------------------------------namtrd----------------------------------------------------

\nlst{namtrd} 
%-------------------------------------------------------------------------------------------------------------

Each trend of the dynamics and/or temperature and salinity time evolution equations can be send to
\mdl{trddyn} and/or \mdl{trdtra} modules (see TRD directory) just after their computation
(\ie at the end of each $dyn\cdots.F90$ and/or $tra\cdots.F90$ routines).
This capability is controlled by options offered in \ngn{namtrd} namelist.
Note that the output are done with xIOS, and therefore the \key{IOM} is required.

What is done depends on the \ngn{namtrd} logical set to \forcode{.true.}:

\begin{description}
\item[\np{ln\_glo\_trd}]:
  at each \np{nn\_trd} time-step a check of the basin averaged properties of
  the momentum and tracer equations is performed.
  This also includes a check of $T^2$, $S^2$, $\tfrac{1}{2} (u^2+v2)$,
  and potential energy time evolution equations properties;
\item[\np{ln\_dyn\_trd}]:
  each 3D trend of the evolution of the two momentum components is output;
\item[\np{ln\_dyn\_mxl}]:
  each 3D trend of the evolution of the two momentum components averaged over the mixed layer is output;
\item[\np{ln\_vor\_trd}]:
  a vertical summation of the moment tendencies is performed,
  then the curl is computed to obtain the barotropic vorticity tendencies which are output;
\item[\np{ln\_KE\_trd}] :
  each 3D trend of the Kinetic Energy equation is output;
\item[\np{ln\_tra\_trd}]:
  each 3D trend of the evolution of temperature and salinity is output;
\item[\np{ln\_tra\_mxl}]:
  each 2D trend of the evolution of temperature and salinity averaged over the mixed layer is output;
\end{description}

Note that the mixed layer tendency diagnostic can also be used on biogeochemical models via 
the \key{trdtrc} and \key{trdmld\_trc} CPP keys.

\textbf{Note that} in the current version (v3.6), many changes has been introduced but not fully tested.
In particular, options associated with \np{ln\_dyn\_mxl}, \np{ln\_vor\_trd}, and \np{ln\_tra\_mxl} are not working,
and none of the options have been tested with variable volume (\ie \key{vvl} defined).

% -------------------------------------------------------------------------------------------------------------
%       On-line Floats trajectories
% -------------------------------------------------------------------------------------------------------------
\section{FLO: On-Line Floats trajectories (\protect\key{floats})}
\label{sec:FLO}
%--------------------------------------------namflo-------------------------------------------------------

\nlst{namflo} 
%--------------------------------------------------------------------------------------------------------------

The on-line computation of floats advected either by the three dimensional velocity field or constraint to
remain at a given depth ($w = 0$ in the computation) have been introduced in the system during the CLIPPER project.
Options are defined by \ngn{namflo} namelis variables.
The algorithm used is based either on the work of \cite{Blanke_Raynaud_JPO97} (default option),
or on a $4^th$ Runge-Hutta algorithm (\np{ln\_flork4}\forcode{ = .true.}).
Note that the \cite{Blanke_Raynaud_JPO97} algorithm have the advantage of providing trajectories which
are consistent with the numeric of the code, so that the trajectories never intercept the bathymetry.

\subsubsection{Input data: initial coordinates}

Initial coordinates can be given with Ariane Tools convention
(IJK coordinates, (\np{ln\_ariane}\forcode{ = .true.}) ) or with longitude and latitude.

In case of Ariane convention, input filename is \np{init\_float\_ariane}.
Its format is: \\
{\scriptsize \texttt{I J K nisobfl itrash itrash}}

\noindent with:

 - I,J,K  : indexes of initial position

 - nisobfl: 0 for an isobar float, 1 for a float following the w velocity  

 - itrash : set to zero; it is a dummy variable to respect Ariane Tools convention

\noindent Example: \\
\noindent
{\scriptsize
  \texttt{
    100.00000  90.00000  -1.50000 1.00000   0.00000   \\
    102.00000  90.00000  -1.50000 1.00000   0.00000   \\
    104.00000  90.00000  -1.50000 1.00000   0.00000   \\
    106.00000  90.00000  -1.50000 1.00000   0.00000   \\
    108.00000  90.00000  -1.50000 1.00000   0.00000}
} \\

In the other case (longitude and latitude), input filename is init\_float.
Its format is: \\
{\scriptsize \texttt{Long Lat depth nisobfl ngrpfl itrash}}

\noindent with:

 - Long, Lat, depth  : Longitude, latitude, depth

 - nisobfl: 0 for an isobar float, 1 for a float following the w velocity

 - ngrpfl : number to identify searcher group

 - itrash :set to 1; it is a dummy variable.

\noindent Example: \\
\noindent
{\scriptsize
  \texttt{
    20.0 0.0 0.0 0 1 1    \\
    -21.0 0.0 0.0 0 1 1    \\
    -22.0 0.0 0.0 0 1 1    \\
    -23.0 0.0 0.0 0 1 1    \\
    -24.0 0.0 0.0 0 1 1 }
} \\

\np{jpnfl} is the total number of floats during the run.
When initial positions are read in a restart file (\np{ln\_rstflo}\forcode{ = .true.} ),
\np{jpnflnewflo} can be added in the initialization file.

\subsubsection{Output data}

\np{nn\_writefl} is the frequency of writing in float output file and \np{nn\_stockfl} is the frequency of
creation of the float restart file.

Output data can be written in ascii files (\np{ln\_flo\_ascii}\forcode{ = .true.}).
In that case, output filename is trajec\_float.

Another possiblity of writing format is Netcdf (\np{ln\_flo\_ascii}\forcode{ = .false.}).
There are 2 possibilities:

- if (\key{iomput}) is used, outputs are selected in  iodef.xml.
Here it is an example of specification to put in files description section:

\begin{xmllines}
<group id="1d_grid_T" name="auto" description="ocean T grid variables" >   }
	<file id="floats"  description="floats variables"> }
		<field ref="traj_lon"   name="floats_longitude"   freq_op="86400" />}
		<field ref="traj_lat"   name="floats_latitude"    freq_op="86400" />}
		<field ref="traj_dep"   name="floats_depth"       freq_op="86400" />}
		<field ref="traj_temp"  name="floats_temperature" freq_op="86400" />}
		<field ref="traj_salt"  name="floats_salinity"    freq_op="86400" />}
		<field ref="traj_dens"  name="floats_density"     freq_op="86400" />}
		<field ref="traj_group" name="floats_group"       freq_op="86400" />}
	</file>}
</group>}
\end{xmllines}

 -  if (\key{iomput}) is not used, a file called \ifile{trajec\_float} will be created by IOIPSL library.

 See also \href{http://stockage.univ-brest.fr/~grima/Ariane/}{here} the web site describing the off-line use of
 this marvellous diagnostic tool.

% -------------------------------------------------------------------------------------------------------------
%       Harmonic analysis of tidal constituents
% -------------------------------------------------------------------------------------------------------------
\section{Harmonic analysis of tidal constituents (\protect\key{diaharm}) }
\label{sec:DIA_diag_harm}

%------------------------------------------namdia_harm----------------------------------------------------
%
\nlst{nam_diaharm}
%----------------------------------------------------------------------------------------------------------

A module is available to compute the amplitude and phase of tidal waves.
This on-line Harmonic analysis is actived with \key{diaharm}.

Some parameters are available in namelist \ngn{namdia\_harm}:

 - \np{nit000\_han} is the first time step used for harmonic analysis

 - \np{nitend\_han} is the  last time step used for harmonic analysis

 - \np{nstep\_han}  is the  time step frequency for harmonic analysis

 - \np{nb\_ana}     is the number of harmonics to analyse

 - \np{tname}       is an array with names of tidal constituents to analyse

 \np{nit000\_han} and \np{nitend\_han} must be between \np{nit000} and \np{nitend} of the simulation.
 The restart capability is not implemented.

 The Harmonic analysis solve the following equation:

 \[
   h_{i} - A_{0} + \sum^{nb\_ana}_{j=1}[A_{j}cos(\nu_{j}t_{j}-\phi_{j})] = e_{i}
 \]

With $A_{j}$, $\nu_{j}$, $\phi_{j}$, the amplitude, frequency and phase for each wave and $e_{i}$ the error.
$h_{i}$ is the sea level for the time $t_{i}$ and $A_{0}$ is the mean sea level. \\
We can rewrite this equation:

\[
  h_{i} - A_{0} + \sum^{nb\_ana}_{j=1}[C_{j}cos(\nu_{j}t_{j})+S_{j}sin(\nu_{j}t_{j})] = e_{i}
\]

with $A_{j}=\sqrt{C^{2}_{j}+S^{2}_{j}}$ and $\phi_{j}=arctan(S_{j}/C_{j})$.

We obtain in output $C_{j}$ and $S_{j}$ for each tidal wave.

% -------------------------------------------------------------------------------------------------------------
%       Sections transports
% -------------------------------------------------------------------------------------------------------------
\section{Transports across sections (\protect\key{diadct}) }
\label{sec:DIA_diag_dct}

%------------------------------------------namdct----------------------------------------------------

\nlst{namdct}
%-------------------------------------------------------------------------------------------------------------

A module is available to compute the transport of volume, heat and salt through sections.
This diagnostic is actived with \key{diadct}.

Each section is defined by the coordinates of its 2 extremities.
The pathways between them are contructed using tools which can be found in \texttt{NEMOGCM/TOOLS/SECTIONS\_DIADCT}
and are written in a binary file \texttt{section\_ijglobal.diadct\_ORCA2\_LIM} which is later read in by
NEMO to compute on-line transports.

The on-line transports module creates three output ascii files:

- \texttt{volume\_transport} for volume transports (unit: $10^{6} m^{3} s^{-1}$)

- \texttt{heat\_transport}   for   heat transports (unit: $10^{15} W$)

- \texttt{salt\_transport}   for   salt transports (unit: $10^{9}Kg s^{-1}$) \\

Namelist variables in \ngn{namdct} control how frequently the flows are summed and the time scales over which
they are averaged, as well as the level of output for debugging:
\np{nn\_dct}   : frequency of instantaneous transports computing
\np{nn\_dctwri}: frequency of writing ( mean of instantaneous transports )
\np{nn\_debug} : debugging of the section

\subsubsection{Creating a binary file containing the pathway of each section}

In \texttt{NEMOGCM/TOOLS/SECTIONS\_DIADCT/run},
the file \textit{ {list\_sections.ascii\_global}} contains a list of all the sections that are to be computed
(this list of sections is based on MERSEA project metrics).

Another file is available for the GYRE configuration (\texttt{ {list\_sections.ascii\_GYRE}}).

Each section is defined by: \\
\noindent {\scriptsize \texttt{long1 lat1 long2 lat2 nclass (ok/no)strpond (no)ice section\_name}} \\
with:

 - \texttt{long1 lat1}, coordinates of the  first extremity of the section;

 - \texttt{long2 lat2}, coordinates of the second extremity of the section;

 - \texttt{nclass}    the number of bounds of your classes (\eg bounds for 2 classes);

 - \texttt{okstrpond} to compute    heat and       salt transports, \texttt{nostrpond} if no;

 - \texttt{ice}       to compute surface and volume ice transports, \texttt{noice}     if no. \\

 \noindent The results of the computing of transports, and the directions of positive and
 negative flow do not depend on the order of the 2 extremities in this file. \\

\noindent If nclass $\neq$ 0, the next lines contain the class type and the nclass bounds: \\
{\scriptsize
  \texttt{
    long1 lat1 long2 lat2 nclass (ok/no)strpond (no)ice section\_name \\
    classtype                                                         \\
    zbound1                                                           \\
    zbound2                                                           \\
    .                                                                 \\
    .                                                                 \\
    nclass-1                                                          \\
    nclass}
}

\noindent where \texttt{classtype} can be:

 - \texttt{zsal}  for          salinity classes

 - \texttt{ztem}  for       temperature classes

 - \texttt{zlay}  for             depth classes

 - \texttt{zsigi} for    insitu density classes

 - \texttt{zsigp} for potential density classes \\
  
 The script \texttt{job.ksh} computes the pathway for each section and creates a binary file
 \texttt{section\_ijglobal.diadct\_ORCA2\_LIM} which is read by NEMO. \\

 It is possible to use this tools for new configuations: \texttt{job.ksh} has to be updated with
 the coordinates file name and path. \\

 Examples of two sections, the ACC\_Drake\_Passage with no classes,
 and the ATL\_Cuba\_Florida with 4 temperature clases (5 class bounds), are shown: \\
 \noindent
 {\scriptsize
   \texttt{
     -68.    -54.5   -60.    -64.7  00 okstrpond noice ACC\_Drake\_Passage \\
     -80.5    22.5   -80.5    25.5  05 nostrpond noice ATL\_Cuba\_Florida  \\
     ztem                                                                  \\
     -2.0                                                                  \\
     4.5                                                                  \\
     7.0                                                                  \\
     12.0                                                                  \\
     40.0}
 }

\subsubsection{To read the output files}

The output format is: \\
{\scriptsize
  \texttt{
    date, time-step number, section number,                \\
    section name, section slope coefficient, class number, \\
    class name, class bound 1 , classe bound2,             \\
    transport\_direction1, transport\_direction2,          \\
    transport\_total}
}                                     \\

For sections with classes, the first \texttt{nclass-1} lines correspond to the transport for each class and
the last line corresponds to the total transport summed over all classes.
For sections with no classes, class number \texttt{1} corresponds to \texttt{total class} and
this class is called \texttt{N}, meaning \texttt{none}.

- \texttt{transport\_direction1} is the positive part of the transport ($\geq$ 0).

- \texttt{transport\_direction2} is the negative part of the transport ($\leq$ 0). \\

\noindent The \texttt{section slope coefficient} gives information about the significance of transports signs and
direction: \\

\begin{table}
  \scriptsize
  \begin{tabular}{|l|l|l|l|l|}
    \hline
    section slope coefficient      & section type & direction 1 & direction 2 & total transport    \\
    \hline
    0.                             & horizontal	 & northward	& southward   & postive: northward    \\
    \hline
    1000.                          & vertical     & eastward    & westward    & postive: eastward		\\
    \hline
    \texttt{$\neq$ 0, $\neq$ 1000.} & diagonal     & eastward    & westward	  & postive: eastward		\\
    \hline
  \end{tabular}
\end{table}

% ================================================================
% Steric effect in sea surface height
% ================================================================
\section{Diagnosing the steric effect in sea surface height}
\label{sec:DIA_steric}


Changes in steric sea level are caused when changes in the density of the water column imply an expansion or
contraction of the column.
It is essentially produced through surface heating/cooling and to a lesser extent through non-linear effects of
the equation of state (cabbeling, thermobaricity...).
Non-Boussinesq models contain all ocean effects within the ocean acting on the sea level.
In particular, they include the steric effect.
In contrast, Boussinesq models, such as \NEMO, conserve volume, rather than mass,
and so do not properly represent expansion or contraction.
The steric effect is therefore not explicitely represented.
This approximation does not represent a serious error with respect to the flow field calculated by the model
\citep{Greatbatch_JGR94}, but extra attention is required when investigating sea level,
as steric changes are an important contribution to local changes in sea level on seasonal and climatic time scales.
This is especially true for investigation into sea level rise due to global warming.

Fortunately, the steric contribution to the sea level consists of a spatially uniform component that
can be diagnosed by considering the mass budget of the world ocean \citep{Greatbatch_JGR94}.
In order to better understand how global mean sea level evolves and thus how the steric sea level can be diagnosed,
we compare, in the following, the non-Boussinesq and Boussinesq cases.

Let denote
$\mathcal{M}$ the total mass    of liquid seawater ($\mathcal{M} = \int_D \rho dv$), 
$\mathcal{V}$ the total volume  of        seawater      ($\mathcal{V} = \int_D dv$), 
$\mathcal{A}$ the total surface of       the ocean      ($\mathcal{A} = \int_S ds$), 
$\bar{\rho}$ the global mean  seawater (\textit{in situ}) density 
($\bar{\rho} = 1/\mathcal{V} \int_D \rho \,dv$), and
$\bar{\eta}$ the global mean sea level 
($\bar{\eta} = 1/\mathcal{A} \int_S \eta \,ds$).

A non-Boussinesq fluid conserves mass. It satisfies the following relations:

\begin{equation}
  \begin{split}
    \mathcal{M} &=  \mathcal{V}  \;\bar{\rho} \\
    \mathcal{V} &=  \mathcal{A}  \;\bar{\eta}
  \end{split}
  \label{eq:MV_nBq}
\end{equation}

Temporal changes in total mass is obtained from the density conservation equation:

\begin{equation}
  \frac{1}{e_3} \partial_t ( e_3\,\rho) + \nabla( \rho \, \textbf{U} )
  = \left. \frac{\textit{emp}}{e_3}\right|_\textit{surface}
  \label{eq:Co_nBq}
\end{equation}

where $\rho$ is the \textit{in situ} density, and \textit{emp} the surface mass exchanges with the other media of
the Earth system (atmosphere, sea-ice, land).
Its global averaged leads to the total mass change 

\begin{equation}
  \partial_t \mathcal{M} = \mathcal{A} \;\overline{\textit{emp}}
  \label{eq:Mass_nBq}
\end{equation}

where $\overline{\textit{emp}} = \int_S \textit{emp}\,ds$ is the net mass flux through the ocean surface.
Bringing \autoref{eq:Mass_nBq} and the time derivative of \autoref{eq:MV_nBq} together leads to
the evolution equation of the mean sea level

\begin{equation}
  \partial_t \bar{\eta} = \frac{\overline{\textit{emp}}}{ \bar{\rho}}
  - \frac{\mathcal{V}}{\mathcal{A}} \;\frac{\partial_t \bar{\rho} }{\bar{\rho}}
  \label{eq:ssh_nBq}
\end{equation}

The first term in equation \autoref{eq:ssh_nBq} alters sea level by adding or subtracting mass from the ocean. 
The second term arises from temporal changes in the global mean density; \ie from steric effects.

In a Boussinesq fluid, $\rho$ is replaced by $\rho_o$ in all the equation except when $\rho$ appears multiplied by
the gravity (\ie in the hydrostatic balance of the primitive Equations).
In particular, the mass conservation equation, \autoref{eq:Co_nBq}, degenerates into the incompressibility equation:

\[
  \frac{1}{e_3} \partial_t ( e_3 ) + \nabla( \textbf{U} ) = \left. \frac{\textit{emp}}{\rho_o \,e_3}\right|_ \textit{surface}
  % \label{eq:Co_Bq}
\]

and the global average of this equation now gives the temporal change of the total volume,

\[
  \partial_t \mathcal{V} = \mathcal{A} \;\frac{\overline{\textit{emp}}}{\rho_o}
  % \label{eq:V_Bq}
\]

Only the volume is conserved, not mass, or, more precisely, the mass which is conserved is the Boussinesq mass,
$\mathcal{M}_o = \rho_o \mathcal{V}$.
The total volume (or equivalently the global mean sea level) is altered only by net volume fluxes across
the ocean surface, not by changes in mean mass of the ocean: the steric effect is missing in a Boussinesq fluid.

Nevertheless, following \citep{Greatbatch_JGR94}, the steric effect on the volume can be diagnosed by
considering the mass budget of the ocean. 
The apparent changes in $\mathcal{M}$, mass of the ocean, which are not induced by surface mass flux
must be compensated by a spatially uniform change in the mean sea level due to expansion/contraction of the ocean
\citep{Greatbatch_JGR94}.
In others words, the Boussinesq mass, $\mathcal{M}_o$, can be related to $\mathcal{M}$,
the total mass of the ocean seen by the Boussinesq model, via the steric contribution to the sea level,
$\eta_s$, a spatially uniform variable, as follows:

\begin{equation}
  \mathcal{M}_o = \mathcal{M} + \rho_o \,\eta_s \,\mathcal{A}
  \label{eq:M_Bq}
\end{equation}

Any change in $\mathcal{M}$ which cannot be explained by the net mass flux through the ocean surface
is converted into a mean change in sea level.
Introducing the total density anomaly, $\mathcal{D}= \int_D d_a \,dv$,
where $d_a = (\rho -\rho_o ) / \rho_o$ is the density anomaly used in \NEMO (cf. \autoref{subsec:TRA_eos})
in \autoref{eq:M_Bq} leads to a very simple form for the steric height:

\begin{equation}
  \eta_s = - \frac{1}{\mathcal{A}} \mathcal{D}
  \label{eq:steric_Bq}
\end{equation}

The above formulation of the steric height of a Boussinesq ocean requires four remarks.
First, one can be tempted to define $\rho_o$ as the initial value of $\mathcal{M}/\mathcal{V}$,
\ie set $\mathcal{D}_{t=0}=0$, so that the initial steric height is zero.
We do not recommend that.
Indeed, in this case $\rho_o$ depends on the initial state of the ocean.
Since $\rho_o$ has a direct effect on the dynamics of the ocean
(it appears in the pressure gradient term of the momentum equation)
it is definitively not a good idea when inter-comparing experiments.
We better recommend to fixe once for all $\rho_o$ to $1035\;Kg\,m^{-3}$.
This value is a sensible choice for the reference density used in a Boussinesq ocean climate model since,
with the exception of only a small percentage of the ocean, density in the World Ocean varies by no more than
2$\%$ from this value (\cite{Gill1982}, page 47).

Second, we have assumed here that the total ocean surface, $\mathcal{A}$,
does not change when the sea level is changing as it is the case in all global ocean GCMs
(wetting and drying of grid point is not allowed).
  
Third, the discretisation of \autoref{eq:steric_Bq} depends on the type of free surface which is considered.
In the non linear free surface case, \ie \key{vvl} defined, it is given by

\[
  \eta_s = - \frac{ \sum_{i,\,j,\,k} d_a\; e_{1t} e_{2t} e_{3t} }{ \sum_{i,\,j,\,k}       e_{1t} e_{2t} e_{3t} }
  % \label{eq:discrete_steric_Bq_nfs}
\]

whereas in the linear free surface,
the volume above the \textit{z=0} surface must be explicitly taken into account to
better approximate the total ocean mass and thus the steric sea level:

\[
  \eta_s = - \frac{ \sum_{i,\,j,\,k} d_a\; e_{1t}e_{2t}e_{3t} + \sum_{i,\,j} d_a\; e_{1t}e_{2t} \eta }
                  { \sum_{i,\,j,\,k}       e_{1t}e_{2t}e_{3t} + \sum_{i,\,j}       e_{1t}e_{2t} \eta }
  % \label{eq:discrete_steric_Bq_fs}
\]

The fourth and last remark concerns the effective sea level and the presence of sea-ice.
In the real ocean, sea ice (and snow above it)  depresses the liquid seawater through its mass loading.
This depression is a result of the mass of sea ice/snow system acting on the liquid ocean.
There is, however, no dynamical effect associated with these depressions in the liquid ocean sea level,
so that there are no associated ocean currents.
Hence, the dynamically relevant sea level is the effective sea level,
\ie the sea level as if sea ice (and snow) were converted to liquid seawater \citep{Campin_al_OM08}.
However, in the current version of \NEMO the sea-ice is levitating above the ocean without mass exchanges between
ice and ocean.
Therefore the model effective sea level is always given by $\eta + \eta_s$, whether or not there is sea ice present.

In AR5 outputs, the thermosteric sea level is demanded.
It is steric sea level due to changes in ocean density arising just from changes in temperature.
It is given by:

\[
  \eta_s = - \frac{1}{\mathcal{A}} \int_D d_a(T,S_o,p_o) \,dv
  % \label{eq:thermosteric_Bq}
\]

where $S_o$ and $p_o$ are the initial salinity and pressure, respectively.

Both steric and thermosteric sea level are computed in \mdl{diaar5} which needs the \key{diaar5} defined to
be called.

% -------------------------------------------------------------------------------------------------------------
%       Other Diagnostics
% -------------------------------------------------------------------------------------------------------------
\section{Other diagnostics (\protect\key{diahth}, \protect\key{diaar5})}
\label{sec:DIA_diag_others}

Aside from the standard model variables, other diagnostics can be computed on-line.
The available ready-to-add diagnostics modules can be found in directory DIA.

\subsection{Depth of various quantities (\protect\mdl{diahth})}

Among the available diagnostics the following ones are obtained when defining the \key{diahth} CPP key:

- the mixed layer depth (based on a density criterion \citep{de_Boyer_Montegut_al_JGR04}) (\mdl{diahth})

- the turbocline depth (based on a turbulent mixing coefficient criterion) (\mdl{diahth})

- the depth of the 20\deg{C} isotherm (\mdl{diahth})

- the depth of the thermocline (maximum of the vertical temperature gradient) (\mdl{diahth})

% -----------------------------------------------------------
%     Poleward heat and salt transports
% -----------------------------------------------------------

\subsection{Poleward heat and salt transports (\protect\mdl{diaptr})}

%------------------------------------------namptr-----------------------------------------

\nlst{namptr} 
%-----------------------------------------------------------------------------------------

The poleward heat and salt transports, their advective and diffusive component,
and the meriodional stream function can be computed on-line in \mdl{diaptr} \np{ln\_diaptr} to true
(see the \textit{\ngn{namptr} } namelist below).
When \np{ln\_subbas}\forcode{ = .true.}, transports and stream function are computed for the Atlantic, Indian,
Pacific and Indo-Pacific Oceans (defined north of 30\deg{S}) as well as for the World Ocean.
The sub-basin decomposition requires an input file (\ifile{subbasins}) which contains three 2D mask arrays,
the Indo-Pacific mask been deduced from the sum of the Indian and Pacific mask (\autoref{fig:mask_subasins}).

%>>>>>>>>>>>>>>>>>>>>>>>>>>>>
\begin{figure}[!t]
  \begin{center}
    \includegraphics[width=1.0\textwidth]{Fig_mask_subasins}
    \caption{
      \protect\label{fig:mask_subasins}
		Decomposition of the World Ocean (here ORCA2) into sub-basin used in to
      compute the heat and salt transports as well as the meridional stream-function:
      Atlantic basin (red), Pacific basin (green), Indian basin (bleue), Indo-Pacific basin (bleue+green).
		Note that semi-enclosed seas (Red, Med and Baltic seas) as well as Hudson Bay are removed from the sub-basins.
      Note also that the Arctic Ocean has been split into Atlantic and Pacific basins along the North fold line.
    }
  \end{center}
\end{figure}  
%>>>>>>>>>>>>>>>>>>>>>>>>>>>>

% -----------------------------------------------------------
%       CMIP specific diagnostics 
% -----------------------------------------------------------
\subsection{CMIP specific diagnostics (\protect\mdl{diaar5})}

A series of diagnostics has been added in the \mdl{diaar5}.
They corresponds to outputs that are required for AR5 simulations (CMIP5)
(see also \autoref{sec:DIA_steric} for one of them).
Activating those outputs requires to define the \key{diaar5} CPP key.

% -----------------------------------------------------------
%       25 hour mean and hourly Surface, Mid and Bed 
% -----------------------------------------------------------
\subsection{25 hour mean output for tidal models}

%------------------------------------------nam_dia25h-------------------------------------

\nlst{nam_dia25h}
%-----------------------------------------------------------------------------------------

A module is available to compute a crudely detided M2 signal by obtaining a 25 hour mean.
The 25 hour mean is available for daily runs by summing up the 25 hourly instantananeous hourly values from
midnight at the start of the day to midight at the day end.
This diagnostic is actived with the logical $ln\_dia25h$.

% -----------------------------------------------------------
%     Top Middle and Bed hourly output
% -----------------------------------------------------------
\subsection{Top middle and bed hourly output}

%------------------------------------------nam_diatmb-----------------------------------------------------

\nlst{nam_diatmb}
%----------------------------------------------------------------------------------------------------------

A module is available to output the surface (top), mid water and bed diagnostics of a set of standard variables.
This can be a useful diagnostic when hourly or sub-hourly output is required in high resolution tidal outputs.
The tidal signal is retained but the overall data usage is cut to just three vertical levels.
Also the bottom level is calculated for each cell.
This diagnostic is actived with the logical $ln\_diatmb$.

% -----------------------------------------------------------
%     Courant numbers
% -----------------------------------------------------------
\subsection{Courant numbers}

Courant numbers provide a theoretical indication of the model's numerical stability.
The advective Courant numbers can be calculated according to

\[
  C_u = |u|\frac{\rdt}{e_{1u}}, \quad C_v = |v|\frac{\rdt}{e_{2v}}, \quad C_w = |w|\frac{\rdt}{e_{3w}}
  % \label{eq:CFL}
\]

in the zonal, meridional and vertical directions respectively.
The vertical component is included although it is not strictly valid as the vertical velocity is calculated from
the continuity equation rather than as a prognostic variable.
Physically this represents the rate at which information is propogated across a grid cell.
Values greater than 1 indicate that information is propagated across more than one grid cell in a single time step.

The variables can be activated by setting the \np{nn\_diacfl} namelist parameter to 1 in the \ngn{namctl} namelist.
The diagnostics will be written out to an ascii file named cfl\_diagnostics.ascii.
In this file the maximum value of $C_u$, $C_v$, and $C_w$ are printed at each timestep along with the coordinates of
where the maximum value occurs.
At the end of the model run the maximum value of $C_u$, $C_v$, and $C_w$ for the whole model run is printed along
with the coordinates of each.
The maximum values from the run are also copied to the ocean.output file. 

% ================================================================

\biblio

\pindex

\end{document}
