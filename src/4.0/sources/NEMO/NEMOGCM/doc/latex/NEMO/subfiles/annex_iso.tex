\documentclass[../main/NEMO_manual]{subfiles}

%% Local cmds
%\newcommand{\rML}[1][i]{\ensuremath{_{\mathrm{ML}\,#1}}}
%\newcommand{\rMLt}[1][i]{\tilde{r}_{\mathrm{ML}\,#1}}
%\newcommand{\triad}[6][]{\ensuremath{{}_{#2}^{#3}{\mathbb{#4}_{#1}}_{#5}^{\,#6}}}
%\newcommand{\triadd}[5]{\ensuremath{{}_{#1}^{#2}{\mathbb{#3}}_{#4}^{\,#5}}}
%\newcommand{\triadt}[5]{\ensuremath{{}_{#1}^{#2}{\tilde{\mathbb{#3}}}_{#4}^{\,#5}}}
%\newcommand{\rtriad}[2][]{\ensuremath{\triad[#1]{i}{k}{#2}{i_p}{k_p}}}
%\newcommand{\rtriadt}[1]{\ensuremath{\triadt{i}{k}{#1}{i_p}{k_p}}}

\begin{document}
% ================================================================
% Iso-neutral diffusion :
% ================================================================
\chapter{Iso-Neutral Diffusion and Eddy Advection using Triads}
\label{apdx:triad}

\minitoc

\newpage

\section{Choice of \protect\ngn{namtra\_ldf} namelist parameters}
%-----------------------------------------nam_traldf------------------------------------------------------

\nlst{namtra_ldf}
%---------------------------------------------------------------------------------------------------------

Two scheme are available to perform the iso-neutral diffusion.
If the namelist logical \np{ln\_traldf\_triad} is set true,
\NEMO updates both active and passive tracers using the Griffies triad representation of iso-neutral diffusion and
the eddy-induced advective skew (GM) fluxes.
If the namelist logical \np{ln\_traldf\_iso} is set true,
the filtered version of Cox's original scheme (the Standard scheme) is employed (\autoref{sec:LDF_slp}).
In the present implementation of the Griffies scheme,
the advective skew fluxes are implemented even if \np{ln\_traldf\_eiv} is false.

Values of iso-neutral diffusivity and GM coefficient are set as described in \autoref{sec:LDF_coef}.
Note that when GM fluxes are used, the eddy-advective (GM) velocities are output for diagnostic purposes using xIOS,
even though the eddy advection is accomplished by means of the skew fluxes.

The options specific to the Griffies scheme include:
\begin{description}
\item[\np{ln\_triad\_iso}]
  See \autoref{sec:taper}.
  If this is set false (the default),
  then `iso-neutral' mixing is accomplished within the surface mixed-layer along slopes linearly decreasing with
  depth from the value immediately below the mixed-layer to zero (flat) at the surface (\autoref{sec:lintaper}).
  This is the same treatment as used in the default implementation
  \autoref{subsec:LDF_slp_iso}; \autoref{fig:eiv_slp}.
  Where \np{ln\_triad\_iso} is set true,
  the vertical skew flux is further reduced to ensure no vertical buoyancy flux,
  giving an almost pure horizontal diffusive tracer flux within the mixed layer.
  This is similar to the tapering suggested by \citet{Gerdes1991}. See \autoref{subsec:Gerdes-taper}
\item[\np{ln\_botmix\_triad}]
  See \autoref{sec:iso_bdry}. 
  If this is set false (the default) then the lateral diffusive fluxes
  associated with triads partly masked by topography are neglected. 
  If it is set true, however, then these lateral diffusive fluxes are applied, 
  giving smoother bottom tracer fields at the cost of introducing diapycnal mixing.
\item[\np{rn\_sw\_triad}]
  blah blah to be added....
\end{description}
The options shared with the Standard scheme include:
\begin{description}
\item[\np{ln\_traldf\_msc}]   blah blah to be added
\item[\np{rn\_slpmax}]  blah blah to be added
\end{description}

\section{Triad formulation of iso-neutral diffusion}
\label{sec:iso}

We have implemented into \NEMO a scheme inspired by \citet{Griffies_al_JPO98},
but formulated within the \NEMO framework, using scale factors rather than grid-sizes.

\subsection{Iso-neutral diffusion operator}

The iso-neutral second order tracer diffusive operator for small angles between
iso-neutral surfaces and geopotentials is given by \autoref{eq:iso_tensor_1}:
\begin{subequations}
  \label{eq:iso_tensor_1}
  \begin{equation}
    D^{lT}=-\nabla \cdot\vect{f}^{lT}\equiv
    -\frac{1}{e_1e_2e_3}\left[\pd{i}\left (f_1^{lT}e_2e_3\right) +
      \pd{j}\left (f_2^{lT}e_2e_3\right) + \pd{k}\left (f_3^{lT}e_1e_2\right)\right],
  \end{equation}
  where the diffusive flux per unit area of physical space
  \begin{equation}
    \vect{f}^{lT}=-{A^{lT}}\Re\cdot\nabla T,
  \end{equation}
  \begin{equation}
    \label{eq:iso_tensor_2}
    \mbox{with}\quad \;\;\Re =
    \begin{pmatrix}
      1   &  0   & -r_1           \rule[-.9 em]{0pt}{1.79 em} \\
      0   &  1   & -r_2           \rule[-.9 em]{0pt}{1.79 em} \\
      -r_1 & -r_2 &  r_1 ^2+r_2 ^2 \rule[-.9 em]{0pt}{1.79 em}
    \end{pmatrix}
    \quad \text{and} \quad\nabla T=
    \begin{pmatrix}
      \frac{1}{e_1} \pd[T]{i} \rule[-.9 em]{0pt}{1.79 em} \\
      \frac{1}{e_2} \pd[T]{j} \rule[-.9 em]{0pt}{1.79 em} \\
      \frac{1}{e_3} \pd[T]{k} \rule[-.9 em]{0pt}{1.79 em}
    \end{pmatrix}
    .
  \end{equation}
\end{subequations}
% \left( {{\begin{array}{*{20}c}
%  1 \hfill & 0 \hfill & {-r_1 } \hfill \\
%  0 \hfill & 1 \hfill & {-r_2 } \hfill \\
%  {-r_1 } \hfill & {-r_2 } \hfill & {r_1 ^2+r_2 ^2} \hfill \\
% \end{array} }} \right)
Here \autoref{eq:PE_iso_slopes} 
\begin{align*}
  r_1 &=-\frac{e_3 }{e_1 } \left( \frac{\partial \rho }{\partial i}
        \right)
        \left( {\frac{\partial \rho }{\partial k}} \right)^{-1} \\
      &=-\frac{e_3 }{e_1 } \left( -\alpha\frac{\partial T }{\partial i} +
        \beta\frac{\partial S }{\partial i} \right) \left(
        -\alpha\frac{\partial T }{\partial k} + \beta\frac{\partial S
        }{\partial k} \right)^{-1}
\end{align*}
is the $i$-component of the slope of the iso-neutral surface relative to the computational surface,
and $r_2$ is the $j$-component.

We will find it useful to consider the fluxes per unit area in $i,j,k$ space; we write
\[
  % \label{eq:Fijk}
  \vect{F}_{\mathrm{iso}}=\left(f_1^{lT}e_2e_3, f_2^{lT}e_1e_3, f_3^{lT}e_1e_2\right).
\]
Additionally, we will sometimes write the contributions towards the fluxes $\vect{f}$ and
$\vect{F}_{\mathrm{iso}}$ from the component $R_{ij}$ of $\Re$ as $f_{ij}$, $F_{\mathrm{iso}\: ij}$,
with $f_{ij}=R_{ij}e_i^{-1}\partial T/\partial x_i$ (no summation) etc.

The off-diagonal terms of the small angle diffusion tensor
\autoref{eq:iso_tensor_1}, \autoref{eq:iso_tensor_2} produce skew-fluxes along
the $i$- and $j$-directions resulting from the vertical tracer gradient:
\begin{align}
  \label{eq:i13c}
  f_{13}=&+{A^{lT}} r_1\frac{1}{e_3}\frac{\partial T}{\partial k},\qquad f_{23}=+{A^{lT}} r_2\frac{1}{e_3}\frac{\partial T}{\partial k}\\
  \intertext{and in the k-direction resulting from the lateral tracer gradients}
  \label{eq:i31c}
  f_{31}+f_{32}=& {A^{lT}} r_1\frac{1}{e_1}\frac{\partial T}{\partial i}+{A^{lT}} r_2\frac{1}{e_1}\frac{\partial T}{\partial i}
\end{align}

The vertical diffusive flux associated with the $_{33}$ component of the small angle diffusion tensor is
\begin{equation}
  \label{eq:i33c}
  f_{33}=-{A^{lT}}(r_1^2 +r_2^2) \frac{1}{e_3}\frac{\partial T}{\partial k}.
\end{equation}

Since there are no cross terms involving $r_1$ and $r_2$ in the above,
we can consider the iso-neutral diffusive fluxes separately in the $i$-$k$ and $j$-$k$ planes,
just adding together the vertical components from each plane.
The following description will describe the fluxes on the $i$-$k$ plane.

There is no natural discretization for the $i$-component of the skew-flux, \autoref{eq:i13c},
as although it must be evaluated at $u$-points,
it involves vertical gradients (both for the tracer and the slope $r_1$), defined at $w$-points.
Similarly, the vertical skew flux, \autoref{eq:i31c},
is evaluated at $w$-points but involves horizontal gradients defined at $u$-points.

\subsection{Standard discretization}

The straightforward approach to discretize the lateral skew flux
\autoref{eq:i13c} from tracer cell $i,k$ to $i+1,k$, introduced in 1995 into OPA,
\autoref{eq:tra_ldf_iso}, is to calculate a mean vertical gradient at the $u$-point from
the average of the four surrounding vertical tracer gradients, and multiply this by a mean slope at the $u$-point,
calculated from the averaged surrounding vertical density gradients.
The total area-integrated skew-flux (flux per unit area in $ijk$ space) from tracer cell $i,k$ to $i+1,k$,
noting that the $e_{{3}_{i+1/2}^k}$ in the area $e{_{3}}_{i+1/2}^k{e_{2}}_{i+1/2}i^k$ at the $u$-point cancels out with
the $1/{e_{3}}_{i+1/2}^k$ associated with the vertical tracer gradient, is then \autoref{eq:tra_ldf_iso}
\[
  \left(F_u^{13} \right)_{i+\frac{1}{2}}^k = {A}_{i+\frac{1}{2}}^k
  {e_{2}}_{i+1/2}^k \overline{\overline
    r_1} ^{\,i,k}\,\overline{\overline{\delta_k T}}^{\,i,k},
\]
where
\[
  \overline{\overline
    r_1} ^{\,i,k} = -\frac{{e_{3u}}_{i+1/2}^k}{{e_{1u}}_{i+1/2}^k}
  \frac{\delta_{i+1/2} [\rho]}{\overline{\overline{\delta_k \rho}}^{\,i,k}},
\]
and here and in the following we drop the $^{lT}$ superscript from ${A^{lT}}$ for simplicity.
Unfortunately the resulting combination $\overline{\overline{\delta_k\bullet}}^{\,i,k}$ of a $k$ average and
a $k$ difference of the tracer reduces to $\bullet_{k+1}-\bullet_{k-1}$,
so two-grid-point oscillations are invisible to this discretization of the iso-neutral operator.
These \emph{computational modes} will not be damped by this operator, and may even possibly be amplified by it.
Consequently, applying this operator to a tracer does not guarantee the decrease of its global-average variance.
To correct this, we introduced a smoothing of the slopes of the iso-neutral surfaces (see \autoref{chap:LDF}).
This technique works for $T$ and $S$ in so far as they are active tracers
(\ie they enter the computation of density), but it does not work for a passive tracer.

\subsection{Expression of the skew-flux in terms of triad slopes}

\citep{Griffies_al_JPO98} introduce a different discretization of the off-diagonal terms that
nicely solves the problem.
% Instead of multiplying the mean slope calculated at the $u$-point by
% the mean vertical gradient at the $u$-point,
% >>>>>>>>>>>>>>>>>>>>>>>>>>>>
\begin{figure}[tb]
  \begin{center}
    \includegraphics[width=1.05\textwidth]{Fig_GRIFF_triad_fluxes}
    \caption{
      \protect\label{fig:ISO_triad}
      (a) Arrangement of triads $S_i$ and tracer gradients to
      give lateral tracer flux from box $i,k$ to $i+1,k$
      (b) Triads $S'_i$ and tracer gradients to give vertical tracer flux from
      box $i,k$ to $i,k+1$.
    }
  \end{center}
\end{figure}
% >>>>>>>>>>>>>>>>>>>>>>>>>>>>
They get the skew flux from the products of the vertical gradients at each $w$-point surrounding the $u$-point with
the corresponding `triad' slope calculated from the lateral density gradient across the $u$-point divided by
the vertical density gradient at the same $w$-point as the tracer gradient.
See \autoref{fig:ISO_triad}a, where the thick lines denote the tracer gradients,
and the thin lines the corresponding triads, with slopes $s_1, \dotsc s_4$.
The total area-integrated skew-flux from tracer cell $i,k$ to $i+1,k$
\begin{multline}
  \label{eq:i13}
  \left( F_u^{13}  \right)_{i+\frac{1}{2}}^k = {A}_{i+1}^k a_1 s_1
  \delta_{k+\frac{1}{2}} \left[ T^{i+1}
  \right]/e_{{3w}_{i+1}}^{k+\frac{1}{2}}  + {A} _i^k a_2 s_2 \delta
  _{k+\frac{1}{2}} \left[ T^i
  \right]/e_{{3w}_{i+1}}^{k+\frac{1}{2}} \\
  +{A} _{i+1}^k a_3 s_3 \delta_{k-\frac{1}{2}} \left[ T^{i+1}
  \right]/e_{{3w}_{i+1}}^{k+\frac{1}{2}}  +{A} _i^k a_4 s_4 \delta
  _{k-\frac{1}{2}} \left[ T^i \right]/e_{{3w}_{i+1}}^{k+\frac{1}{2}},
\end{multline}
where the contributions of the triad fluxes are weighted by areas $a_1, \dotsc a_4$,
and ${A}$ is now defined at the tracer points rather than the $u$-points.
This discretization gives a much closer stencil, and disallows the two-point computational modes.

The vertical skew flux \autoref{eq:i31c} from tracer cell $i,k$ to $i,k+1$ at
the $w$-point $i,k+\frac{1}{2}$ is constructed similarly (\autoref{fig:ISO_triad}b) by
multiplying lateral tracer gradients from each of the four surrounding $u$-points by the appropriate triad slope:
\begin{multline}
  \label{eq:i31}
  \left( F_w^{31} \right) _i ^{k+\frac{1}{2}} =  {A}_i^{k+1} a_{1}'
  s_{1}' \delta_{i-\frac{1}{2}} \left[ T^{k+1} \right]/{e_{3u}}_{i-\frac{1}{2}}^{k+1}
  +{A}_i^{k+1} a_{2}' s_{2}' \delta_{i+\frac{1}{2}} \left[ T^{k+1} \right]/{e_{3u}}_{i+\frac{1}{2}}^{k+1} \\
  + {A}_i^k a_{3}' s_{3}' \delta_{i-\frac{1}{2}} \left[ T^k\right]/{e_{3u}}_{i-\frac{1}{2}}^k
  +{A}_i^k a_{4}' s_{4}' \delta_{i+\frac{1}{2}} \left[ T^k \right]/{e_{3u}}_{i+\frac{1}{2}}^k.
\end{multline}

We notate the triad slopes $s_i$ and $s'_i$ in terms of the `anchor point' $i,k$
(appearing in both the vertical and lateral gradient),
and the $u$- and $w$-points $(i+i_p,k)$, $(i,k+k_p)$ at the centres of the `arms' of the triad as follows
(see also \autoref{fig:ISO_triad}):
\begin{equation}
  \label{eq:R}
  _i^k \mathbb{R}_{i_p}^{k_p}
  =-\frac{ {e_{3w}}_{\,i}^{\,k+k_p}} { {e_{1u}}_{\,i+i_p}^{\,k}}
  \
  \frac
  { \alpha_i^k  \ \delta_{i+i_p}[T^k] - \beta_i^k \ \delta_{i+i_p}[S^k] }
  { \alpha_i^k  \ \delta_{k+k_p}[T^i] - \beta_i^k \ \delta_{k+k_p}[S^i] }.
\end{equation}
In calculating the slopes of the local neutral surfaces,
the expansion coefficients $\alpha$ and $\beta$ are evaluated at the anchor points of the triad,
while the metrics are calculated at the $u$- and $w$-points on the arms.

% >>>>>>>>>>>>>>>>>>>>>>>>>>>>
\begin{figure}[tb]
  \begin{center}
    \includegraphics[width=0.80\textwidth]{Fig_GRIFF_qcells}
    \caption{
      \protect\label{fig:qcells}
      Triad notation for quarter cells. $T$-cells are inside boxes,
      while the  $i+\fractext{1}{2},k$ $u$-cell is shaded in green and
      the $i,k+\fractext{1}{2}$ $w$-cell is shaded in pink.
    }
  \end{center}
\end{figure}
% >>>>>>>>>>>>>>>>>>>>>>>>>>>>

Each triad $\{_i^{k}\:_{i_p}^{k_p}\}$ is associated (\autoref{fig:qcells}) with the quarter cell that is
the intersection of the $i,k$ $T$-cell, the $i+i_p,k$ $u$-cell and the $i,k+k_p$ $w$-cell.
Expressing the slopes $s_i$ and $s'_i$ in \autoref{eq:i13} and \autoref{eq:i31} in this notation,
we have \eg \ $s_1=s'_1={\:}_i^k \mathbb{R}_{1/2}^{1/2}$.
Each triad slope $_i^k\mathbb{R}_{i_p}^{k_p}$ is used once (as an $s$) to
calculate the lateral flux along its $u$-arm, at $(i+i_p,k)$,
and then again as an $s'$ to calculate the vertical flux along its $w$-arm at $(i,k+k_p)$.
Each vertical area $a_i$ used to calculate the lateral flux and horizontal area $a'_i$ used to
calculate the vertical flux can also be identified as the area across the $u$- and $w$-arms of a unique triad,
and we notate these areas, similarly to the triad slopes,
as $_i^k{\mathbb{A}_u}_{i_p}^{k_p}$, $_i^k{\mathbb{A}_w}_{i_p}^{k_p}$,
where \eg in \autoref{eq:i13} $a_{1}={\:}_i^k{\mathbb{A}_u}_{1/2}^{1/2}$,
and in \autoref{eq:i31} $a'_{1}={\:}_i^k{\mathbb{A}_w}_{1/2}^{1/2}$.

\subsection{Full triad fluxes}

A key property of iso-neutral diffusion is that it should not affect the (locally referenced) density.
In particular there should be no lateral or vertical density flux.
The lateral density flux disappears so long as the area-integrated lateral diffusive flux from
tracer cell $i,k$ to $i+1,k$ coming from the $_{11}$ term of the diffusion tensor takes the form
\begin{equation}
  \label{eq:i11}
  \left( F_u^{11} \right) _{i+\frac{1}{2}} ^{k} =
  - \left( {A}_i^{k+1} a_{1} + {A}_i^{k+1} a_{2} + {A}_i^k
    a_{3} + {A}_i^k a_{4} \right)
  \frac{\delta_{i+1/2} \left[ T^k\right]}{{e_{1u}}_{\,i+1/2}^{\,k}},
\end{equation}
where the areas $a_i$ are as in \autoref{eq:i13}.
In this case, separating the total lateral flux, the sum of \autoref{eq:i13} and \autoref{eq:i11},
into triad components, a lateral tracer flux
\begin{equation}
  \label{eq:latflux-triad}
  _i^k {\mathbb{F}_u}_{i_p}^{k_p} (T) = - {A}_i^k{ \:}_i^k{\mathbb{A}_u}_{i_p}^{k_p}
  \left(
    \frac{ \delta_{i+ i_p}[T^k] }{ {e_{1u}}_{\,i + i_p}^{\,k} }
    -\ {_i^k\mathbb{R}_{i_p}^{k_p}} \
    \frac{ \delta_{k+k_p} [T^i] }{{e_{3w}}_{\,i}^{\,k+k_p} }
  \right)
\end{equation}
can be identified with each triad.
Then, because the same metric factors ${e_{3w}}_{\,i}^{\,k+k_p}$ and ${e_{1u}}_{\,i+i_p}^{\,k}$ are employed for both
the density gradients in $ _i^k \mathbb{R}_{i_p}^{k_p}$ and the tracer gradients,
the lateral density flux associated with each triad separately disappears.
\begin{equation}
  \label{eq:latflux-rho}
  {\mathbb{F}_u}_{i_p}^{k_p} (\rho)=-\alpha _i^k {\:}_i^k {\mathbb{F}_u}_{i_p}^{k_p} (T) + \beta_i^k {\:}_i^k {\mathbb{F}_u}_{i_p}^{k_p} (S)=0
\end{equation}
Thus the total flux $\left( F_u^{31} \right) ^i _{i,k+\frac{1}{2}} + \left( F_u^{11} \right) ^i _{i,k+\frac{1}{2}}$ from
tracer cell $i,k$ to $i+1,k$ must also vanish since it is a sum of four such triad fluxes.

The squared slope $r_1^2$ in the expression \autoref{eq:i33c} for the $_{33}$ component is also expressed in
terms of area-weighted squared triad slopes,
so the area-integrated vertical flux from tracer cell $i,k$ to $i,k+1$ resulting from the $r_1^2$ term is
\begin{equation}
  \label{eq:i33}
  \left( F_w^{33} \right) _i^{k+\frac{1}{2}} =
  - \left( {A}_i^{k+1} a_{1}' s_{1}'^2
    + {A}_i^{k+1} a_{2}' s_{2}'^2
    + {A}_i^k a_{3}' s_{3}'^2
    + {A}_i^k a_{4}' s_{4}'^2 \right)\delta_{k+\frac{1}{2}} \left[ T^{i+1} \right],
\end{equation}
where the areas $a'$ and slopes $s'$ are the same as in \autoref{eq:i31}.
Then, separating the total vertical flux, the sum of \autoref{eq:i31} and \autoref{eq:i33},
into triad components, a vertical flux
\begin{align}
  \label{eq:vertflux-triad}
  _i^k {\mathbb{F}_w}_{i_p}^{k_p} (T)
  &= {A}_i^k{\: }_i^k{\mathbb{A}_w}_{i_p}^{k_p}
    \left(
    {_i^k\mathbb{R}_{i_p}^{k_p}}\frac{ \delta_{i+ i_p}[T^k] }{ {e_{1u}}_{\,i + i_p}^{\,k} }
    -\ \left({_i^k\mathbb{R}_{i_p}^{k_p}}\right)^2 \
    \frac{ \delta_{k+k_p} [T^i] }{{e_{3w}}_{\,i}^{\,k+k_p} }
    \right) \\
  &= - \left(\left.{ }_i^k{\mathbb{A}_w}_{i_p}^{k_p}\right/{ }_i^k{\mathbb{A}_u}_{i_p}^{k_p}\right)
    {_i^k\mathbb{R}_{i_p}^{k_p}}{\: }_i^k{\mathbb{F}_u}_{i_p}^{k_p} (T) \label{eq:vertflux-triad2}
\end{align}
may be associated with each triad.
Each vertical density flux $_i^k {\mathbb{F}_w}_{i_p}^{k_p} (\rho)$ associated with a triad then
separately disappears (because the lateral flux $_i^k{\mathbb{F}_u}_{i_p}^{k_p} (\rho)$ disappears).
Consequently the total vertical density flux
$\left( F_w^{31} \right)_i ^{k+\frac{1}{2}} + \left( F_w^{33} \right)_i^{k+\frac{1}{2}}$ from
tracer cell $i,k$ to $i,k+1$ must also vanish since it is a sum of four such triad fluxes.

We can explicitly identify (\autoref{fig:qcells}) the triads associated with the $s_i$, $a_i$,
and $s'_i$, $a'_i$ used in the definition of the $u$-fluxes and $w$-fluxes in \autoref{eq:i31},
\autoref{eq:i13}, \autoref{eq:i11} \autoref{eq:i33} and \autoref{fig:ISO_triad} to write out
the iso-neutral fluxes at $u$- and $w$-points as sums of the triad fluxes that cross the $u$- and $w$-faces:
%(\autoref{fig:ISO_triad}):
\begin{flalign}
  \label{eq:iso_flux} \vect{F}_{\mathrm{iso}}(T) &\equiv
  \sum_{\substack{i_p,\,k_p}}
  \begin{pmatrix}
    {_{i+1/2-i_p}^k {\mathbb{F}_u}_{i_p}^{k_p} } (T) \\ \\
    {_i^{k+1/2-k_p} {\mathbb{F}_w}_{i_p}^{k_p} } (T) \\
  \end{pmatrix}.
\end{flalign}

\subsection{Ensuring the scheme does not increase tracer variance}
\label{subsec:variance}

We now require that this operator should not increase the globally-integrated tracer variance.
%This changes according to
% \begin{align*}
% &\int_D  D_l^T \; T \;dv \equiv  \sum_{i,k} \left\{ T \ D_l^T \ b_T \right\}    \\
% &\equiv + \sum_{i,k} \sum_{\substack{i_p,\,k_p}} \left\{
% 		\delta_{i} \left[{_{i+1/2-i_p}^k {\mathbb{F}_u }_{i_p}^{k_p}} \right]
% 	     + \delta_{k} \left[ {_i^{k+1/2-k_p} {\mathbb{F}_w}_{i_p}^{k_p}} \right]  \ T \right\}    \\
% &\equiv  - \sum_{i,k} \sum_{\substack{i_p,\,k_p}} \left\{
%                 {_{i+1/2-i_p}^k {\mathbb{F}_u }_{i_p}^{k_p}} \ \delta_{i+1/2} [T]
%              + {_i^{k+1/2-k_p} {\mathbb{F}_w}_{i_p}^{k_p}}  \ \delta_{k+1/2} [T]   \right\}      \\
% \end{align*}
Each triad slope $_i^k\mathbb{R}_{i_p}^{k_p}$ drives a lateral flux $_i^k{\mathbb{F}_u}_{i_p}^{k_p} (T)$ across
the $u$-point $i+i_p,k$ and a vertical flux $_i^k{\mathbb{F}_w}_{i_p}^{k_p} (T)$ across the $w$-point $i,k+k_p$.
The lateral flux drives a net rate of change of variance,
summed over the two $T$-points $i+i_p-\fractext{1}{2},k$ and $i+i_p+\fractext{1}{2},k$, of
\begin{multline}
  {b_T}_{i+i_p-1/2}^k\left(\frac{\partial T}{\partial t}T\right)_{i+i_p-1/2}^k+
  \quad {b_T}_{i+i_p+1/2}^k\left(\frac{\partial T}{\partial
      t}T\right)_{i+i_p+1/2}^k \\
  \begin{aligned}
    &= -T_{i+i_p-1/2}^k{\;} _i^k{\mathbb{F}_u}_{i_p}^{k_p} (T) \quad + \quad  T_{i+i_p+1/2}^k
    {\;}_i^k{\mathbb{F}_u}_{i_p}^{k_p} (T) \\
    &={\;} _i^k{\mathbb{F}_u}_{i_p}^{k_p} (T)\,\delta_{i+ i_p}[T^k], \label{eq:dvar_iso_i}
  \end{aligned}
\end{multline}
while the vertical flux similarly drives a net rate of change of variance summed over
the $T$-points $i,k+k_p-\fractext{1}{2}$ (above) and $i,k+k_p+\fractext{1}{2}$ (below) of
\begin{equation}
  \label{eq:dvar_iso_k}
  _i^k{\mathbb{F}_w}_{i_p}^{k_p} (T) \,\delta_{k+ k_p}[T^i].
\end{equation}
The total variance tendency driven by the triad is the sum of these two.
Expanding $_i^k{\mathbb{F}_u}_{i_p}^{k_p} (T)$ and $_i^k{\mathbb{F}_w}_{i_p}^{k_p} (T)$ with
\autoref{eq:latflux-triad} and \autoref{eq:vertflux-triad}, it is
\begin{multline*}
  -{A}_i^k\left \{
    { } _i^k{\mathbb{A}_u}_{i_p}^{k_p}
    \left(
      \frac{ \delta_{i+ i_p}[T^k] }{ {e_{1u}}_{\,i + i_p}^{\,k} }
      - {_i^k\mathbb{R}_{i_p}^{k_p}} \
      \frac{ \delta_{k+k_p} [T^i] }{{e_{3w}}_{\,i}^{\,k+k_p} }\right)\,\delta_{i+ i_p}[T^k] \right.\\
  - \left. { } _i^k{\mathbb{A}_w}_{i_p}^{k_p}
    \left(
      \frac{ \delta_{i+ i_p}[T^k] }{ {e_{1u}}_{\,i + i_p}^{\,k} }
      -{\:}_i^k\mathbb{R}_{i_p}^{k_p}
      \frac{ \delta_{k+k_p} [T^i] }{{e_{3w}}_{\,i}^{\,k+k_p} }
    \right) {\,}_i^k\mathbb{R}_{i_p}^{k_p}\delta_{k+ k_p}[T^i]
  \right \}.
\end{multline*}
The key point is then that if we require $_i^k{\mathbb{A}_u}_{i_p}^{k_p}$ and $_i^k{\mathbb{A}_w}_{i_p}^{k_p}$ to
be related to a triad volume $_i^k\mathbb{V}_{i_p}^{k_p}$ by
\begin{equation}
  \label{eq:V-A}
  _i^k\mathbb{V}_{i_p}^{k_p}
  ={\;}_i^k{\mathbb{A}_u}_{i_p}^{k_p}\,{e_{1u}}_{\,i + i_p}^{\,k}
  ={\;}_i^k{\mathbb{A}_w}_{i_p}^{k_p}\,{e_{3w}}_{\,i}^{\,k + k_p},
\end{equation}
the variance tendency reduces to the perfect square
\begin{equation}
  \label{eq:perfect-square}
  -{A}_i^k{\:} _i^k\mathbb{V}_{i_p}^{k_p}
  \left(
    \frac{ \delta_{i+ i_p}[T^k] }{ {e_{1u}}_{\,i + i_p}^{\,k} }
    -{\:}_i^k\mathbb{R}_{i_p}^{k_p}
    \frac{ \delta_{k+k_p} [T^i] }{{e_{3w}}_{\,i}^{\,k+k_p} }
  \right)^2\leq 0.
\end{equation}
Thus, the constraint \autoref{eq:V-A} ensures that the fluxes
(\autoref{eq:latflux-triad}, \autoref{eq:vertflux-triad}) associated with
a given slope triad $_i^k\mathbb{R}_{i_p}^{k_p}$ do not increase the net variance.
Since the total fluxes are sums of such fluxes from the various triads, this constraint, applied to all triads,
is sufficient to ensure that the globally integrated variance does not increase.

The expression \autoref{eq:V-A} can be interpreted as a discretization of the global integral
\begin{equation}
  \label{eq:cts-var}
  \frac{\partial}{\partial t}\int\!\fractext{1}{2} T^2\, dV =
  \int\!\mathbf{F}\cdot\nabla T\, dV,
\end{equation}
where, within each triad volume $_i^k\mathbb{V}_{i_p}^{k_p}$, the lateral and vertical fluxes/unit area
\[
  \mathbf{F}=\left(
    \left.{}_i^k{\mathbb{F}_u}_{i_p}^{k_p} (T)\right/{}_i^k{\mathbb{A}_u}_{i_p}^{k_p},
    \left.{\:}_i^k{\mathbb{F}_w}_{i_p}^{k_p} (T)\right/{}_i^k{\mathbb{A}_w}_{i_p}^{k_p}
  \right)
\]
and the gradient
\[
  \nabla T = \left(
    \left.\delta_{i+ i_p}[T^k] \right/ {e_{1u}}_{\,i + i_p}^{\,k},
    \left.\delta_{k+ k_p}[T^i] \right/ {e_{3w}}_{\,i}^{\,k + k_p}
  \right)
\]

\subsection{Triad volumes in Griffes's scheme and in \NEMO}

To complete the discretization we now need only specify the triad volumes $_i^k\mathbb{V}_{i_p}^{k_p}$.
\citet{Griffies_al_JPO98} identifies these $_i^k\mathbb{V}_{i_p}^{k_p}$ as the volumes of the quarter cells,
defined in terms of the distances between $T$, $u$,$f$ and $w$-points.
This is the natural discretization of \autoref{eq:cts-var}.
The \NEMO model, however, operates with scale factors instead of grid sizes,
and scale factors for the quarter cells are not defined.
Instead, therefore we simply choose
\begin{equation}
  \label{eq:V-NEMO}
  _i^k\mathbb{V}_{i_p}^{k_p}=\fractext{1}{4} {b_u}_{i+i_p}^k,
\end{equation}
as a quarter of the volume of the $u$-cell inside which the triad quarter-cell lies.
This has the nice property that when the slopes $\mathbb{R}$ vanish,
the lateral flux from tracer cell $i,k$ to $i+1,k$ reduces to the classical form
\begin{equation}
  \label{eq:lat-normal}
  -\overline{A}_{\,i+1/2}^k\;
  \frac{{b_u}_{i+1/2}^k}{{e_{1u}}_{\,i + i_p}^{\,k}}
  \;\frac{\delta_{i+ 1/2}[T^k] }{{e_{1u}}_{\,i + i_p}^{\,k}}
  = -\overline{A}_{\,i+1/2}^k\;\frac{{e_{1w}}_{\,i + 1/2}^{\,k}\:{e_{1v}}_{\,i + 1/2}^{\,k}\;\delta_{i+ 1/2}[T^k]}{{e_{1u}}_{\,i + 1/2}^{\,k}}.
\end{equation}
In fact if the diffusive coefficient is defined at $u$-points,
so that we employ ${A}_{i+i_p}^k$ instead of  ${A}_i^k$ in the definitions of the triad fluxes
\autoref{eq:latflux-triad} and \autoref{eq:vertflux-triad},
we can replace $\overline{A}_{\,i+1/2}^k$ by $A_{i+1/2}^k$ in the above.

\subsection{Summary of the scheme}

The iso-neutral fluxes at $u$- and $w$-points are the sums of the triad fluxes that
cross the $u$- and $w$-faces \autoref{eq:iso_flux}:
\begin{subequations}
  % \label{eq:alltriadflux}
  \begin{flalign*}
    % \label{eq:vect_isoflux}
    \vect{F}_{\mathrm{iso}}(T) &\equiv
    \sum_{\substack{i_p,\,k_p}}
    \begin{pmatrix}
      {_{i+1/2-i_p}^k {\mathbb{F}_u}_{i_p}^{k_p} } (T) \\ \\
      {_i^{k+1/2-k_p} {\mathbb{F}_w}_{i_p}^{k_p} } (T)
    \end{pmatrix},
  \end{flalign*}
  where \autoref{eq:latflux-triad}:
  \begin{align}
    \label{eq:triadfluxu}
    _i^k {\mathbb{F}_u}_{i_p}^{k_p} (T) &= - {A}_i^k{
                                          \:}\frac{{{}_i^k\mathbb{V}}_{i_p}^{k_p}}{{e_{1u}}_{\,i + i_p}^{\,k}}
                                          \left(
                                          \frac{ \delta_{i+ i_p}[T^k] }{ {e_{1u}}_{\,i + i_p}^{\,k} }
                                          -\ {_i^k\mathbb{R}_{i_p}^{k_p}} \
                                          \frac{ \delta_{k+k_p} [T^i] }{{e_{3w}}_{\,i}^{\,k+k_p} }
                                          \right),\\
    \intertext{and}
    _i^k {\mathbb{F}_w}_{i_p}^{k_p} (T)
                                        &= {A}_i^k{\: }\frac{{{}_i^k\mathbb{V}}_{i_p}^{k_p}}{{e_{3w}}_{\,i}^{\,k+k_p}}
                                          \left(
                                          {_i^k\mathbb{R}_{i_p}^{k_p}}\frac{ \delta_{i+ i_p}[T^k] }{ {e_{1u}}_{\,i + i_p}^{\,k} }
                                          -\ \left({_i^k\mathbb{R}_{i_p}^{k_p}}\right)^2 \
                                          \frac{ \delta_{k+k_p} [T^i] }{{e_{3w}}_{\,i}^{\,k+k_p} }
                                          \right),\label{eq:triadfluxw}
  \end{align}
  with \autoref{eq:V-NEMO}
  \[
    % \label{eq:V-NEMO2}
    _i^k{\mathbb{V}}_{i_p}^{k_p}=\fractext{1}{4} {b_u}_{i+i_p}^k.
  \]
\end{subequations}

The divergence of the expression \autoref{eq:iso_flux} for the fluxes gives the iso-neutral diffusion tendency at
each tracer point:
\[
  % \label{eq:iso_operator}
  D_l^T = \frac{1}{b_T}
  \sum_{\substack{i_p,\,k_p}} \left\{ \delta_{i} \left[{_{i+1/2-i_p}^k
        {\mathbb{F}_u }_{i_p}^{k_p}} \right] + \delta_{k} \left[
      {_i^{k+1/2-k_p} {\mathbb{F}_w}_{i_p}^{k_p}} \right] \right\}
\]
where $b_T= e_{1T}\,e_{2T}\,e_{3T}$ is the volume of $T$-cells.
The diffusion scheme satisfies the following six properties:
\begin{description}
\item[$\bullet$ horizontal diffusion]
  The discretization of the diffusion operator recovers the traditional five-point Laplacian
  \autoref{eq:lat-normal} in the limit of flat iso-neutral direction:
  \[
    % \label{eq:iso_property0}
    D_l^T = \frac{1}{b_T} \
    \delta_{i} \left[ \frac{e_{2u}\,e_{3u}}{e_{1u}} \;
      \overline{A}^{\,i} \; \delta_{i+1/2}[T] \right] \qquad
    \text{when} \quad { _i^k \mathbb{R}_{i_p}^{k_p} }=0
  \]

\item[$\bullet$ implicit treatment in the vertical]
  Only tracer values associated with a single water column appear in the expression \autoref{eq:i33} for
  the $_{33}$ fluxes, vertical fluxes driven by vertical gradients.
  This is of paramount importance since it means that a time-implicit algorithm can be used to
  solve the vertical diffusion equation.
  This is necessary since the vertical eddy diffusivity associated with this term,
  \[
    \frac{1}{b_w}\sum_{\substack{i_p, \,k_p}} \left\{
      {\:}_i^k\mathbb{V}_{i_p}^{k_p} \: {A}_i^k \: \left(_i^k \mathbb{R}_{i_p}^{k_p}\right)^2
    \right\}  =
    \frac{1}{4b_w}\sum_{\substack{i_p, \,k_p}} \left\{
      {b_u}_{i+i_p}^k\: {A}_i^k \: \left(_i^k \mathbb{R}_{i_p}^{k_p}\right)^2
    \right\},
  \]
  (where $b_w= e_{1w}\,e_{2w}\,e_{3w}$ is the volume of $w$-cells) can be quite large.

\item[$\bullet$ pure iso-neutral operator]
  The iso-neutral flux of locally referenced potential density is zero.
  See \autoref{eq:latflux-rho} and \autoref{eq:vertflux-triad2}.

\item[$\bullet$ conservation of tracer]
  The iso-neutral diffusion conserves tracer content, \ie
  \[
    % \label{eq:iso_property1}
    \sum_{i,j,k} \left\{ D_l^T \      b_T \right\} = 0
  \]
  This property is trivially satisfied since the iso-neutral diffusive operator is written in flux form.

\item[$\bullet$ no increase of tracer variance]
  The iso-neutral diffusion does not increase the tracer variance, \ie
  \[
    % \label{eq:iso_property2}
    \sum_{i,j,k} \left\{ T \ D_l^T      \ b_T \right\} \leq 0
  \]
  The property is demonstrated in \autoref{subsec:variance} above.
  It is a key property for a diffusion term.
  It means that it is also a dissipation term,
  \ie it dissipates the square of the quantity on which it is applied.
  It therefore ensures that, when the diffusivity coefficient is large enough,
  the field on which it is applied becomes free of grid-point noise.

\item[$\bullet$ self-adjoint operator]
  The iso-neutral diffusion operator is self-adjoint, \ie
  \begin{equation}
    \label{eq:iso_property3}
    \sum_{i,j,k} \left\{ S \ D_l^T \ b_T \right\} = \sum_{i,j,k} \left\{ D_l^S \ T \ b_T \right\}
  \end{equation}
  In other word, there is no need to develop a specific routine from the adjoint of this operator.
  We just have to apply the same routine.
  This property can be demonstrated similarly to the proof of the `no increase of tracer variance' property.
  The contribution by a single triad towards the left hand side of \autoref{eq:iso_property3},
  can be found by replacing $\delta[T]$ by $\delta[S]$ in \autoref{eq:dvar_iso_i} and \autoref{eq:dvar_iso_k}.
  This results in a term similar to \autoref{eq:perfect-square},
  \[
    % \label{eq:TScovar}
    - {A}_i^k{\:} _i^k\mathbb{V}_{i_p}^{k_p}
    \left(
      \frac{ \delta_{i+ i_p}[T^k] }{ {e_{1u}}_{\,i + i_p}^{\,k} }
      -{\:}_i^k\mathbb{R}_{i_p}^{k_p}
      \frac{ \delta_{k+k_p} [T^i] }{{e_{3w}}_{\,i}^{\,k+k_p} }
    \right)
    \left(
      \frac{ \delta_{i+ i_p}[S^k] }{ {e_{1u}}_{\,i + i_p}^{\,k} }
      -{\:}_i^k\mathbb{R}_{i_p}^{k_p}
      \frac{ \delta_{k+k_p} [S^i] }{{e_{3w}}_{\,i}^{\,k+k_p} }
    \right).
  \]
This is symmetrical in $T $ and $S$, so exactly the same term arises from
the discretization of this triad's contribution towards the RHS of \autoref{eq:iso_property3}.
\end{description}

\subsection{Treatment of the triads at the boundaries}
\label{sec:iso_bdry}

The triad slope can only be defined where both the grid boxes centred at the end of the arms exist.
Triads that would poke up through the upper ocean surface into the atmosphere,
or down into the ocean floor, must be masked out.
See \autoref{fig:bdry_triads}.
Surface layer triads \triad{i}{1}{R}{1/2}{-1/2} (magenta) and \triad{i+1}{1}{R}{-1/2}{-1/2} (blue) that
require density to be specified above the ocean surface are masked (\autoref{fig:bdry_triads}a):
this ensures that lateral tracer gradients produce no flux through the ocean surface.
However, to prevent surface noise, it is customary to retain the $_{11}$ contributions towards
the lateral triad fluxes \triad[u]{i}{1}{F}{1/2}{-1/2} and \triad[u]{i+1}{1}{F}{-1/2}{-1/2};
this drives diapycnal tracer fluxes.
Similar comments apply to triads that would intersect the ocean floor (\autoref{fig:bdry_triads}b).
Note that both near bottom triad slopes \triad{i}{k}{R}{1/2}{1/2} and \triad{i+1}{k}{R}{-1/2}{1/2} are masked when
either of the $i,k+1$ or $i+1,k+1$ tracer points is masked, \ie the $i,k+1$ $u$-point is masked.
The associated lateral fluxes (grey-black dashed line) are masked if \np{ln\_botmix\_triad}\forcode{ = .false.},
but left unmasked, giving bottom mixing, if \np{ln\_botmix\_triad}\forcode{ = .true.}.

The default option \np{ln\_botmix\_triad}\forcode{ = .false.} is suitable when the bbl mixing option is enabled
(\key{trabbl}, with \np{nn\_bbl\_ldf}\forcode{ = 1}), or for simple idealized problems.
For setups with topography without bbl mixing, \np{ln\_botmix\_triad}\forcode{ = .true.} may be necessary.
% >>>>>>>>>>>>>>>>>>>>>>>>>>>>
\begin{figure}[h]
  \begin{center}
    \includegraphics[width=0.60\textwidth]{Fig_GRIFF_bdry_triads}
    \caption{
      \protect\label{fig:bdry_triads}
      (a) Uppermost model layer $k=1$ with $i,1$ and $i+1,1$ tracer points (black dots),
      and $i+1/2,1$ $u$-point (blue square).
      Triad slopes \triad{i}{1}{R}{1/2}{-1/2} (magenta) and \triad{i+1}{1}{R}{-1/2}{-1/2} (blue) poking through
      the ocean surface are masked (faded in figure).
      However, the lateral $_{11}$ contributions towards \triad[u]{i}{1}{F}{1/2}{-1/2} and
      \triad[u]{i+1}{1}{F}{-1/2}{-1/2} (yellow line) are still applied,
      giving diapycnal diffusive fluxes.
      \newline
      (b) Both near bottom triad slopes \triad{i}{k}{R}{1/2}{1/2} and
      \triad{i+1}{k}{R}{-1/2}{1/2} are masked when either of the $i,k+1$ or $i+1,k+1$ tracer points is masked,
      \ie the $i,k+1$ $u$-point is masked.
      The associated lateral fluxes (grey-black dashed line) are masked if
      \protect\np{botmix\_triad}\forcode{ = .false.}, but left unmasked,
      giving bottom mixing, if \protect\np{botmix\_triad}\forcode{ = .true.}
    }
  \end{center}
\end{figure}
% >>>>>>>>>>>>>>>>>>>>>>>>>>>>

\subsection{ Limiting of the slopes within the interior}
\label{sec:limit}

As discussed in \autoref{subsec:LDF_slp_iso},
iso-neutral slopes relative to geopotentials must be bounded everywhere,
both for consistency with the small-slope approximation and for numerical stability \citep{Cox1987, Griffies_Bk04}.
The bound chosen in \NEMO is applied to each component of the slope separately and
has a value of $1/100$ in the ocean interior.
%, ramping linearly down above 70~m depth to zero at the surface
It is of course relevant to the iso-neutral slopes $\tilde{r}_i=r_i+\sigma_i$ relative to geopotentials
(here the $\sigma_i$ are the slopes of the coordinate surfaces relative to geopotentials)
\autoref{eq:PE_slopes_eiv} rather than the slope $r_i$ relative to coordinate surfaces, so we require
\[
  |\tilde{r}_i|\leq \tilde{r}_\mathrm{max}=0.01.
\]
and then recalculate the slopes $r_i$ relative to coordinates.
Each individual triad slope
\begin{equation}
  \label{eq:Rtilde}
  _i^k\tilde{\mathbb{R}}_{i_p}^{k_p} = {}_i^k\mathbb{R}_{i_p}^{k_p}  + \frac{\delta_{i+i_p}[z_T^k]}{{e_{1u}}_{\,i + i_p}^{\,k}}
\end{equation}
is limited like this and then the corresponding $_i^k\mathbb{R}_{i_p}^{k_p} $ are recalculated and
combined to form the fluxes.
Note that where the slopes have been limited, there is now a non-zero iso-neutral density flux that
drives dianeutral mixing.
In particular this iso-neutral density flux is always downwards,
and so acts to reduce gravitational potential energy.

\subsection{Tapering within the surface mixed layer}
\label{sec:taper}

Additional tapering of the iso-neutral fluxes is necessary within the surface mixed layer.
When the Griffies triads are used, we offer two options for this.

\subsubsection{Linear slope tapering within the surface mixed layer}
\label{sec:lintaper}

This is the option activated by the default choice \np{ln\_triad\_iso}\forcode{ = .false.}.
Slopes $\tilde{r}_i$ relative to geopotentials are tapered linearly from their value immediately below
the mixed layer to zero at the surface, as described in option (c) of \autoref{fig:eiv_slp}, to values
\begin{equation}
  \label{eq:rmtilde}
  \rMLt = -\frac{z}{h}\left.\tilde{r}_i\right|_{z=-h}\quad \text{ for  } z>-h,
\end{equation}
and then the $r_i$ relative to vertical coordinate surfaces are appropriately adjusted to
\[
  % \label{eq:rm}
  \rML =\rMLt -\sigma_i \quad \text{ for  } z>-h.
\]
Thus the diffusion operator within the mixed layer is given by:
\[
  % \label{eq:iso_tensor_ML}
  D^{lT}=\nabla {\rm {\bf .}}\left( {A^{lT}\;\Re \;\nabla T} \right) \qquad
  \mbox{with}\quad \;\;\Re =\left( {{
        \begin{array}{*{20}c}
          1 \hfill & 0 \hfill & {-\rML[1]}\hfill \\
          0 \hfill & 1 \hfill & {-\rML[2]} \hfill \\
          {-\rML[1]}\hfill &   {-\rML[2]} \hfill & {\rML[1]^2+\rML[2]^2} \hfill
        \end{array}
      }} \right)
\]

This slope tapering gives a natural connection between tracer in the mixed-layer and
in isopycnal layers immediately below, in the thermocline.
It is consistent with the way the $\tilde{r}_i$ are tapered within the mixed layer
(see \autoref{sec:taperskew} below) so as to ensure a uniform GM eddy-induced velocity throughout the mixed layer.
However, it gives a downwards density flux and so acts so as to reduce potential energy in the same way as
does the slope limiting discussed above in \autoref{sec:limit}.
 
As in \autoref{sec:limit} above, the tapering \autoref{eq:rmtilde} is applied separately to
each triad $_i^k\tilde{\mathbb{R}}_{i_p}^{k_p}$, and the $_i^k\mathbb{R}_{i_p}^{k_p}$ adjusted.
For clarity, we assume $z$-coordinates in the following;
the conversion from $\mathbb{R}$ to $\tilde{\mathbb{R}}$ and back to $\mathbb{R}$ follows exactly as
described above by \autoref{eq:Rtilde}.
\begin{enumerate}
\item
  Mixed-layer depth is defined so as to avoid including regions of weak vertical stratification in
  the slope definition.
  At each $i,j$ (simplified to $i$ in \autoref{fig:MLB_triad}),
  we define the mixed-layer by setting the vertical index of the tracer point immediately below the mixed layer,
  $k_{\mathrm{ML}}$, as the maximum $k$ (shallowest tracer point) such that
  the potential density ${\rho_0}_{i,k}>{\rho_0}_{i,k_{10}}+\Delta\rho_c$,
  where $i,k_{10}$ is the tracer gridbox within which the depth reaches 10~m.
  See the left side of \autoref{fig:MLB_triad}.
  We use the $k_{10}$-gridbox instead of the surface gridbox to avoid problems \eg with thin daytime mixed-layers.
  Currently we use the same $\Delta\rho_c=0.01\;\mathrm{kg\:m^{-3}}$ for ML triad tapering as is used to
  output the diagnosed mixed-layer depth $h_{\mathrm{ML}}=|z_{W}|_{k_{\mathrm{ML}}+1/2}$,
  the depth of the $w$-point above the $i,k_{\mathrm{ML}}$ tracer point.
\item
  We define `basal' triad slopes ${\:}_i{\mathbb{R}_{\mathrm{base}}}_{\,i_p}^{k_p}$ as
  the slopes of those triads whose vertical `arms' go down from the $i,k_{\mathrm{ML}}$ tracer point to
  the $i,k_{\mathrm{ML}}-1$ tracer point below.
  This is to ensure that the vertical density gradients associated with
  these basal triad slopes ${\:}_i{\mathbb{R}_{\mathrm{base}}}_{\,i_p}^{k_p}$ are representative of the thermocline.
  The four basal triads defined in the bottom part of \autoref{fig:MLB_triad} are then
  \begin{align*}
    {\:}_i{\mathbb{R}_{\mathrm{base}}}_{\,i_p}^{k_p} &=
                                                       {\:}^{k_{\mathrm{ML}}-k_p-1/2}_i{\mathbb{R}_{\mathrm{base}}}_{\,i_p}^{k_p},
                                                       % \label{eq:Rbase}
    \\
    \intertext{with \eg the green triad}
    {\:}_i{\mathbb{R}_{\mathrm{base}}}_{1/2}^{-1/2}&=
                                                     {\:}^{k_{\mathrm{ML}}}_i{\mathbb{R}_{\mathrm{base}}}_{\,1/2}^{-1/2}.
  \end{align*}
The vertical flux associated with each of these triads passes through
the $w$-point $i,k_{\mathrm{ML}}-1/2$ lying \emph{below} the $i,k_{\mathrm{ML}}$ tracer point, so it is this depth
\[
  % \label{eq:zbase}
  {z_\mathrm{base}}_{\,i}={z_{w}}_{k_\mathrm{ML}-1/2}
\]
one gridbox deeper than the diagnosed ML depth $z_{\mathrm{ML}})$ that sets the $h$ used to taper the slopes in
\autoref{eq:rmtilde}.
\item
  Finally, we calculate the adjusted triads ${\:}_i^k{\mathbb{R}_{\mathrm{ML}}}_{\,i_p}^{k_p}$ within
  the mixed layer, by multiplying the appropriate ${\:}_i{\mathbb{R}_{\mathrm{base}}}_{\,i_p}^{k_p}$ by
  the ratio of the depth of the $w$-point ${z_w}_{k+k_p}$ to ${z_{\mathrm{base}}}_{\,i}$.
  For instance the green triad centred on $i,k$
  \begin{align*}
    {\:}_i^k{\mathbb{R}_{\mathrm{ML}}}_{\,1/2}^{-1/2} &=
                                                        \frac{{z_w}_{k-1/2}}{{z_{\mathrm{base}}}_{\,i}}{\:}_i{\mathbb{R}_{\mathrm{base}}}_{\,1/2}^{-1/2} \\
    \intertext{and more generally}
    {\:}_i^k{\mathbb{R}_{\mathrm{ML}}}_{\,i_p}^{k_p} &=
                                                       \frac{{z_w}_{k+k_p}}{{z_{\mathrm{base}}}_{\,i}}{\:}_i{\mathbb{R}_{\mathrm{base}}}_{\,i_p}^{k_p}.
                                                       % \label{eq:RML}
  \end{align*}
\end{enumerate}

% >>>>>>>>>>>>>>>>>>>>>>>>>>>>
\begin{figure}[h]
%  \fcapside {
  \caption{
    \protect\label{fig:MLB_triad}
    Definition of mixed-layer depth and calculation of linearly tapered triads.
    The figure shows a water column at a given $i,j$ (simplified to $i$), with the ocean surface at the top.
    Tracer points are denoted by bullets, and black lines the edges of the tracer cells;
    $k$ increases upwards.
    \newline
    \hspace{5 em}
    We define the mixed-layer by setting the vertical index of the tracer point immediately below the mixed layer,
    $k_{\mathrm{ML}}$, as the maximum $k$ (shallowest tracer point) such that
    ${\rho_0}_{i,k}>{\rho_0}_{i,k_{10}}+\Delta\rho_c$,
    where $i,k_{10}$ is the tracer gridbox within which the depth reaches 10~m.
    We calculate the triad slopes within the mixed layer by linearly tapering them from zero
    (at the surface) to the `basal' slopes,
    the slopes of the four triads passing through the $w$-point $i,k_{\mathrm{ML}}-1/2$ (blue square),
    ${\:}_i{\mathbb{R}_{\mathrm{base}}}_{\,i_p}^{k_p}$.
    Triads with different $i_p,k_p$, denoted by different colours,
    (\eg the green triad $i_p=1/2,k_p=-1/2$) are tapered to the appropriate basal triad.}
  % }
  \includegraphics[width=0.60\textwidth]{Fig_GRIFF_MLB_triads}
\end{figure}
% >>>>>>>>>>>>>>>>>>>>>>>>>>>>

\subsubsection{Additional truncation of skew iso-neutral flux components}
\label{subsec:Gerdes-taper}

The alternative option is activated by setting \np{ln\_triad\_iso} = true.
This retains the same tapered slope $\rML$  described above for the calculation of the $_{33}$ term of
the iso-neutral diffusion tensor (the vertical tracer flux driven by vertical tracer gradients),
but replaces the $\rML$ in the skew term by
\begin{equation}
  \label{eq:rm*}
  \rML^*=\left.\rMLt^2\right/\tilde{r}_i-\sigma_i,
\end{equation}
giving a ML diffusive operator
\[
  % \label{eq:iso_tensor_ML2}
  D^{lT}=\nabla {\rm {\bf .}}\left( {A^{lT}\;\Re \;\nabla T} \right) \qquad
  \mbox{with}\quad \;\;\Re =\left( {{
        \begin{array}{*{20}c}
          1 \hfill & 0 \hfill & {-\rML[1]^*}\hfill \\
          0 \hfill & 1 \hfill & {-\rML[2]^*} \hfill \\
          {-\rML[1]^*}\hfill &   {-\rML[2]^*} \hfill & {\rML[1]^2+\rML[2]^2} \hfill \\
        \end{array}
      }} \right).
\]
This operator
\footnote{
  To ensure good behaviour where horizontal density gradients are weak,
  we in fact follow \citet{Gerdes1991} and
  set $\rML^*=\mathrm{sgn}(\tilde{r}_i)\min(|\rMLt^2/\tilde{r}_i|,|\tilde{r}_i|)-\sigma_i$.
}
then has the property it gives no vertical density flux, and so does not change the potential energy.
This approach is similar to multiplying the iso-neutral diffusion coefficient by
$\tilde{r}_{\mathrm{max}}^{-2}\tilde{r}_i^{-2}$ for steep slopes,
as suggested by \citet{Gerdes1991} (see also \citet{Griffies_Bk04}).
Again it is applied separately to each triad $_i^k\mathbb{R}_{i_p}^{k_p}$

In practice, this approach gives weak vertical tracer fluxes through the mixed-layer,
as well as vanishing density fluxes.
While it is theoretically advantageous that it does not change the potential energy,
it may give a discontinuity between the fluxes within the mixed-layer (purely horizontal) and
just below (along iso-neutral surfaces).
% This may give strange looking results,
% particularly where the mixed-layer depth varies strongly laterally.
% ================================================================
% Skew flux formulation for Eddy Induced Velocity :
% ================================================================
\section{Eddy induced advection formulated as a skew flux}
\label{sec:skew-flux}

\subsection{Continuous skew flux formulation}
\label{sec:continuous-skew-flux}

When Gent and McWilliams's [1990] diffusion is used, an additional advection term is added.
The associated velocity is the so called eddy induced velocity,
the formulation of which depends on the slopes of iso-neutral surfaces.
Contrary to the case of iso-neutral mixing, the slopes used here are referenced to the geopotential surfaces,
\ie \autoref{eq:ldfslp_geo} is used in $z$-coordinate,
and the sum \autoref{eq:ldfslp_geo} + \autoref{eq:ldfslp_iso} in $z^*$ or $s$-coordinates.

The eddy induced velocity is given by:
\begin{subequations}
  % \label{eq:eiv}
  \begin{equation}
    \label{eq:eiv_v}
    \begin{split}
      u^* & = - \frac{1}{e_{3}}\;          \partial_i\psi_1,  \\
      v^* & = - \frac{1}{e_{3}}\;          \partial_j\psi_2,    \\
      w^* & =    \frac{1}{e_{1}e_{2}}\; \left\{ \partial_i  \left( e_{2} \, \psi_1\right)
        + \partial_j  \left( e_{1} \, \psi_2\right) \right\},
    \end{split}
  \end{equation}
  where the streamfunctions $\psi_i$ are given by
  \begin{equation}
    \label{eq:eiv_psi}
    \begin{split}
      \psi_1 & = A_{e} \; \tilde{r}_1,   \\
      \psi_2 & = A_{e} \; \tilde{r}_2,
    \end{split}
  \end{equation}
\end{subequations}
with $A_{e}$ the eddy induced velocity coefficient,
and $\tilde{r}_1$ and $\tilde{r}_2$ the slopes between the iso-neutral and the geopotential surfaces.

The traditional way to implement this additional advection is to add it to the Eulerian velocity prior to
computing the tracer advection.
This is implemented if \key{traldf\_eiv} is set in the default implementation,
where \np{ln\_traldf\_triad} is set false.
This allows us to take advantage of all the advection schemes offered for the tracers
(see \autoref{sec:TRA_adv}) and not just a $2^{nd}$ order advection scheme.
This is particularly useful for passive tracers where
\emph{positivity} of the advection scheme is of paramount importance.

However, when \np{ln\_traldf\_triad} is set true,
\NEMO instead implements eddy induced advection according to the so-called skew form \citep{Griffies_JPO98}.
It is based on a transformation of the advective fluxes using the non-divergent nature of the eddy induced velocity.
For example in the (\textbf{i},\textbf{k}) plane,
the tracer advective fluxes per unit area in $ijk$ space can be transformed as follows:
\begin{flalign*}
  \begin{split}
    \textbf{F}_{\mathrm{eiv}}^T =
    \begin{pmatrix}
      {e_{2}\,e_{3}\;  u^*} \\
 		{e_{1}\,e_{2}\; w^*}
    \end{pmatrix}   \;   T
    &=
    \begin{pmatrix}
      { - \partial_k \left( e_{2} \,\psi_1 \right) \; T \;} \\
 		{+ \partial_i  \left( e_{2} \, \psi_1 \right) \; T \;}
    \end{pmatrix} 			\\
    &=
    \begin{pmatrix}
      { - \partial_k \left( e_{2} \, \psi_1  \; T \right) \;} \\
 		{+ \partial_i  \left( e_{2} \,\psi_1 \; T \right) \;}
    \end{pmatrix}
    +
    \begin{pmatrix}
      {+ e_{2} \, \psi_1  \; \partial_k T} \\
 		{ - e_{2} \, \psi_1  \; \partial_i  T}
    \end{pmatrix}
  \end{split}
\end{flalign*}
and since the eddy induced velocity field is non-divergent,
we end up with the skew form of the eddy induced advective fluxes per unit area in $ijk$ space:
\begin{equation}
  \label{eq:eiv_skew_ijk}
  \textbf{F}_\mathrm{eiv}^T =
  \begin{pmatrix}
    {+ e_{2} \, \psi_1  \; \partial_k T}   \\
    { - e_{2} \, \psi_1  \; \partial_i  T}
  \end{pmatrix}
\end{equation}
The total fluxes per unit physical area are then
\begin{equation}
  \label{eq:eiv_skew_physical}
  \begin{split}
    f^*_1 & = \frac{1}{e_{3}}\; \psi_1 \partial_k T   \\
    f^*_2 & = \frac{1}{e_{3}}\; \psi_2 \partial_k T   \\
    f^*_3 & =  -\frac{1}{e_{1}e_{2}}\; \left\{ e_{2} \psi_1 \partial_i T + e_{1} \psi_2 \partial_j T \right\}.
\end{split}
\end{equation}
Note that \autoref{eq:eiv_skew_physical} takes the same form whatever the vertical coordinate,
though of course the slopes $\tilde{r}_i$ which define the $\psi_i$ in \autoref{eq:eiv_psi} are relative to
geopotentials.
The tendency associated with eddy induced velocity is then simply the convergence of the fluxes
(\autoref{eq:eiv_skew_ijk}, \autoref{eq:eiv_skew_physical}), so
\[
  % \label{eq:skew_eiv_conv}
  \frac{\partial T}{\partial t}= -\frac{1}{e_1 \, e_2 \, e_3 }      \left[
    \frac{\partial}{\partial i} \left( e_2 \psi_1 \partial_k T\right)
    + \frac{\partial}{\partial j} \left( e_1  \;
      \psi_2 \partial_k T\right)
    -  \frac{\partial}{\partial k} \left( e_{2} \psi_1 \partial_i T
      + e_{1} \psi_2 \partial_j T \right)  \right]
\]
It naturally conserves the tracer content, as it is expressed in flux form.
Since it has the same divergence as the advective form it also preserves the tracer variance.

\subsection{Discrete skew flux formulation}

The skew fluxes in (\autoref{eq:eiv_skew_physical}, \autoref{eq:eiv_skew_ijk}),
like the off-diagonal terms (\autoref{eq:i13c}, \autoref{eq:i31c}) of the small angle diffusion tensor,
are best expressed in terms of the triad slopes, as in \autoref{fig:ISO_triad} and
(\autoref{eq:i13}, \autoref{eq:i31});
but now in terms of the triad slopes $\tilde{\mathbb{R}}$ relative to geopotentials instead of
the $\mathbb{R}$ relative to coordinate surfaces.
The discrete form of \autoref{eq:eiv_skew_ijk} using the slopes \autoref{eq:R} and
defining $A_e$ at $T$-points is then given by:

\begin{subequations}
  % \label{eq:allskewflux}
  \begin{flalign*}
    % \label{eq:vect_skew_flux}
    \vect{F}_{\mathrm{eiv}}(T) &\equiv    \sum_{\substack{i_p,\,k_p}}
    \begin{pmatrix}
      {_{i+1/2-i_p}^k {\mathbb{S}_u}_{i_p}^{k_p} } (T)      \\      \\
      {_i^{k+1/2-k_p} {\mathbb{S}_w}_{i_p}^{k_p} } (T)      \\
    \end{pmatrix},
  \end{flalign*}
  where the skew flux in the $i$-direction associated with a given triad is (\autoref{eq:latflux-triad},
  \autoref{eq:triadfluxu}):
  \begin{align}
    \label{eq:skewfluxu}
    _i^k {\mathbb{S}_u}_{i_p}^{k_p} (T) &= + \fractext{1}{4} {A_e}_i^k{
                                          \:}\frac{{b_u}_{i+i_p}^k}{{e_{1u}}_{\,i + i_p}^{\,k}}
                                          \ {_i^k\tilde{\mathbb{R}}_{i_p}^{k_p}} \
                                          \frac{ \delta_{k+k_p} [T^i] }{{e_{3w}}_{\,i}^{\,k+k_p} }, \\
    \intertext{
    and \autoref{eq:triadfluxw} in the $k$-direction, changing the sign
    to be consistent with \autoref{eq:eiv_skew_ijk}:
    }
    _i^k {\mathbb{S}_w}_{i_p}^{k_p} (T)
                                        &= -\fractext{1}{4} {A_e}_i^k{\: }\frac{{b_u}_{i+i_p}^k}{{e_{3w}}_{\,i}^{\,k+k_p}}
                                          {_i^k\tilde{\mathbb{R}}_{i_p}^{k_p}}\frac{ \delta_{i+ i_p}[T^k] }{ {e_{1u}}_{\,i + i_p}^{\,k} }.\label{eq:skewfluxw}
  \end{align}
\end{subequations}

Such a discretisation is consistent with the iso-neutral operator as it uses the same definition for the slopes.
It also ensures the following two key properties.

\subsubsection{No change in tracer variance}

The discretization conserves tracer variance, \ie it does not include a diffusive component but is a `pure' advection term.
This can be seen %either from Appendix \autoref{apdx:eiv_skew} or
by considering the fluxes associated with a given triad slope $_i^k{\mathbb{R}}_{i_p}^{k_p} (T)$.
For, following \autoref{subsec:variance} and \autoref{eq:dvar_iso_i},
the associated horizontal skew-flux $_i^k{\mathbb{S}_u}_{i_p}^{k_p} (T)$ drives a net rate of change of variance,
summed over the two $T$-points $i+i_p-\fractext{1}{2},k$ and $i+i_p+\fractext{1}{2},k$, of
\begin{equation}
  \label{eq:dvar_eiv_i}
  _i^k{\mathbb{S}_u}_{i_p}^{k_p} (T)\,\delta_{i+ i_p}[T^k],
\end{equation}
while the associated vertical skew-flux gives a variance change summed over
the $T$-points $i,k+k_p-\fractext{1}{2}$ (above) and $i,k+k_p+\fractext{1}{2}$ (below) of
\begin{equation}
  \label{eq:dvar_eiv_k}
  _i^k{\mathbb{S}_w}_{i_p}^{k_p} (T) \,\delta_{k+ k_p}[T^i].
\end{equation}
Inspection of the definitions (\autoref{eq:skewfluxu}, \autoref{eq:skewfluxw}) shows that
these two variance changes (\autoref{eq:dvar_eiv_i}, \autoref{eq:dvar_eiv_k}) sum to zero.
Hence the two fluxes associated with each triad make no net contribution to the variance budget.

\subsubsection{Reduction in gravitational PE}

The vertical density flux associated with the vertical skew-flux always has the same sign as
the vertical density gradient;
thus, so long as the fluid is stable (the vertical density gradient is negative)
the vertical density flux is negative (downward) and hence reduces the gravitational PE.

For the change in gravitational PE driven by the $k$-flux is
\begin{align}
  \label{eq:vert_densityPE}
  g {e_{3w}}_{\,i}^{\,k+k_p}{\mathbb{S}_w}_{i_p}^{k_p} (\rho)
  &=g {e_{3w}}_{\,i}^{\,k+k_p}\left[-\alpha _i^k {\:}_i^k
    {\mathbb{S}_w}_{i_p}^{k_p} (T) + \beta_i^k {\:}_i^k
    {\mathbb{S}_w}_{i_p}^{k_p} (S) \right]. \notag \\
  \intertext{Substituting  ${\:}_i^k {\mathbb{S}_w}_{i_p}^{k_p}$ from \autoref{eq:skewfluxw}, gives}
  % and separating out
  % $\rtriadt{R}=\rtriad{R} + \delta_{i+i_p}[z_T^k]$,
  % gives two terms. The
  % first $\rtriad{R}$ term (the only term for $z$-coordinates) is:
  &=-\fractext{1}{4} g{A_e}_i^k{\: }{b_u}_{i+i_p}^k {_i^k\tilde{\mathbb{R}}_{i_p}^{k_p}}
    \frac{ -\alpha _i^k\delta_{i+ i_p}[T^k]+ \beta_i^k\delta_{i+ i_p}[S^k]} { {e_{1u}}_{\,i + i_p}^{\,k} } \notag \\
  &=+\fractext{1}{4} g{A_e}_i^k{\: }{b_u}_{i+i_p}^k
    \left({_i^k\mathbb{R}_{i_p}^{k_p}}+\frac{\delta_{i+i_p}[z_T^k]}{{e_{1u}}_{\,i + i_p}^{\,k}}\right) {_i^k\mathbb{R}_{i_p}^{k_p}}
    \frac{-\alpha_i^k \delta_{k+ k_p}[T^i]+ \beta_i^k\delta_{k+ k_p}[S^i]} {{e_{3w}}_{\,i}^{\,k+k_p}},
\end{align}
using the definition of the triad slope $\rtriad{R}$, \autoref{eq:R} to
express $-\alpha _i^k\delta_{i+ i_p}[T^k]+\beta_i^k\delta_{i+ i_p}[S^k]$ in terms of
$-\alpha_i^k \delta_{k+ k_p}[T^i]+ \beta_i^k\delta_{k+ k_p}[S^i]$.

Where the coordinates slope, the $i$-flux gives a PE change
\begin{multline}
  \label{eq:lat_densityPE}
  g \delta_{i+i_p}[z_T^k]
  \left[
    -\alpha _i^k {\:}_i^k {\mathbb{S}_u}_{i_p}^{k_p} (T) + \beta_i^k {\:}_i^k {\mathbb{S}_u}_{i_p}^{k_p} (S)
  \right] \\
  = +\fractext{1}{4} g{A_e}_i^k{\: }{b_u}_{i+i_p}^k
  \frac{\delta_{i+i_p}[z_T^k]}{{e_{1u}}_{\,i + i_p}^{\,k}}
  \left({_i^k\mathbb{R}_{i_p}^{k_p}}+\frac{\delta_{i+i_p}[z_T^k]}{{e_{1u}}_{\,i + i_p}^{\,k}}\right)
  \frac{-\alpha_i^k \delta_{k+ k_p}[T^i]+ \beta_i^k\delta_{k+ k_p}[S^i]} {{e_{3w}}_{\,i}^{\,k+k_p}},
\end{multline}
(using \autoref{eq:skewfluxu}) and so the total PE change \autoref{eq:vert_densityPE} +
\autoref{eq:lat_densityPE} associated with the triad fluxes is
\begin{multline*}
  % \label{eq:tot_densityPE}
  g{e_{3w}}_{\,i}^{\,k+k_p}{\mathbb{S}_w}_{i_p}^{k_p} (\rho) +
  g\delta_{i+i_p}[z_T^k] {\:}_i^k {\mathbb{S}_u}_{i_p}^{k_p} (\rho) \\
  = +\fractext{1}{4} g{A_e}_i^k{\: }{b_u}_{i+i_p}^k
  \left({_i^k\mathbb{R}_{i_p}^{k_p}}+\frac{\delta_{i+i_p}[z_T^k]}{{e_{1u}}_{\,i + i_p}^{\,k}}\right)^2
  \frac{-\alpha_i^k \delta_{k+ k_p}[T^i]+ \beta_i^k\delta_{k+ k_p}[S^i]} {{e_{3w}}_{\,i}^{\,k+k_p}}.
\end{multline*}
Where the fluid is stable, with $-\alpha_i^k \delta_{k+ k_p}[T^i]+
\beta_i^k\delta_{k+ k_p}[S^i]<0$, this PE change is negative.

\subsection{Treatment of the triads at the boundaries}
\label{sec:skew_bdry}

Triad slopes \rtriadt{R} used for the calculation of the eddy-induced skew-fluxes are masked at the boundaries 
in exactly the same way as are the triad slopes \rtriad{R} used for the iso-neutral diffusive fluxes, 
as described in \autoref{sec:iso_bdry} and \autoref{fig:bdry_triads}. 
Thus surface layer triads $\triadt{i}{1}{R}{1/2}{-1/2}$ and $\triadt{i+1}{1}{R}{-1/2}{-1/2}$ are masked, 
and both near bottom triad slopes $\triadt{i}{k}{R}{1/2}{1/2}$ and $\triadt{i+1}{k}{R}{-1/2}{1/2}$ are masked when 
either of the $i,k+1$ or $i+1,k+1$ tracer points is masked, \ie the $i,k+1$ $u$-point is masked. 
The namelist parameter \np{ln\_botmix\_triad} has no effect on the eddy-induced skew-fluxes.

\subsection{Limiting of the slopes within the interior}
\label{sec:limitskew}

Presently, the iso-neutral slopes $\tilde{r}_i$ relative to geopotentials are limited to be less than $1/100$, 
exactly as in calculating the iso-neutral diffusion, \S \autoref{sec:limit}. 
Each individual triad \rtriadt{R} is so limited.

\subsection{Tapering within the surface mixed layer}
\label{sec:taperskew}

The slopes $\tilde{r}_i$ relative to geopotentials (and thus the individual triads \rtriadt{R}) 
are always tapered linearly from their value immediately below the mixed layer to zero at the surface 
\autoref{eq:rmtilde}, as described in \autoref{sec:lintaper}. 
This is option (c) of \autoref{fig:eiv_slp}. 
This linear tapering for the slopes used to calculate the eddy-induced fluxes is unaffected by 
the value of \np{ln\_triad\_iso}.

The justification for this linear slope tapering is that, for $A_e$ that is constant or varies only in
the horizontal (the most commonly used options in \NEMO: see \autoref{sec:LDF_coef}),
it is equivalent to a horizontal eiv (eddy-induced velocity) that is uniform within the mixed layer
\autoref{eq:eiv_v}.
This ensures that the eiv velocities do not restratify the mixed layer \citep{Treguier1997,Danabasoglu_al_2008}.
Equivantly, in terms of the skew-flux formulation we use here,
the linear slope tapering within the mixed-layer gives a linearly varying vertical flux,
and so a tracer convergence uniform in depth
(the horizontal flux convergence is relatively insignificant within the mixed-layer).

\subsection{Streamfunction diagnostics}
\label{sec:sfdiag}

Where the namelist parameter \np{ln\_traldf\_gdia}\forcode{ = .true.},
diagnosed mean eddy-induced velocities are output.
Each time step, streamfunctions are calculated in the $i$-$k$ and $j$-$k$ planes at
$uw$ (integer +1/2 $i$, integer $j$, integer +1/2 $k$) and $vw$ (integer $i$, integer +1/2 $j$, integer +1/2 $k$)
points (see Table \autoref{tab:cell}) respectively.
We follow \citep{Griffies_Bk04} and calculate the streamfunction at a given $uw$-point from
the surrounding four triads according to:
\[
  % \label{eq:sfdiagi}
  {\psi_1}_{i+1/2}^{k+1/2}={\fractext{1}{4}}\sum_{\substack{i_p,\,k_p}}
  {A_e}_{i+1/2-i_p}^{k+1/2-k_p}\:\triadd{i+1/2-i_p}{k+1/2-k_p}{R}{i_p}{k_p}.
\]
The streamfunction $\psi_1$ is calculated similarly at $vw$ points.
The eddy-induced velocities are then calculated from the straightforward discretisation of \autoref{eq:eiv_v}:
\[
  % \label{eq:eiv_v_discrete}
  \begin{split}
    {u^*}_{i+1/2}^{k} & = - \frac{1}{{e_{3u}}_{i}^{k}}\left({\psi_1}_{i+1/2}^{k+1/2}-{\psi_1}_{i+1/2}^{k+1/2}\right),   \\
    {v^*}_{j+1/2}^{k} & = - \frac{1}{{e_{3v}}_{j}^{k}}\left({\psi_2}_{j+1/2}^{k+1/2}-{\psi_2}_{j+1/2}^{k+1/2}\right),   \\
    {w^*}_{i,j}^{k+1/2} & =    \frac{1}{e_{1t}e_{2t}}\; \left\{
      {e_{2u}}_{i+1/2}^{k+1/2} \,{\psi_1}_{i+1/2}^{k+1/2} -
      {e_{2u}}_{i-1/2}^{k+1/2} \,{\psi_1}_{i-1/2}^{k+1/2} \right. + \\
    \phantom{=} & \qquad\qquad\left. {e_{2v}}_{j+1/2}^{k+1/2} \,{\psi_2}_{j+1/2}^{k+1/2} - {e_{2v}}_{j-1/2}^{k+1/2} \,{\psi_2}_{j-1/2}^{k+1/2} \right\},
  \end{split}
\]

\biblio

\pindex

\end{document}
