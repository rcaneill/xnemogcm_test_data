\documentclass[../main/NEMO_manual]{subfiles}

\begin{document}

\chapter{Diurnal SST Models (DIU)}
\label{chap:DIU}

\chaptertoc

\paragraph{Changes record} ~\\

{\footnotesize
  \begin{tabularx}{\textwidth}{l||X|X}
    Release & Author(s) & Modifications \\
    \hline
    {\em   4.0} & {\em ...} & {\em ...} \\
    {\em   3.6} & {\em ...} & {\em ...} \\
    {\em   3.4} & {\em ...} & {\em ...} \\
    {\em <=3.4} & {\em ...} & {\em ...}
  \end{tabularx}
}

\clearpage

Code to produce an estimate of the diurnal warming and cooling of the sea surface skin
temperature (skin SST) is found in the DIU directory.
The skin temperature can be split into three parts:
\begin{itemize}
\item A foundation SST which is free from diurnal warming.
\item A warm layer, typically ~3\,m thick,
  where heating from solar radiation can cause a warm stably stratified layer during the daytime
\item A cool skin, a thin layer, approximately ~1\, mm thick,
  where long wave cooling is dominant and cools the immediate ocean surface.
\end{itemize}

Models are provided for both the warm layer, \mdl{diurnal\_bulk}, and the cool skin, \mdl{cool\_skin}.
Foundation SST is not considered as it can be obtained either from the main \NEMO\ model
(\ie\ from the temperature of the top few model levels) or from some other source.
It must be noted that both the cool skin and warm layer models produce estimates of the change in temperature
($\Delta T_{\mathrm{cs}}$ and $\Delta T_{\mathrm{wl}}$) and
both must be added to a foundation SST to obtain the true skin temperature.

Both the cool skin and warm layer models are controlled through the namelist \nam{diu}{diu}:

\begin{listing}
  \nlst{namdiu}
  \caption{\forcode{&namdiu}}
  \label{lst:namdiu}
\end{listing}

This namelist contains only two variables:

\begin{description}
\item [{\np{ln_diurnal}{ln\_diurnal}}] A logical switch for turning on/off both the cool skin and warm layer.
\item [{\np{ln_diurnal_only}{ln\_diurnal\_only}}] A logical switch which if \forcode{.true.} will run the diurnal model without the other dynamical parts of \NEMO.
  \np{ln_diurnal_only}{ln\_diurnal\_only} must be \forcode{.false.} if \np{ln_diurnal}{ln\_diurnal} is \forcode{.false.}.
\end{description}

Output for the diurnal model is through the variables `sst\_wl' (warm\_layer) and `sst\_cs' (cool skin).
These are 2-D variables which will be included in the model output if they are specified in the iodef.xml file.

Initialisation is through the restart file.
Specifically the code will expect the presence of the 2-D variable ``Dsst'' to initialise the warm layer.
The cool skin model, which is determined purely by the instantaneous fluxes, has no initialisation variable.

%% =================================================================================================
\section{Warm layer model}
\label{sec:DIU_warm_layer_sec}

The warm layer is calculated using the model of \citet{takaya.bidlot.ea_JGR10} (TAKAYA10 model hereafter).
This is a simple flux based model that is defined by the equations
\begin{align}
\frac{\partial{\Delta T_{\mathrm{wl}}}}{\partial{t}}&=&\frac{Q(\nu+1)}{D_T\rho_w c_p
\nu}-\frac{(\nu+1)ku^*_{w}f(L_a)\Delta T}{D_T\Phi\!\left(\frac{D_T}{L}\right)} \mbox{,}
\label{eq:DIU_ecmwf1} \\
L&=&\frac{\rho_w c_p u^{*^3}_{w}}{\kappa g \alpha_w Q }\mbox{,}\label{eq:DIU_ecmwf2}
\end{align}
where $\Delta T_{\mathrm{wl}}$ is the temperature difference between the top of the warm layer and the depth $D_T=3$\,m at which there is assumed to be no diurnal signal.
In equation (\autoref{eq:DIU_ecmwf1}) $\alpha_w=2\times10^{-4}$ is the thermal expansion coefficient of water,
$\kappa=0.4$ is von K\'{a}rm\'{a}n's constant, $c_p$ is the heat capacity at constant pressure of sea water,
$\rho_w$ is the water density, and $L$ is the Monin-Obukhov length.
The tunable variable $\nu$ is a shape parameter that defines the expected subskin temperature profile via
$T(z) = T(0) - \left( \frac{z}{D_T} \right)^\nu \Delta T_{\mathrm{wl}}$,
where $T$ is the absolute temperature and $z\le D_T$ is the depth below the top of the warm layer.
The influence of wind on TAKAYA10 comes through the magnitude of the friction velocity of the water $u^*_{w}$,
which can be related to the 10\,m wind speed $u_{10}$ through
the relationship $u^*_{w} = u_{10}\sqrt{\frac{C_d\rho_a}{\rho_w}}$, where $C_d$ is the drag coefficient,
and $\rho_a$ is the density of air.
The symbol $Q$ in equation (\autoref{eq:DIU_ecmwf1}) is the instantaneous total thermal energy flux into
the diurnal layer, \ie
\[
  Q = Q_{\mathrm{sol}} + Q_{\mathrm{lw}} + Q_{\mathrm{h}}\mbox{,}
  % \label{eq:DIU_e_flux_eqn}
\]
where $Q_{\mathrm{h}}$ is the sensible and latent heat flux, $Q_{\mathrm{lw}}$ is the long wave flux,
and $Q_{\mathrm{sol}}$ is the solar flux absorbed within the diurnal warm layer.
For $Q_{\mathrm{sol}}$ the 9 term representation of \citet{gentemann.minnett.ea_JGR09} is used.
In equation \autoref{eq:DIU_ecmwf1} the function $f(L_a)=\max(1,L_a^{\frac{2}{3}})$,
where $L_a=0.3$\footnote{
  This is a global average value, more accurately $L_a$ could be computed as $L_a=(u^*_{w}/u_s)^{\frac{1}{2}}$,
  where $u_s$ is the stokes drift, but this is not currently done
} is the turbulent Langmuir number and is a parametrization of the effect of waves.
The function $\Phi\!\left(\frac{D_T}{L}\right)$ is the similarity function that
parametrizes the stability of the water column and is given by:
\begin{equation}
\Phi(\zeta) = \left\{ \begin{array}{cc} 1 + \frac{5\zeta +
4\zeta^2}{1+3\zeta+0.25\zeta^2} &(\zeta \ge 0) \\
                                    (1 - 16\zeta)^{-\frac{1}{2}} & (\zeta < 0) \mbox{,}
                                    \end{array} \right. \label{eq:DIU_stab_func_eqn}
\end{equation}
where $\zeta=\frac{D_T}{L}$.  It is clear that the first derivative of (\autoref{eq:DIU_stab_func_eqn}),
and thus of (\autoref{eq:DIU_ecmwf1}), is discontinuous at $\zeta=0$ (\ie\ $Q\rightarrow0$ in
equation (\autoref{eq:DIU_ecmwf2})).

The two terms on the right hand side of (\autoref{eq:DIU_ecmwf1}) represent different processes.
The first term is simply the diabatic heating or cooling of the diurnal warm layer due to
thermal energy fluxes into and out of the layer.
The second term parametrizes turbulent fluxes of heat out of the diurnal warm layer due to wind induced mixing.
In practice the second term acts as a relaxation on the temperature.

%% =================================================================================================
\section{Cool skin model}
\label{sec:DIU_cool_skin_sec}

The cool skin is modelled using the framework of \citet{saunders_JAS67} who used a formulation of the near surface temperature difference based upon the heat flux and the friction velocity $u^*_{w}$.
As the cool skin is so thin (~1\,mm) we ignore the solar flux component to the heat flux and the Saunders equation for the cool skin temperature difference $\Delta T_{\mathrm{cs}}$ becomes
\[
  % \label{eq:DIU_sunders_eqn}
  \Delta T_{\mathrm{cs}}=\frac{Q_{\mathrm{ns}}\delta}{k_t} \mbox{,}
\]
where $Q_{\mathrm{ns}}$ is the, usually negative, non-solar heat flux into the ocean and
$k_t$ is the thermal conductivity of sea water.
$\delta$ is the thickness of the skin layer and is given by
\begin{equation}
\label{eq:DIU_sunders_thick_eqn}
\delta=\frac{\lambda \mu}{u^*_{w}} \mbox{,}
\end{equation}
where $\mu$ is the kinematic viscosity of sea water and $\lambda$ is a constant of proportionality which
\citet{saunders_JAS67} suggested varied between 5 and 10.

The value of $\lambda$ used in equation (\autoref{eq:DIU_sunders_thick_eqn}) is that of \citet{artale.iudicone.ea_JGR02},
which is shown in \citet{tu.tsuang_GRL05} to outperform a number of other parametrisations at
both low and high wind speeds.
Specifically,
\[
  % \label{eq:DIU_artale_lambda_eqn}
  \lambda = \frac{ 8.64\times10^4 u^*_{w} k_t }{ \rho c_p h \mu \gamma }\mbox{,}
\]
where $h=10$\,m is a reference depth and
$\gamma$ is a dimensionless function of wind speed $u$:
\[
  % \label{eq:DIU_artale_gamma_eqn}
  \gamma =
  \begin{cases}
    0.2u+0.5\mbox{,} & u \le 7.5\,\mbox{ms}^{-1} \\
    1.6u-10\mbox{,} & 7.5 < u < 10\,\mbox{ms}^{-1} \\
    6\mbox{,} & u \ge 10\,\mbox{ms}^{-1} \\
  \end{cases}
\]

\subinc{%% =================================================================================================
%% Backmatter
%% =================================================================================================

%% Bibliography
%% =================================================================================================

\phantomsection
\addcontentsline{toc}{chapter}{Bibliography}
\lohead{Bibliography}
\rehead{Bibliography}
\bibliography{../main/bibliography}

\clearpage

%% Indices
%% =================================================================================================

\phantomsection
\addcontentsline{toc}{chapter}{Indices}
\lohead{Indices}
\rehead{Indices}
\printindex[blocks]
\printindex[keys]
\printindex[modules]
\printindex[parameters]
\printindex[subroutines]

\clearpage

%% Glossary
%% =================================================================================================

%\phantomsection
%\addcontentsline{toc}{chapter}{Glossary}
%\lohead{Glossary}\rehead{Glossary}
%\printglossaries
}

\end{document}
