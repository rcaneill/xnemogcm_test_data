\documentclass[../main/NEMO_manual]{subfiles}

\begin{document}

\chapter{Apply Assimilation Increments (ASM)}
\label{chap:ASM}

%    {\em 4.0} & {\em D. J. Lea} & {\em \NEMO\ 4.0 updates}  \\
%    {\em 3.4} & {\em D. J. Lea, M. Martin, K. Mogensen, A. Weaver} & {\em Initial version}  \\

\chaptertoc

\paragraph{Changes record} ~\\

{\footnotesize
  \begin{tabularx}{\textwidth}{l||X|X}
    Release & Author(s) & Modifications \\
    \hline
    {\em   4.0} & {\em ...} & {\em ...} \\
    {\em   3.6} & {\em ...} & {\em ...} \\
    {\em   3.4} & {\em ...} & {\em ...} \\
    {\em <=3.4} & {\em ...} & {\em ...}
  \end{tabularx}
}

\clearpage

The ASM code adds the functionality to apply increments to the model variables: temperature, salinity,
sea surface height, velocity and sea ice concentration.
These are read into the model from a NetCDF file which may be produced by separate data assimilation code.
The code can also output model background fields which are used as an input to data assimilation code.
This is all controlled by the namelist \nam{_asminc}{\_asminc}.
There is a brief description of all the namelist options provided.
To build the ASM code \key{asminc} must be set.

%% =================================================================================================
\section{Direct initialization}
\label{sec:ASM_DI}

Direct initialization (DI) refers to the instantaneous correction of the model background state using
the analysis increment.
DI is used when \np{ln_asmdin}{ln\_asmdin} is set to true.

%% =================================================================================================
\section{Incremental analysis updates}
\label{sec:ASM_IAU}

Rather than updating the model state directly with the analysis increment,
it may be preferable to introduce the increment gradually into the ocean model in order to
minimize spurious adjustment processes.
This technique is referred to as Incremental Analysis Updates (IAU) \citep{bloom.takacs.ea_MWR96}.
IAU is a common technique used with 3D assimilation methods such as 3D-Var or OI.
IAU is used when \np{ln_asmiau}{ln\_asmiau} is set to true.

With IAU, the model state trajectory ${\mathbf x}$ in the assimilation window ($t_{0} \leq t_{i} \leq t_{N}$)
is corrected by adding the analysis increments for temperature, salinity, horizontal velocity and SSH as
additional tendency terms to the prognostic equations:
\begin{align*}
  % \label{eq:ASM_wa_traj_iau}
  {\mathbf x}^{a}(t_{i}) = M(t_{i}, t_{0})[{\mathbf x}^{b}(t_{0})] \; + \; F_{i} \delta \tilde{\mathbf x}^{a}
\end{align*}
where $F_{i}$ is a weighting function for applying the increments $\delta\tilde{\mathbf x}^{a}$ defined such that
$\sum_{i=1}^{N} F_{i}=1$.
${\mathbf x}^b$ denotes the model initial state and ${\mathbf x}^a$ is the model state after the increments are applied.
To control the adjustment time of the model to the increment,
the increment can be applied over an arbitrary sub-window, $t_{m} \leq t_{i} \leq t_{n}$,
of the main assimilation window, where $t_{0} \leq t_{m} \leq t_{i}$ and $t_{i} \leq t_{n} \leq t_{N}$.
Typically the increments are spread evenly over the full window.
In addition, two different weighting functions have been implemented.
The first function (namelist option \np{niaufn}{niaufn}=0) employs constant weights,
\begin{align}
  \label{eq:ASM_F1_i}
  F^{(1)}_{i}
  =\left\{
  \begin{array}{ll}
    0     &    {\mathrm if} \; \; \; t_{i} < t_{m}                \\
    1/M &    {\mathrm if} \; \; \; t_{m} < t_{i} \leq t_{n} \\
    0     &    {\mathrm if} \; \; \; t_{i} > t_{n}
  \end{array}
            \right.
\end{align}
where $M = m-n$.
The second function (namelist option \np{niaufn}{niaufn}=1) employs peaked hat-like weights in order to give maximum weight in the centre of the sub-window,
with the weighting reduced linearly to a small value at the window end-points:
\begin{align}
  \label{eq:ASM_F2_i}
  F^{(2)}_{i}
  =\left\{
  \begin{array}{ll}
    0                           &    {\mathrm if} \; \; \; t_{i}       <     t_{m}                        \\
    \alpha \, i               &    {\mathrm if} \; \; \; t_{m}    \leq t_{i}    \leq   t_{M/2}   \\
    \alpha \, (M - i +1) &    {\mathrm if} \; \; \; t_{M/2}  <    t_{i}    \leq   t_{n}       \\
    0                            &   {\mathrm if} \; \; \; t_{i}        >    t_{n}
  \end{array}
                                   \right.
\end{align}
where $\alpha^{-1} = \sum_{i=1}^{M/2} 2i$ and $M$ is assumed to be even.
The weights described by \autoref{eq:ASM_F2_i} provide a smoother transition of the analysis trajectory from
one assimilation cycle to the next than that described by \autoref{eq:ASM_F1_i}.

%% =================================================================================================
\section{Divergence damping initialisation}
\label{sec:ASM_div_dmp}

It is quite challenging for data assimilation systems to provide non-divergent velocity increments.
Applying divergent velocity increments will likely cause spurious vertical velocities in the model. This section describes a method to take velocity increments provided to \NEMO\ ($u^0_I$ and $v^0_I$) and adjust them by the iterative application of a divergence damping operator. The method is also described in \citet{dobricic.pinardi.ea_OS07}.

In iteration step $n$ (starting at $n=1$) new estimates of velocity increments $u^{n}_I$ and $v^{n}_I$ are updated by:

\begin{equation}
  \label{eq:ASM_dmp}
  \left\{
    \begin{aligned}
      u^{n}_I = u^{n-1}_I + \frac{1}{e_{1u} } \delta_{i+1/2} \left( {A_D
          \;\chi^{n-1}_I } \right) \\ \\
      v^{n}_I = v^{n-1}_I + \frac{1}{e_{2v} } \delta_{j+1/2} \left( {A_D
          \;\chi^{n-1}_I } \right) \\
    \end{aligned}
  \right.,
\end{equation}

where the divergence is defined as

\[
  % \label{eq:ASM_div}
  \chi^{n-1}_I = \frac{1}{e_{1t}\,e_{2t}\,e_{3t} }
  \left( {\delta_i \left[ {e_{2u}\,e_{3u}\,u^{n-1}_I} \right]
      +\delta_j \left[ {e_{1v}\,e_{3v}\,v^{n-1}_I} \right]} \right).
\]

By the application of \autoref{eq:ASM_dmp} the divergence is filtered in each iteration,
and the vorticity is left unchanged.
In the presence of coastal boundaries with zero velocity increments perpendicular to the coast
the divergence is strongly damped.
This type of the initialisation reduces the vertical velocity magnitude and
alleviates the problem of the excessive unphysical vertical mixing in the first steps of the model integration
\citep{talagrand_JAS72, dobricic.pinardi.ea_OS07}.
Diffusion coefficients are defined as $A_D = \alpha e_{1t} e_{2t}$, where $\alpha = 0.2$.
The divergence damping is activated by assigning to \np{nn_divdmp}{nn\_divdmp} in the \nam{_asminc}{\_asminc} namelist
a value greater than zero.
This specifies the number of iterations of the divergence damping. Setting a value of the order of 100 will result in a significant reduction in the vertical velocity induced by the increments.

%% =================================================================================================
\section{Implementation details}
\label{sec:ASM_details}

Here we show an example \nam{_asminc}{\_asminc} namelist and the header of an example assimilation increments file on
the ORCA2 grid.

\begin{listing}
  \nlst{nam_asminc}
  \caption{\forcode{&nam_asminc}}
  \label{lst:nam_asminc}
\end{listing}

The header of an assimilation increments file produced using the NetCDF tool
\mbox{\textit{ncdump~-h}} is shown below

\begin{clines}
netcdf assim_background_increments {
dimensions:
        x = 182 ;
        y = 149 ;
        z = 31 ;
        t = UNLIMITED ; // (1 currently)
variables:
        float nav_lon(y, x) ;
        float nav_lat(y, x) ;
        float nav_lev(z) ;
        double time_counter(t) ;
        double time ;
        double z_inc_dateb ;
        double z_inc_datef ;
        double bckint(t, z, y, x) ;
        double bckins(t, z, y, x) ;
        double bckinu(t, z, y, x) ;
        double bckinv(t, z, y, x) ;
        double bckineta(t, y, x) ;

// global attributes:
                :DOMAIN_number_total = 1 ;
                :DOMAIN_number = 0 ;
                :DOMAIN_dimensions_ids = 1, 2 ;
                :DOMAIN_size_global = 182, 149 ;
                :DOMAIN_size_local = 182, 149 ;
                :DOMAIN_position_first = 1, 1 ;
                :DOMAIN_position_last = 182, 149 ;
                :DOMAIN_halo_size_start = 0, 0 ;
                :DOMAIN_halo_size_end = 0, 0 ;
                :DOMAIN_type = "BOX" ;
}
\end{clines}

\subinc{%% =================================================================================================
%% Backmatter
%% =================================================================================================

%% Bibliography
%% =================================================================================================

\phantomsection
\addcontentsline{toc}{chapter}{Bibliography}
\lohead{Bibliography}
\rehead{Bibliography}
\bibliography{../main/bibliography}

\clearpage

%% Indices
%% =================================================================================================

\phantomsection
\addcontentsline{toc}{chapter}{Indices}
\lohead{Indices}
\rehead{Indices}
\printindex[blocks]
\printindex[keys]
\printindex[modules]
\printindex[parameters]
\printindex[subroutines]

\clearpage

%% Glossary
%% =================================================================================================

%\phantomsection
%\addcontentsline{toc}{chapter}{Glossary}
%\lohead{Glossary}\rehead{Glossary}
%\printglossaries
}

\end{document}
