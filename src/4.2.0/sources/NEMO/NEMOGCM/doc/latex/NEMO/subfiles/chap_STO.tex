\documentclass[../main/NEMO_manual]{subfiles}

\begin{document}

\chapter{Stochastic Parametrization of EOS (STO)}
\label{chap:STO}

\chaptertoc

\paragraph{Changes record} ~\\

{\footnotesize
  \begin{tabularx}{\textwidth}{l||X|X}
    Release & Author(s) & Modifications \\
    \hline
    {\em   4.0} & {\em ...} & {\em ...} \\
    {\em   3.6} & {\em ...} & {\em ...} \\
    {\em   3.4} & {\em ...} & {\em ...} \\
    {\em <=3.4} & {\em ...} & {\em ...}
  \end{tabularx}
}

% \vfill
% \begin{figure}[b]
%% =================================================================================================
% \subsubsection*{Changes record}
% \begin{tabular}{l||l|m{0.65\linewidth}}
%    Release   & Author        & Modifications \\
%    {\em 4.0.1} & {\em C. Levy} & {\em 4.0.1 update}  \\
%    {\em 3.6} & {\em P.-A. Bouttier} & {\em initial version}  \\
% \end{tabular}
% \end{figure}

\clearpage

As a result of the nonlinearity of the seawater equation of state, unresolved scales represent a major source of uncertainties in the computation of the large-scale horizontal density gradient from the large-scale temperature and salinity fields. Following  \cite{brankart_OM13}, the impact of these uncertainties can be simulated by random processes representing unresolved T/S fluctuations. The Stochastic Parametrization of EOS (STO) module implements this parametrization.

As detailed in \cite{brankart_OM13}, the stochastic formulation of the equation of state can be written as:
\begin{equation}
  \label{eq:STO_eos_sto}
  \rho = \frac{1}{2} \sum_{i=1}^m\{ \rho[T+\Delta T_i,S+\Delta S_i,p_o(z)] + \rho[T-\Delta T_i,S-\Delta S_i,p_o(z)] \}
\end{equation}
where $p_o(z)$ is the reference pressure depending on the depth and,
$\Delta T_i$ and $\Delta S_i$ (i=1,m) is a set of T/S perturbations defined as
the scalar product of the respective local T/S gradients with random walks $\mathbf{\xi}$:
\begin{equation}
  \label{eq:STO_sto_pert}
  \Delta T_i = \mathbf{\xi}_i \cdot \nabla T \qquad \hbox{and} \qquad \Delta S_i = \mathbf{\xi}_i \cdot \nabla S
\end{equation}
$\mathbf{\xi}_i$ are produced by a first-order autoregressive process (AR-1) with
a parametrized decorrelation time scale, and horizontal and vertical standard deviations $\sigma_s$.
$\mathbf{\xi}$ are uncorrelated over the horizontal and fully correlated along the vertical.

%% =================================================================================================
\section{Stochastic processes}
\label{sec:STO_the_details}

There are many existing parameterizations based on autoregressive processes,
which are used as a basic source of randomness to transform a deterministic model into a probabilistic model.
The generic approach here is to a new STO module,
generating processes features with appropriate statistics to simulate these uncertainties in the model
(see \cite{brankart.candille.ea_GMD15} for more details).

In practice, at each model grid point,
independent Gaussian autoregressive processes~$\xi^{(i)},\,i=1,\ldots,m$ are first generated using
the same basic equation:

\begin{equation}
  \label{eq:STO_autoreg}
  \xi^{(i)}_{k+1} = a^{(i)} \xi^{(i)}_k + b^{(i)} w^{(i)} + c^{(i)}
\end{equation}

\noindent
where $k$ is the index of the model timestep and
$a^{(i)}$, $b^{(i)}$, $c^{(i)}$ are parameters defining the mean ($\mu^{(i)}$) standard deviation ($\sigma^{(i)}$) and
correlation timescale ($\tau^{(i)}$) of each process:

\begin{itemize}
\item for order~1 processes, $w^{(i)}$ is a Gaussian white noise, with zero mean and standard deviation equal to~1,
  and the parameters $a^{(i)}$, $b^{(i)}$, $c^{(i)}$ are given by:

  \[
    % \label{eq:STO_ord1}
    \left\{
      \begin{array}{l}
        a^{(i)} = \varphi \\
        b^{(i)} = \sigma^{(i)} \sqrt{ 1 - \varphi^2 }        \qquad\qquad\mbox{with}\qquad\qquad \varphi = \exp \left( - 1 / \tau^{(i)} \right) \\
        c^{(i)} = \mu^{(i)} \left( 1 - \varphi \right) \\
      \end{array}
    \right.
  \]

\item for order~$n>1$ processes, $w^{(i)}$ is an order~$n-1$ autoregressive process, with zero mean,
  standard deviation equal to~$\sigma^{(i)}$;
  correlation timescale equal to~$\tau^{(i)}$;
  and the parameters $a^{(i)}$, $b^{(i)}$, $c^{(i)}$ are given by:

  \begin{equation}
    \label{eq:STO_ord2}
    \left\{
      \begin{array}{l}
        a^{(i)} = \varphi \\
        b^{(i)} = \frac{n-1}{2(4n-3)} \sqrt{ 1 - \varphi^2 }
        \qquad\qquad\mbox{with}\qquad\qquad
        \varphi = \exp \left( - 1 / \tau^{(i)} \right) \\
        c^{(i)} = \mu^{(i)} \left( 1 - \varphi \right) \\
      \end{array}
    \right.
  \end{equation}

\end{itemize}

\noindent
In this way, higher order processes can be easily generated recursively using the same piece of code implementing
\autoref{eq:STO_autoreg}, and using successive processes from order $0$ to~$n-1$ as~$w^{(i)}$.
The parameters in \autoref{eq:STO_ord2} are computed so that this recursive application of
\autoref{eq:STO_autoreg} leads to processes with the required standard deviation and correlation timescale,
with the additional condition that the $n-1$ first derivatives of the autocorrelation function are equal to
zero at~$t=0$, so that the resulting processes become smoother and smoother as $n$ increases.

Overall, this method provides quite a simple and generic way of generating a wide class of stochastic processes.
However, this also means that new model parameters are needed to specify each of these stochastic processes.
As in any parameterization, the main issue is to tune the parameters using
either first principles, model simulations, or real-world observations.
The parameters are set by default as described in \cite{brankart_OM13}, which has been shown in the paper
to give good results for a global low resolution (2°) \NEMO\ configuration. where this parametrization produces a major effect on the average large-scale circulation, especilally in regions of intense mesoscale activity.
The set of parameters will need further investigation to find appropriate values
for any other configuration or resolution of the model.

%% =================================================================================================
\section{Implementation details}
\label{sec:STO_thech_details}

The code implementing stochastic parametrisation is located in the src/OCE/STO directory.
It contains three modules :
% \begin{description}

\mdl{stopar} : define the Stochastic parameters and their time evolution

\mdl{storng} : random number generator based on and including the 64-bit KISS (Keep It Simple Stupid) random number generator distributed by George Marsaglia

\mdl{stopts} : stochastic parametrisation associated with the non-linearity of the equation of
 seawater, implementing \autoref{eq:STO_sto_pert} so as specifics in the equation of state
 implementing \autoref{eq:STO_eos_sto}.
% \end{description}

The \mdl{stopar} module includes three public routines called in the model:

(\rou{sto\_par}) is a direct implementation of \autoref{eq:STO_autoreg},
applied at each model grid point (in 2D or 3D), and called at each model time step ($k$) to
update every autoregressive process ($i=1,\ldots,m$).
This routine also includes a filtering operator, applied to $w^{(i)}$,
to introduce a spatial correlation between the stochastic processes.

(\rou{sto\_par\_init}) is the initialization routine computing
the values $a^{(i)}, b^{(i)}, c^{(i)}$ for each autoregressive process,
as a function of the statistical properties required by the model user
(mean, standard deviation, time correlation, order of the process,\ldots).
This routine also includes the initialization (seeding) of the random number generator.

(\rou{sto\_rst\_write}) writes a restart file
(which suffix name is given by \np{cn_storst_out}{cn\_storst\_out} namelist parameter) containing the current value of
all autoregressive processes to allow creating the file needed for a restart.
This restart file also contains the current state of the random number generator.
When \np{ln_rststo}{ln\_rststo} is set to \forcode{.true.}),
the restart file (which suffix name is given by \np{cn_storst_in}{cn\_storst\_in} namelist parameter) is read by
the initialization routine (\rou{sto\_par\_init}).
The simulation will continue exactly as if it was not interrupted only
when \np{ln_rstseed}{ln\_rstseed} is set to \forcode{.true.},
\ie\ when the state of the random number generator is read in the restart file.\\

The implementation includes the basics for a few possible stochastic parametrisations including equation of state,
lateral diffusion, horizontal pressure gradient, ice strength, trend, tracers dynamics.
As for this release, only the stochastic parametrisation of equation of state is fully available and tested. \\

Options and parameters \\

The \np{ln_sto_eos}{ln\_sto\_eos} namelist variable activates stochastic parametrisation of equation of state.
By default it set to \forcode{.false.}) and not active.
The set of parameters is available in \nam{sto}{sto} namelist
(only the subset for equation of state stochastic parametrisation is listed below):

\begin{listing}
  \nlst{namsto}
  \caption{\forcode{&namsto}}
  \label{lst:namsto}
\end{listing}

The variables of stochastic paramtetrisation itself (based on the global 2° experiments as in \cite{brankart_OM13} are:

\begin{description}
\item [{\np{nn_sto_eos}{nn\_sto\_eos}:}]     number of independent random walks
\item [{\np{rn_eos_stdxy}{rn\_eos\_stdxy}:}] random walk horizontal standard deviation
  (in grid points)
\item [{\np{rn_eos_stdz}{rn\_eos\_stdz}:}]   random walk vertical standard deviation
  (in grid points)
\item [{\np{rn_eos_tcor}{rn\_eos\_tcor}:}]   random walk time correlation (in timesteps)
\item [{\np{nn_eos_ord}{nn\_eos\_ord}:}]     order of autoregressive processes
\item [{\np{nn_eos_flt}{nn\_eos\_flt}:}]     passes of Laplacian filter
\item [{\np{rn_eos_lim}{rn\_eos\_lim}:}]     limitation factor (default = 3.0)
\end{description}

The first four parameters define the stochastic part of equation of state.

\subinc{%% =================================================================================================
%% Backmatter
%% =================================================================================================

%% Bibliography
%% =================================================================================================

\phantomsection
\addcontentsline{toc}{chapter}{Bibliography}
\lohead{Bibliography}
\rehead{Bibliography}
\bibliography{../main/bibliography}

\clearpage

%% Indices
%% =================================================================================================

\phantomsection
\addcontentsline{toc}{chapter}{Indices}
\lohead{Indices}
\rehead{Indices}
\printindex[blocks]
\printindex[keys]
\printindex[modules]
\printindex[parameters]
\printindex[subroutines]

\clearpage

%% Glossary
%% =================================================================================================

%\phantomsection
%\addcontentsline{toc}{chapter}{Glossary}
%\lohead{Glossary}\rehead{Glossary}
%\printglossaries
}

\end{document}
