\documentclass[../main/NEMO_manual]{subfiles}

\begin{document}

% ================================================================
% INTRODUCTION
% ================================================================

\chapter{Introduction}

The Nucleus for European Modelling of the Ocean (\NEMO) is a framework of ocean related engines,
namely OPA \footnote{OPA: Oc\'{e}an PArall\'{e}lis\'{e} (french)} for the ocean dynamics and thermodynamics,
SI$^3$ \footnote{SI$^3$: Sea-Ice modelling Integrated Initiative} for the sea-ice dynamics and thermodynamics,
TOP \footnote{TOP: Tracer in the Ocean Paradigm} for the biogeochemistry 
(both transport (TRP) and sources minus sinks 
(PISCES \footnote{PISCES: Pelagic Interactions Scheme for Carbon and Ecosystem Studies})).
It is intended to be a flexible tool for studying the ocean and its interactions with the other components of
the earth climate system (atmosphere, sea-ice, biogeochemical tracers, ...) over
a wide range of space and time scales.
This manual provides information about the physics represented by the ocean component of \NEMO and
the rationale for the choice of numerical schemes and the model design.
For the use of framework, 
a guide which gathers the \texttt{README} files spread out in the source code can be build and 
exported in web or printable format (see \path{./doc/rst}).
An online version of the guide is also available on the 
\href{http://forge.ipsl.jussieu.fr/nemo}{\NEMO forge platform}.

The ocean component of \NEMO has been developed from the legacy of the OPA model, release 8.2, 
described in \citet{madec.delecluse.ea_NPM98}.
This model has been used for a wide range of applications, both regional or global, as a forced ocean model and 
as a model coupled with the sea-ice and/or the atmosphere.

This manual is organised in as follows.
\autoref{chap:PE} presents the model basics, \ie the equations and their assumptions,
the vertical coordinates used, and the subgrid scale physics.
This part deals with the continuous equations of the model
(primitive equations, with temperature, salinity and an equation of seawater).
The equations are written in a curvilinear coordinate system, with a choice of vertical coordinates
($z$, $s$, \zstar, \sstar, \ztilde, \stilde, and a mix of them).
Momentum equations are formulated in vector invariant or flux form.
Dimensional units in the meter, kilogram, second (MKS) international system are used throughout.

The following chapters deal with the discrete equations.
\autoref{chap:STP} presents the time domain.
The model time stepping environment is a three level scheme in which 
the tendency terms of the equations are evaluated either centered in time, or forward, or backward depending of 
the nature of the term.
\autoref{chap:DOM} presents the space domain.
The model is discretised on a staggered grid (Arakawa C grid) with masking of land areas.
Vertical discretisation used depends on both how the bottom topography is represented and whether 
the free surface is linear or not.
Full step or partial step $z$-coordinate or $s$- (terrain-following) coordinate is used with linear free surface 
(level position are then fixed in time).
In non-linear free surface, the corresponding rescaled height coordinate formulation (\zstar or \sstar) is used
(the level position then vary in time as a function of the sea surface heigh).
The following two chapters (\autoref{chap:TRA} and \autoref{chap:DYN}) describe the discretisation of
the prognostic equations for the active tracers and the momentum.
Explicit, split-explicit and filtered free surface formulations are implemented.
A number of numerical schemes are available for momentum advection, 
for the computation of the pressure gradients, as well as for the advection of tracers 
(second or higher order advection schemes, including positive ones).

Surface boundary conditions (\autoref{chap:SBC}) can be implemented as prescribed fluxes, or bulk formulations for 
the surface fluxes (wind stress, heat, freshwater).
The model allows penetration of solar radiation.
There is an optional geothermal heating at the ocean bottom.
Within the \NEMO system the ocean model is interactively coupled with a sea ice model (SI$^3$) and
a biogeochemistry model (PISCES).
Interactive coupling to Atmospheric models is possible via the \href{https://portal.enes.org/oasis}{OASIS coupler}.
Two-way nesting is also available through an interface to the AGRIF package
(Adaptative Grid Refinement in \fortran) \citep{debreu.vouland.ea_CG08}.

Other model characteristics are the lateral boundary conditions (\autoref{chap:LBC}).
Global configurations of the model make use of the ORCA tripolar grid, with special north fold boundary condition.
Free-slip or no-slip boundary conditions are allowed at land boundaries.
Closed basin geometries as well as periodic domains and open boundary conditions are possible.

Physical parameterisations are described in \autoref{chap:LDF} and \autoref{chap:ZDF}.
The model includes an implicit treatment of vertical viscosity and diffusivity.
The lateral Laplacian and biharmonic viscosity and diffusion can be rotated following
a geopotential or neutral direction.
There is an optional eddy induced velocity \citep{gent.mcwilliams_JPO90} with a space and time variable coefficient
\citet{treguier.held.ea_JPO97}.
The model has vertical harmonic viscosity and diffusion with a space and time variable coefficient,
with options to compute the coefficients with \citet{blanke.delecluse_JPO93}, \citet{pacanowski.philander_JPO81}, or 
\citet{umlauf.burchard_JMR03} mixing schemes.
 
%%gm    To be put somewhere else ....
%%nm    We should consider creating a glossary for all this kind of stuff (terms, acronyms and symbols)
%%      http://en.wikibooks.org/wiki/LaTeX/Glossary
\noindent CPP keys and namelists are used as inputs to the code.

\noindent \index{CPP keys} CPP keys

Some CPP keys are implemented in the \fortran code to allow code selection at compiling step.
This selection of code at compilation time reduces the reliability of the whole platform since
it changes the code from one set of CPP keys to the other.
It is used only when the addition/suppression of the part of code highly changes the amount of memory at run time.
Usual coding looks like: 

\begin{forlines}
#if defined key_option1
	! This part of the \fortran code will be active
	! only if key_option1 is activated at compiling step
#endif
\end{forlines}

\noindent \index{Namelist} Namelists

The namelist allows to input variables (character, logical, real and integer) into the code.
There is one namelist file for each component of NEMO (dynamics, sea-ice, biogeochemistry...)
containing all the \fortran namelists needed.
The implementation in NEMO uses a 2-step process. 
For each \fortran namelist, two files are read:

\begin{enumerate}
\item
  A reference namelist (in \path{./cfgs/SHARED/namelist_ref}) is read first.
  This file contains all the namelist variables which are initialised to default values
\item
  A configuration namelist (in \path{./cfgs/CFG_NAME/EXP00/namelist_cfg}) is read aferwards.
  This file contains only the namelist variables which are changed from default values, and overwrites those.
\end{enumerate}
A template can be found in \path{NEMO/OPA_SRC/module.example}.
The effective namelist, taken in account during the run, is stored at execution time in
an \texttt{output\_namelist\_dyn} (or \texttt{\_ice} or \texttt{\_top}) file.
%%gm  end

Model outputs management and specific online diagnostics are described in \autoref{chap:DIA}.
The diagnostics includes the output of all the tendencies of the momentum and tracers equations,
the output of tracers tendencies averaged over the time evolving mixed layer,
the output of the tendencies of the barotropic vorticity equation,
the computation of on-line floats trajectories...
\autoref{chap:OBS} describes a tool which reads in observation files
(profile temperature and salinity, sea surface temperature, sea level anomaly and sea ice concentration) 
and calculates an interpolated model equivalent value at the observation location and nearest model timestep.
Originally developed of data assimilation, it is a fantastic tool for model and data comparison.
\autoref{chap:ASM} describes how increments produced by data assimilation may be applied to the model equations.
Finally, \autoref{chap:CFG} provides a brief introduction to the pre-defined model configurations
(water column model, ORCA and GYRE families of configurations).

%%nm: Add some words on the NEMO dependencies
The model is implemented in \fninety, with preprocessing (C pre-processor).
It runs under UNIX.
It is optimized for vector computers and parallelised by domain decomposition with MPI.
All input and output is done in NetCDF (Network Common Data Format) with a optional direct access format for output.
To ensure the clarity and readability of the code it is necessary to follow coding rules.
The coding rules for OPA include conventions for naming variables,
with different starting letters for different types of variables (real, integer, parameter\ldots).
Those rules are briefly presented in \autoref{apdx:D} and a more complete document is available .

The model is organized with a high internal modularity based on physics.
For example, each trend (\ie, a term in the RHS of the prognostic equation) for momentum and tracers
is computed in a dedicated module.
To make it easier for the user to find his way around the code, the module names follow a three-letter rule.
For example, \mdl{traldf} is a module related to the TRAcers equation, computing the Lateral DiFfussion.
%The complete list of module names is presented in \autoref{apdx:D}.      %====>>>> to be done !
Furthermore, modules are organized in a few directories that correspond to their category,
as indicated by the first three letters of their name (\autoref{tab:chapters}).

The manual mirrors the organization of the model.
After the presentation of the continuous equations (\autoref{chap:PE}),
the following chapters refer to specific terms of the equations each associated with a group of modules 
(\autoref{tab:chapters}).

%--------------------------------------------------TABLE--------------------------------------------------
\begin{table}[!t]	
  \caption{
    \protect\label{tab:chapters}
    Organization of Chapters mimicking the one of the model directories.
  }
  \begin{center}
    \begin{tabular}{|l|l|l|}
      \hline
      \autoref{chap:STP}  & -   & model time STePping environment \\
      \hline
      \autoref{chap:DOM}  & DOM & model DOMain \\
      \hline
      \autoref{chap:TRA}  & TRA & TRAcer equations (potential temperature and salinity) \\
      \hline
      \autoref{chap:DYN}  & DYN & DYNamic equations (momentum) \\
      \hline
      \autoref{chap:SBC}  & SBC & Surface Boundary Conditions \\
      \hline
      \autoref{chap:LBC}  & LBC & Lateral Boundary Conditions (also OBC and BDY)  \\
      \hline
      \autoref{chap:LDF}  & LDF & Lateral DiFfusion (parameterisations) \\
      \hline
      \autoref{chap:ZDF}  & ZDF & vertical (Z) DiFfusion (parameterisations)  \\
      \hline
      \autoref{chap:DIA}  & DIA & I/O and DIAgnostics (also IOM, FLO and TRD) \\
      \hline
      \autoref{chap:OBS}  & OBS & OBServation and model comparison  \\
      \hline
      \autoref{chap:ASM}  & ASM & ASsiMilation increment  \\
      \hline
      \autoref{chap:MISC} & SOL & Miscellaneous  topics (including solvers)  \\
      \hline
      \autoref{chap:CFG}  & -   & predefined configurations (including C1D) \\
      \hline
    \end{tabular}
  \end{center}
\end{table}
%--------------------------------------------------------------------------------------------------------------

%% nm: the following section has to vastly remodeled to focus only on well-identified versions of NEMO
%% (3.4, 3.6, 4.0 and further releases). Then its formatting must be improved too.
\subsubsection{Changes between releases}

NEMO/OPA, like all research tools, is in perpetual evolution.
The present document describes the OPA version include in the release 3.4 of NEMO.
This release differs significantly from version 8, documented in \citet{madec.delecluse.ea_NPM98}. \\

The main modifications from OPA v8 and NEMO/OPA v3.2 are :

\begin{itemize}
\item
  transition to full native \fninety, deep code restructuring and drastic reduction of CPP keys; 
\item
  introduction of partial step representation of bottom topography
  \citep{barnier.madec.ea_OD06, le-sommer.penduff.ea_OM09, penduff.le-sommer.ea_OS07};
\item
  partial reactivation of a terrain-following vertical coordinate ($s$- and hybrid $s$-$z$) with
  the addition of several options for pressure gradient computation
  \footnote{
    Partial support of $s$-coordinate: there is presently no support for neutral physics in
    $s$-coordinate and for the new options for horizontal pressure gradient computation with
    a non-linear equation of state.
  }
  ;
\item
  more choices for the treatment of the free surface: full explicit, split-explicit or filtered schemes,
  and suppression of the rigid-lid option;
\item
  non linear free surface associated with the rescaled height coordinate \zstar or $s$;
\item
  additional schemes for vector and flux forms of the momentum advection;
\item
  additional advection schemes for tracers;
\item
  implementation of the AGRIF package (Adaptative Grid Refinement in \fortran) \citep{debreu.vouland.ea_CG08};
\item
  online diagnostics : tracers trend in the mixed layer and vorticity balance;
\item
  rewriting of the I/O management with the use of an I/O server;
\item
  generalized ocean-ice-atmosphere-CO2 coupling interface, interfaced with OASIS 3 coupler;
\item
  surface module (SBC) that simplify the way the ocean is forced and include two bulk formulea (CLIO and CORE) and
  which includes an on-the-fly interpolation of input forcing fields;
\item
  RGB light penetration and optional use of ocean color 
\item
  major changes in the TKE schemes: it now includes a Langmuir cell parameterization \citep{axell_JGR02},
  the \citet{mellor.blumberg_JPO04} surface wave breaking parameterization, and has a time discretization which
  is energetically consistent with the ocean model equations \citep{burchard_OM02, marsaleix.auclair.ea_OM08};
\item
  tidal mixing parametrisation (bottom intensification) + Indonesian specific tidal mixing
  \citep{koch-larrouy.madec.ea_GRL07};
\item
  introduction of LIM-3, the new Louvain-la-Neuve sea-ice model
  (C-grid rheology and new thermodynamics including bulk ice salinity)
  \citep{vancoppenolle.fichefet.ea_OM09*a, vancoppenolle.fichefet.ea_OM09*b}
\end{itemize}

The main modifications from NEMO/OPA v3.2 and v3.3 are:

\begin{itemize}
\item
  introduction of a modified leapfrog-Asselin filter time stepping scheme
  \citep{leclair.madec_OM09}; 
\item
  additional scheme for iso-neutral mixing \citep{griffies.gnanadesikan.ea_JPO98}, although it is still a "work in progress";
\item
  a rewriting of the bottom boundary layer scheme, following \citet{campin.goosse_T99};
\item
  addition of a Generic Length Scale vertical mixing scheme, following \citet{umlauf.burchard_JMR03};
\item
  addition of the atmospheric pressure as an external forcing on both ocean and sea-ice dynamics;
\item
  addition of a diurnal cycle on solar radiation \citep{bernie.guilyardi.ea_CD07};
\item
  river runoffs added through a non-zero depth, and having its own temperature and salinity;
\item
  CORE II normal year forcing set as the default forcing of ORCA2-LIM configuration;
\item
  generalisation of the use of \mdl{fldread} for all input fields (ocean climatology, sea-ice damping...);
\item
  addition of an on-line observation and model comparison (thanks to NEMOVAR project);
\item
  optional application of an assimilation increment (thanks to NEMOVAR project);
\item
  coupling interface adjusted for WRF atmospheric model;
\item
  C-grid ice rheology now available fro both LIM-2 and LIM-3 \citep{bouillon.maqueda.ea_OM09};
\item
  LIM-3 ice-ocean momentum coupling applied to LIM-2;
\item
  a deep re-writting and simplification of the off-line tracer component (OFF\_SRC);
\item
  the merge of passive and active advection and diffusion modules;
\item
  Use of the Flexible Configuration Manager (FCM) to build configurations,
  generate the Makefile and produce the executable;
\item
  Linear-tangent and Adjoint component (TAM) added, phased with v3.0
\end{itemize}

\vspace{1cm}

In addition, several minor modifications in the coding have been introduced with the constant concern of
improving the model performance.

The main modifications from NEMO/OPA v3.3 and v3.4 are:

\begin{itemize}
\item finalisation of above iso-neutral mixing \citep{griffies.gnanadesikan.ea_JPO98}";
\item "Neptune effect" parametrisation;
\item horizontal pressure gradient suitable for s-coordinate;
\item semi -implicit bottom friction;
\item finalisation of the merge of passive and active tracers advection-diffusion modules;
\item a new bulk formulae (so-called MFS);
\item use fldread for the off-line tracer component (OFF\_SRC);
\item use MPI point to point communications for north fold;
\item diagnostic of transport;
\end{itemize}

The main modifications from NEMO/OPA v3.4 and v3.6 are:

\begin{itemize}
 \item ...;
\end{itemize}

The main modifications from NEMO/OPA v3.6 and v4.0 are:

\begin{itemize}
\item new definition of configurations;
\item bulk formulation;
\item NEMO-wave large scale interactions;
\item ...;
\end{itemize}

\biblio

\pindex

\end{document}
